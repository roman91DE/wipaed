% Options for packages loaded elsewhere
\PassOptionsToPackage{unicode}{hyperref}
\PassOptionsToPackage{hyphens}{url}
%
\documentclass[
]{article}
\title{Corperate Governance deutscher Unternehmen}
\author{Roman Hoehn}
\date{}

\usepackage{amsmath,amssymb}
\usepackage{lmodern}
\usepackage{iftex}
\ifPDFTeX
  \usepackage[T1]{fontenc}
  \usepackage[utf8]{inputenc}
  \usepackage{textcomp} % provide euro and other symbols
\else % if luatex or xetex
  \usepackage{unicode-math}
  \defaultfontfeatures{Scale=MatchLowercase}
  \defaultfontfeatures[\rmfamily]{Ligatures=TeX,Scale=1}
\fi
% Use upquote if available, for straight quotes in verbatim environments
\IfFileExists{upquote.sty}{\usepackage{upquote}}{}
\IfFileExists{microtype.sty}{% use microtype if available
  \usepackage[]{microtype}
  \UseMicrotypeSet[protrusion]{basicmath} % disable protrusion for tt fonts
}{}
\makeatletter
\@ifundefined{KOMAClassName}{% if non-KOMA class
  \IfFileExists{parskip.sty}{%
    \usepackage{parskip}
  }{% else
    \setlength{\parindent}{0pt}
    \setlength{\parskip}{6pt plus 2pt minus 1pt}}
}{% if KOMA class
  \KOMAoptions{parskip=half}}
\makeatother
\usepackage{xcolor}
\IfFileExists{xurl.sty}{\usepackage{xurl}}{} % add URL line breaks if available
\IfFileExists{bookmark.sty}{\usepackage{bookmark}}{\usepackage{hyperref}}
\hypersetup{
  pdftitle={Corperate Governance deutscher Unternehmen},
  pdfauthor={Roman Hoehn},
  hidelinks,
  pdfcreator={LaTeX via pandoc}}
\urlstyle{same} % disable monospaced font for URLs
\usepackage[margin=1in]{geometry}
\usepackage{longtable,booktabs,array}
\usepackage{calc} % for calculating minipage widths
% Correct order of tables after \paragraph or \subparagraph
\usepackage{etoolbox}
\makeatletter
\patchcmd\longtable{\par}{\if@noskipsec\mbox{}\fi\par}{}{}
\makeatother
% Allow footnotes in longtable head/foot
\IfFileExists{footnotehyper.sty}{\usepackage{footnotehyper}}{\usepackage{footnote}}
\makesavenoteenv{longtable}
\usepackage{graphicx}
\makeatletter
\def\maxwidth{\ifdim\Gin@nat@width>\linewidth\linewidth\else\Gin@nat@width\fi}
\def\maxheight{\ifdim\Gin@nat@height>\textheight\textheight\else\Gin@nat@height\fi}
\makeatother
% Scale images if necessary, so that they will not overflow the page
% margins by default, and it is still possible to overwrite the defaults
% using explicit options in \includegraphics[width, height, ...]{}
\setkeys{Gin}{width=\maxwidth,height=\maxheight,keepaspectratio}
% Set default figure placement to htbp
\makeatletter
\def\fps@figure{htbp}
\makeatother
\setlength{\emergencystretch}{3em} % prevent overfull lines
\providecommand{\tightlist}{%
  \setlength{\itemsep}{0pt}\setlength{\parskip}{0pt}}
\setcounter{secnumdepth}{-\maxdimen} % remove section numbering
\ifLuaTeX
  \usepackage{selnolig}  % disable illegal ligatures
\fi

\begin{document}
\maketitle

\hypertarget{corperate-governance-deutscher-unternehmen}{%
\section{Corperate Governance deutscher
Unternehmen}\label{corperate-governance-deutscher-unternehmen}}

\hypertarget{grundlagen-der-corporate-governance}{%
\subsection{Grundlagen der Corporate
Governance}\label{grundlagen-der-corporate-governance}}

\hypertarget{prinzipal-agenten-theorie-und-corporate-governance}{%
\subsubsection{Prinzipal Agenten Theorie und Corporate
Governance}\label{prinzipal-agenten-theorie-und-corporate-governance}}

Ausgangssituation: Prinzipal beauftrag Agenten mit der Ausführung einer
Aufgabe, die mit der Gewährung von Entscheidungskompetenz verbunden ist

\begin{quote}
Prinzipal: Beauftragung des Agenten auf Vertrauensbasis Agent:
Versprechen der Aufgabenerfüllung gegenüber dem Prinzipal
\end{quote}

Warum Gewährung von Entscheidungskompetenz durch Prinzipal? 1. Agent ist
Spezialist, nicht Prinzipal! Zu genaue Vorgaben durch Prinzipal können
seinen Erfolg negativ beeinträchtigen 2. Komplexität der
Vertragsgestaltung bei zu genauen Vorgaben

Vorteile: * Prinzipal muss keine detaillierten Fachkenntnisse über
Themenbereich haben * Spezialisierte Agenten können für verschiedene
Aufgaben eingesetzt werden

Die Erteilung von Entscheidungskompetenz erfordert jedoch zwingend
Vertrauen von Prinzipal zu Agenten, dieses kann durch den Agenten
ausgenutzt/enttäuscht werden. Ohne Entscheidungskompetenz, wird der
Agent in seinen Handlungsspielraum stark eigeschränkt, eine genaue
vertragliche Spezifikation ist aufgrund des hohen Aufwands und
mangelndem Fachverständnis der Prinzipals oft nicht möglich.

Beispiele für P\&A Beziehungen:

\begin{itemize}
\tightlist
\item
  Autobesitzer zu Werkstatt
\item
  Patient zu Arzt
\item
  Aktionäre zu Vorstand
\end{itemize}

\hypertarget{prinzipal-agenten-probleme-bei-kapitalgesellschaften}{%
\paragraph{Prinzipal Agenten Probleme bei
Kapitalgesellschaften}\label{prinzipal-agenten-probleme-bei-kapitalgesellschaften}}

Kapitalgesellschaften kennzeichnen sich primär durch die Trennung von
Kapitalaufbringung und der Unternehmensleitung

Kapitalaufbringung: Aktionäre legen Mittel unternehmerisch an, ohne
dabei selbst unternehmerisch tätig zu werden

Rechte der Aktionäre: Stimmrecht auf Gesellschaftsversammlung,
Gewinnausschüttungen gemäß Gesellschafterbeschluss, Anteil am
Liquidationserlös bei Auflösung der Gesellschaft

Unternehmensleitung/ Vorstand: Können unternehmerisch tätig sein, ohne
dabei direkt Kapital für das Unternehmen einzubringen

Vorteile der Trennung von Kapitalaufbringung und Unternehmensleitung?

Kapitalaufbringung: Aktionäre/ Anteilseigner können durch
Diversifikationsstrategien effektiv unternehmerisch tätig sein/ Mittel
anlegen ohne dabei selbst direkt unternehmerisch tätig zu werden

Vorstände: Durch die fehlende Voraussetzung der Kapitalaufbringung kann
bei der Auswahl von Vorständen auf einen größeren Pool von potentiellen
Kandidaten zurückgegriffen werden, die Befähigung zur Übernahme der
Unternehmensleitung ist nicht von den individuellen finanziellen
Kapazitäten der Anwärter abhängig.

Vergleich der Gesellschaftsformen:

Gesellschaftsformen Personengesellschaften (OHG, KG, PartnerG)
\textbar{} Kapitalgesellschaften (GmbH, AG, KGaA, SE) --- \textbar{} ---
Eigentümergeführte Gesellschaften \textbar{} Trennung von Kapital und
Geschäftsführung --- \textbar{} --- Grundsätzlich keine
Haftungsbeschränkungen \textbar{} Grundsätzlich beschränkte Haftung ---
\textbar{} --- Weitgehend dispositives Recht(HGB)\footnote{Nachgiebiges
  Recht, Rechtsvorschriften können durch Vereinbarungen abgeändert
  werden. HGB Reglungen über Handelsgeschäfte sind zum Großteil dem
  dispositiven Recht zuzuordnen.} \textbar{} Weitgehens zwingendes Recht
(GmbHG, AktG) --------------------------------------------
--------------------------------------------

Personengesellschaften besitzten auf elementarer Ebene keine P\&A
Beziehung, da Unternehmensleitung und Finanzierung auf den gleichen
Personenkreis fällt.

Die Rolle des Vorstands als Agent:

Grundproblem: Gelingt es dem Prinzipal (=Kapitalanlegern) einen
optimalen Agenten (=Vorstand) auszuwählen?

\begin{itemize}
\tightlist
\item
  Informationsasymmetrie: Prinzipal kann die Eigenschaften des Agenten
  nicht optimal/ vollständig beobachten
\item
  Interessenskonflikt: Agent hat Anreize sich als optimalen Kandidaten
  zu präsentieren da seine Nutzenfunktion maßgeblich von seiner
  individuellen Entlohnung abhängt
\end{itemize}

Folgen des Grundproblems

\begin{enumerate}
\def\labelenumi{\arabic{enumi}.}
\tightlist
\item
  Adverse Selektion
\item
  Moral Hazard
\end{enumerate}

\hypertarget{adverse-selektion}{%
\paragraph{Adverse Selektion}\label{adverse-selektion}}

Der Prinzipal ist aufgrund des Grundproblems der Informationsasymmetrie
nur noch dazu bereit eine durchschnittliche Qualität der Bewerber zu
bezahlen. Aufgrund des niedrigen Durchschnittpreises scheiden
hochqualifizierte/ sehr gut geeignete Agenten aus dem Auswahlpool aus.

Prinzipale antizipieren die beschriebene Entwicklung wodurch die
Zahlungsbereitschaft weiter sinkt, über Zeit verbleiben nur noch
schlecht geeignete Agenten bzw. keine Agenten im Bewerberpool.

\textbf{Adverse Selektion ist eine negative Auslese, gute Agenten werden
vom Markt vertrieben!}

Wie lässt sich adverse Selektion verhindern?

Zwei Möglichkeiten:

\textbf{Informationsasymmetrien reduzieren} * Signaling durch Agenten
(z.B. Hochschulabschluss oder Arbeitszeugnisse) oder Screenings des
Agenten durch den Prinzipal (z.B. Assesment Center, Arbeitsprobe)

\begin{quote}
Screening: Anstrengungen des Prinzipals zur besseren Einschätzung des
Agenten Signaling: Anstrengungen des Agenten, um seine wahren
Eigenschaften glaubwürdig an Prinzipal zu vermitteln
\end{quote}

Sowohl Screening als auch Signaling sind nur dann effizient, wenn
schlechte Agenten die Eigenschaften von guten Agenten nicht ohne
größeren Aufwand imitieren können

\textbf{Interessenskonflikte reduzieren} Vertragsgestaltung/ Self
Selection, der Agent offenbart seine eigenen Eigenschaften durch die
Vertragsauswahl, nur ein geeigneter Agent ist z.B. dazu bereit eine
erfolgsabhängige Vergütung anzunehmen (alternativ die Gewährleistung bei
Schlechterfüllung)

\hypertarget{rechenbeispiel}{%
\paragraph{Rechenbeispiel}\label{rechenbeispiel}}

Auswahl zwischen Agenten A und B

Agent A bringt Prinzipal bei Beauftragung einen brutto Nutzen von 400,
Agent B 200

Mindestvergütung von Agent A: 350

Mindestvergütung von Agent B: 200

\hypertarget{vollstuxe4ndige-information}{%
\subparagraph{Vollständige
Information}\label{vollstuxe4ndige-information}}

\begin{verbatim}
## Netto Nutzen Prinzipal bei Beauftragung von Agent A: 50
## Netto Nutzen Prinzipal bei Beauftragung von Agent B: 0
## Auswahl der Prinzipals: Agent A
\end{verbatim}

\textbf{Ohne Informationsasymmetrien (=Prinzipal kann zwischen A und B
unterscheiden)}

Beauftragung von A

U(A)=400 -- 350 = 50 (=Nettonutzen kann zwischen A und Prinzipal
aufgeteilt werden)

U(B) = 200 -- 200 = 0\\
U(B)\textless{} U(A)

\hypertarget{unvollstuxe4ndige-information}{%
\subparagraph{Unvollständige
Information}\label{unvollstuxe4ndige-information}}

\begin{verbatim}
## Agent A wird vom Markt verdrängt
## Agent B ist der letzte verbleibende Agent am Markt!
\end{verbatim}

\textbf{Mit Informationsasymmetrien}

Prinzipal bildet Erwartungsnutzen

U(A/B) = 400/2 + 200/2 = 300

Maximale ZB des Prinzipals wird von Agenten antizipiert, da kleiner als
Reservationslohn von Agent A verlässt dieser den Markt

Prinzipal antizipiert wiederum

U(B)= 200 -200 = 0

Der Rückgang des Nettonutzens von 50 (vollständige Information) auf 0
(unvollständige Information) entspricht den Agency Kosten i.H.v. 50

\hypertarget{moral-hazard}{%
\paragraph{Moral Hazard}\label{moral-hazard}}

Erbringt der ausgewählte Agent aus Sicht des Prinzipals einen optimalen
Arbeitseinsatz?

Informationsasymmetrie: Prinzipal kann den Arbeitseinsatz des Agenten
nicht perfekt beobachten

Interessenskonflikt: Agenten können den Anreiz besitzen, ihren
Arbeitseinsatz möglichst gering zu halten (z.B. Freizeit, andere
Arbeitgeber)

Folgen des Moral Hazard Problems:

\begin{itemize}
\tightlist
\item
  Agenten erbringen einen geringeren Arbeitseinsatz als von Prinzipal
  erwartet
\item
  Prinzipale antizipieren dies wiederum und sind nur noch dazu bereit
  eine niedrigere Vergütung anzubieten
\end{itemize}

Wie kann Moral Hazard verhindert werden?

\begin{itemize}
\tightlist
\item
  Informationsasymmetrien reduzieren: Ermittlung von Maßen, die einen
  Rückschluss auf den Arbeitseinsatz des Agenten erlauben (Erfolgsmaße
  oder sonstige Kennzahlen)
\item
  Interessenskonflikte verringern:

  \begin{enumerate}
  \def\labelenumi{\arabic{enumi}.}
  \tightlist
  \item
    Agent wird erfolgsabhängig vergütet,
  \item
    Agent wird bei Erfolg befördert und bei Misserfolg gekündigt.
  \end{enumerate}
\end{itemize}

Anwendung: Moral Hazard bei Kapitalgesellschaften

Informationsasymmetrien bestehen grundsätzlich: Qualität der
Vorstandsarbeit ist häufig nicht/ nur schwer messbar (z.B. wurde die
Strategieentwicklung sorgfältig durchdacht und optimiert?)

Interessenskonflikte: Bestehen ebenfalls, Aktionäre wünschen sich
maximale Wertsteigerung während Vorstand andere Interessen verfolgen
kann

\hypertarget{moral-hazard---weitere-vorstandsspezifische-eigeninteressen}{%
\subparagraph{Moral Hazard - Weitere (vorstandsspezifische)
Eigeninteressen}\label{moral-hazard---weitere-vorstandsspezifische-eigeninteressen}}

\textbf{Geringerer oder fehlgeleiteter Arbeitseinsatz} * Quiet life:
Unangenehme Tätigkeiten, wie z.B. das Controlling werden vernachlässigt
* Unternehmensaufgaben werden zu Gunsten von unternehmensexternen
Aufgaben vernachlässigt (z.B. Aufsichtsrat in anderer AG)

\textbf{Selbstbereicherung} * Nebenleistungen werden ausgenutzt (z.B.
Firmenjet Nutzung), Privatnutzen für Vorstand wird trotz negativer
Auswirkung auf Unternehmen durch Kosten höher bewertet * Insiderhandel
und/oder Unterschlagung von Firmenmitteln

\textbf{Streben nach Prestige} * Empire Building: Nicht profitable
Investitionen oder Firmenübernahmen zur Prestigezwecken (negative
Auswirkungen auf Profitabilität werden ignoriert und das Wachstum zum
Prestigezwecken wird höher bewertet -- z.B. Firmenübernahmen, die i.d.R.
den Aktienkurs negativ beeinflussen, werden aus Prestigezwecken
durchgeführt) * Übermäßig hohe Spenden

\textbf{Zementierung der eigenen Position (= entrenchment)} * Abwehr von
Übernahmen, um die eigene Position zu schützen (auch wenN diese aus
Sicht der Aktionäre positiv zu bewerten sind) * Bilanzpolitik (Schlechte
Performance wird verschleiert, um die eigene Position nicht zu
gefährden) * Bereiche stärken, die die individuellen Stärken des
Vorstands wiederspiegeln. Der Vorstand ordnet Unternehmensziele seinem
persönlichen Stärken unter

Anwendung von Moral Hazard und Adverser Selektion auf Fall mit AG

\textbf{Adverse Selektion:} Prüfe ob bisheriger Vorstand die
Eigenschaften aufweist, die für die Aufgabenerfüllung in der neuen
Periode notwendig sind.

Screening: Mithilfe von zugänglichen Informationen versuchen
Rückschlüsse auf die Eigenschaften des Vorstands zu ziehen (bisherige
Performance analysieren und bewerten, wichtig ist zu berücksichtigen, ob
externe Umstände einen signifikanten Einfluss hatten)

Signaling: Vorstandsmitglieder können versuchen benötigte Eigenschaften
glaubhaft gegenüber den Aktionären zu signalisieren (Erfahrungen,
Zertifikate)

Vertragsgestaltung: Vergütung einführen, die die geforderten
Eigenschaften für neue Aufgaben stärker berücksichtigt

\textbf{Moral Hazard}: Prüfe das Bestehen von Informationsasymmetrien
und/oder Interessenskonflikten?

Informationsasymmetrien i.d.R. nicht perfekt beobachtbarer
Arbeitseinsatz des Vorstands (Arbeitszeit ist wenig aussagekräftig,
genaue Planung und Analyse von Unternehmensentscheidungen häufig nicht
möglich -- z.B. alle relevanten Szenarien durchdacht?)

Prüfung des Arbeitseinsatz kann anhand der aufgewendeten Zeit für
unliebsame Arbeiten (Controlling, Umsetzung) erfolgen

Prüfung auf Prestigestreben, Entrenchment, etc.

Prüfung ob Moral Hazard Probleme, falls vorhanden, durch
Vertragsgestaltung und erfolgsabhängige Entlohnung minimiert werden
könnte!

\hypertarget{rolle-des-managers-vorstand-agent-nach-adam-smith}{%
\paragraph{Rolle des Managers (=Vorstand/ Agent) nach Adam
Smith}\label{rolle-des-managers-vorstand-agent-nach-adam-smith}}

\begin{itemize}
\tightlist
\item
  Manager trifft finanzielle Entscheidungen über Mittel, die er nicht
  selbst investiert hat. Daraus resultierend kann man nicht davon
  ausgehen, dass der Manager finanzielle Entscheidungen mit der gleichen
  Sorgfalt trifft, die er bei eigenen finanziellen Mitteln verwenden
  würde
\end{itemize}

Aus dem Grundproblem der Prinzipal-Agenten Theorie (Adverse Selektion,
Moral Hazard) ergibt sich für den Prinzipal die folgenden 3 Aufgaben

\begin{itemize}
\tightlist
\item
  Sorgfältige Auswahl von Agenten (=Vorstand)
\item
  Effektive Überwachung des Agenten (=Vorstand)
\item
  Kontrolle/ Steuerung des Agenten (=Vorstand) durch Anreizsetzung
\end{itemize}

\hypertarget{wer-handelt-aber-als-prinzipal}{%
\subsection{Wer handelt aber als
Prinzipal?}\label{wer-handelt-aber-als-prinzipal}}

\begin{itemize}
\tightlist
\item
  Kleinaktionäre
\item
  Großaktionäre
\item
  Aufsichtsrat
\item
  Staat
\end{itemize}

\textbf{Kleinaktionäre}: Aufgrund von mangelnden Informationsgrundlagen,
Fachkompetenz und geringen Anreizen können Kleinaktionäre die Rolle des
Prinzipals nicht oder nur sehr begrenzt ausüben.

\begin{itemize}
\tightlist
\item
  Anteilsbesitz und Kontrolle fallen i.d.R. auseinander
\end{itemize}

Bei fehlender Kontrolle des Agenten kann es schnell zu einer Situation
kommen, in der der Vorstand sich selbst kontrolliert
(=\textbf{managerial hegemony})

\textbf{Aufsichtsrat:}

Hat i.d.R.:

\begin{itemize}
\tightlist
\item
  Breitere Informationsgrundlage
\item
  Höhere Fachkompetenz
\item
  Anreize können jedoch problematisch sein!
\end{itemize}

Das Problem des AR ist jedoch, dass der AR selbst ein Agent im Auftrag
der Aktionäre ist!

Der AR unterliegt als Agent der Aktionäre selbst den Gefahren der
Prinzipal Agenten Theorie!

Moral Hazard und Adverse Selektion können die Arbeit des AR gefährden

Aktionäre Aufsichtsrat Vorstand

\begin{itemize}
\tightlist
\item
  Beauftragung
\end{itemize}

Vorstand Aufsichtsrat Aktionäre

\begin{itemize}
\tightlist
\item
  Versprechen der Auftragserfüllung
\end{itemize}

Aufgaben des AR

Beauftragung des Vorstands; Aufgabenerfüllung (=Überwachung des
Vorstands) für Aktionäre

\textbf{Großaktionäre als Überwachungsogan?}

Prinzipiell besser geeignet als Kleinaktionäre, da

\begin{itemize}
\tightlist
\item
  Höhere Motivation durch höhere Beteiligung und weniger Diversifikation
\item
  Grundsätzlich gleiche Informationsansprüche aber häufig in engerem
  Kontakt zu Vorstand
\item
  Häufig auch mit besserer Fachkenntnis als Kleinaktionäre
\end{itemize}

\hypertarget{was-ist-corporate-governance}{%
\subsection{Was ist Corporate
Governance?}\label{was-ist-corporate-governance}}

Welche Mechanismen wirken auf den Vorstand, welche Kontrollsysteme gibt
es im Rahmen der CG? Corporate Governance kann als ein System verstanden
werden, dass Unternehmen mit Hilfe von mehreren Mechanismen überwacht
und deren Führung beeinflusst

\begin{enumerate}
\def\labelenumi{\arabic{enumi}.}
\tightlist
\item
  Rechtliche Vorschriften, Kodizes und Normen
\item
  Wettbewerbskräfte
\item
  Stakeholder Einfluss (Kreditgeber, AN, Lieferanten, Kunden und
  Öffentlichkeit)
\item
  Einfluss des AR \footnote{Wahl des Aufsichtsrats bei AG und KGaA auf
    Hauptversammlung, bei GmbH auf der Gesellschaftsversammlung}
\item
  Einfluss der Aktionäre
\item
  Einfluss durch Vergütungssystem (durch AR und Anteilseigner
  mitbestimmt)
\item
  Transparenz im Rahmen der Rechnungslegung, der Wirtschaftsprüfung, der
  Arbeit von Finanzanalysten sowie der Berichterstattungen durch Medien
\end{enumerate}

\textbf{Allgemeine Definition nach DCGK 2019:}

„Unter Corporate Governance wird der rechtliche und faktische
Ordnungsrahmen für die Leitung und

Überwachung eines Unternehmens verstanden. „

\textbf{Zweckgerichtete Definitionen:}

„The ways in which suppliers of finance to corporations assure
themselves of getting a return on their investment. ``(Schleifer \&
Vishny 1997)

„Corporate Governance is the design of institutions that induce or force
management to internalize the welfare of stakeholders. '' (Tirole 2001)

Im Fokus steht, dass Agenten im Sinne der Prinzipale handeln!

\hypertarget{deutscher-cg-kodex-dcgk}{%
\section{Deutscher CG Kodex (=DCGK)}\label{deutscher-cg-kodex-dcgk}}

Hintergrund: Hoher Bedarf an internationalem Kapital für deutsche
Unternehmen in den 1990er Jahren; Schwierigkeiten der Kapitalbeschaffung
durch deutsche Besonderheiten (z.B. betriebliche Mitbestimmung) und
Unternehmensskandale (z.B. Flowtex) am internationalen Markt

Veröffentlichung des DCGK im Jahre 2002 zur Schaffung von Vertrauen und
Transparenz in deutsche börsennotierte Unternehmen

Ziel des DCGK

\textbf{Ziel}: Das System der Corporate Governance in Deutschland soll
transparent und nachvollziehbar gemacht werden

\textbf{Inhalt}: Grundsätze, Empfehlungen und Anregungen zur Leitung und
Überwachung deutscher börsennotierter Gesellschaften, die national und
international als Standards guter und verantwortungsvoller
Unternehmensführung anerkannt sind

\textbf{Erklärtes Ziel}: Schaffung von Vertrauen durch Transparenz (bei
Anlegern, Kunden, Belegschaft, Öffentlichkeit\ldots)

\textbf{Was macht des DCGK?} Wiedergabe von relevanten gesetzlichen
Vorschriften, Entwicklung von Best-Practice Standards

Prinzip: \textbf{\emph{Transparenz}} schaffen, Verbesserung durch
\textbf{\emph{Selbstregulierung}} und \textbf{\emph{Flexibilisierung}}

Wo muss Transparenz geschaffen werden? Überall dort Verhaltensspielräume
bestehen und nicht bereits durch andere Umstände bestimmte
Verhaltensweisen zu erwarten sind (z.B. durch Wettbewerbsdruck)

Idee: Wettbewerbsdruck?

zu schwach

Gesetzliche Reglungen?

fehlend/ vage

Gesellschaftliche Normen?

fehlend/ unklar

Verhaltenskodizes!

Beispiel

Falls ein Unternehmen im Wettbewerb mit seiner Konkurrenz steht, wird es
dadurch bereits zu einer optimalen Ressourcenallokation gedrängt. Eine
Gesetzliche Reglung ist überflüssig

Ein Unternehmen, dass nicht im (vollständigen) Wettbewerb steht kann
jedoch Freiräume ausnutzen, hier stellt sich die Frage ob gesetzliche
Regulierungen notwendig sind? Gesetze müssen jedoch sparsam erlassen
werden da hohe Belastung für das Justizsystem, auch ist es oft nicht
möglich alle Entscheidungsfreiräume umfassend gesetzlich zu regulieren

Keine gesetzlichen Vorgaben? Gibt es etablierte gesellschaftliche
Normen? Diese können ebenfalls Rückschlüsse auf das zu erwartende
Verhalten geben (was ist sozial wünschenswertes Verhalten?)

Falls auf allen drei Ebenen Unklarheit herrscht, kann durch die
Festlegung eines Verhaltenskodex Transparenz erschaffen werden. Wichtig
ist, dass Transparenz über das Verhalten der Unternehmensleitung nur
dann entsteht, wenn ein abweichendes Verhalten vom Kodex auch
sanktioniert wird (Verstöße gegen DCGK werden i.d.R. mit Kursverlusten
sanktioniert)

\hypertarget{beispiel-altersbegrenzung-von-ar-mitgliedern}{%
\subsection{Beispiel: Altersbegrenzung von AR
Mitgliedern}\label{beispiel-altersbegrenzung-von-ar-mitgliedern}}

-Wettbewerbsdruck besteht nicht (keine Wettbewerbsnachteile)

-Gesetze nicht in Kraft, problematisch da spezifische Reglungen erlassen
werden müssten für verschiedene Branchen/ Funktionen

-Gesellschaftliche Normen für einen maximales Alter eistieren ebenfalls
nicht

Die Erlassung eines Höchstalters für AR Mitglieder in einem
Verhaltenskodex kann also das Vertrauen der Anleger/ Stakeholder durch
Schaffung von Transparenz erhöhen (falls vertrauenswürdig und
Sanktionierung von abweichendem Verhalten)

Fazit: Kodizes können Unsicherheiten über zu erwartendes Verhalten durch
die Schaffung von Transparenz reduzieren und somit Nutzen stiften. Damit
dies gelingt, müssen Freiräume durch imperfekten Wettbewerb entstehen
sowie klare gesetzliche oder soziale Reglungen nicht vorhanden sein.

Durch die freiwillige Bindung an einen Kodex kann ein Unternehmen
demnach Vertrauen der Anleger gewinnen

\hypertarget{anwendung-nichtbeachtung-einer-dcgk-empfehlung}{%
\subsection{Anwendung: Nichtbeachtung einer DCGK
Empfehlung?}\label{anwendung-nichtbeachtung-einer-dcgk-empfehlung}}

Situation: AG beachtet Empfehlung nicht

Da das erklärte Ziel der DCGK Empfehlungen darin besteht, Vertrauen
darüber zu schaffen, welches Verhalten Anleger von der
Unternehmensleitung erwarten können, wird durch Nichtbeachtung die
gewünschte Transparenz gemindert

Nichtbeachtung durch AG zeigt zusätzlich, dass das Unternehmen sich
eventuell nicht an DCGK Empfehlungen gebunden fühlt, eventuell kann die
Abweichung auch durch fehlende Angst vor Sanktionen begründet werden.
Das Vertrauen der Anleger in eine zukünftige Beachtung der AG wird unter
der Abweichung höchstwahrscheinlich leiden

Begründung für Abweichung? Falls diese plausibel und für die Anleger
überzeugend begründet ist, hält sich der Vertrauensverlust
möglicherweise in Grenzen bzw. bleibt aus. Die Begründung sollte daher
genau untersucht werden!

\hypertarget{flexibilisierung}{%
\subsection{Flexibilisierung}\label{flexibilisierung}}

Der DCGK soll es ermöglichen branchenspezifische bzw.
unternehmensspezifische Besonderheiten zu berücksichtigen. Der Kodex
erreicht die gewünschte \textbf{Flexibilität} durch die Möglichkeit der
individuellen Ausgestaltung der Corporate Governance anhand von
zusätzlichen freiwilligen Empfehlungen neben den regulären Empfehlungen

Soll Vorschriften -- Empfehlungen

Sollte Vorschriften -- Anregungen

Das Ziel der Flexibilisierung anhand von Soll und Sollte Vorschriften
steht jedoch durchaus im Kontrast zum eigentlichen Ziel des DCGK, der
Schaffung von Vertrauen durch Transparenz. Eine Abweichung von
freiwilligen Empfehlungen führt dazu, dass das Verhalten der
Unternehmensführung schlechter eingeschätzt werden kann

Nicht Beachtung von DCGK Empfehlungen kann bei plausibler Begründung im
Einklang mit dem Konzept der Flexibilisierung stehen!

\hypertarget{selbstregulierung}{%
\subsection{Selbstregulierung}\label{selbstregulierung}}

Ziel: Zur Selbstregulierung der deutschen Wirtschaft beitragen

Corporate Governance Praxis soll primär von Unternehmensvertretern und
Stakeholdern entwickelt werden, der Gesetzgeber soll möglichst wenig
eingreifen

Vermeidung des staatlichen Eingriffs durch erfolgreiche
Selbstregulierung als Ziel

Ausgangslage für gesetzliche Intervention bzw. Selbstregulierung ist
i.d.R. zunehmender öffentlicher Druck auf eine spezifische Praxis (z.B.
zu hohe Vorstands Gehälter)

Die Kommission des DCGK soll in dieser Situation versuchen frühzeitig
aktiv zu werden und den öffentlichen Druck durch die Erlassung von
Empfehlungen zu entschärfen damit der Gesetzgeber nicht in Reaktion auf
den öffentlichen Druck gesetzlich intervenieren muss

\textbf{Erfolgreiche Selbstregulation}

Öffentlicher Druck Erlassung von Soll Vorschrift Unternehmen setzen Soll
Vorschrift häufig um Öffentlicher Druck nimmt wieder ab Keine
gesetzlichen Reglungen erforderlich

\textbf{Gescheiterte Selbstregulation}

Nur begrenzte Umsetzung der Soll Vorschriften durch Unternehmen/
Umsetzung erfolgt aber wird als zu schwach angesehen Öffentlicher Druck
nimmt nicht ab Gesetzliche Intervention

Probleme: Soll Vorschriften können bei frühzeitiger Erlassung die
Öffentlichkeit erst recht auf Problem aufmerksam machen, keine Abnahme
des öffentlichen Drucks bei geringer Umsetzung durch Unternehmen oder
als zu lasch empfundenen Soll Vorschriften

\hypertarget{aufbau-und-bedeutung-des-dcgk}{%
\section{Aufbau und Bedeutung des
DCGK}\label{aufbau-und-bedeutung-des-dcgk}}

Bestandteile des DCGK

\begin{itemize}
\tightlist
\item
  Gesetzeswiedergabe;
\item
  Empfehlungen;
\item
  Anregungen
\end{itemize}

Gesetzeswiedergabe

Alle wesentlichen gesetzlichen Reglungen, die das System der deutschen
Corporate Governance mitbestimmen, werden im DCGK in Form von
Grundsätzen wiedergegeben.

Wesentliche Rechtsquelle: \textbf{Aktiengesetz} (zusätzlich GmbHG;
SE-Ausführungsgesetz)

Wesentliche Bestandteile: Rechte und Pflichten von Vorstand,
Aufsichtsrat und Hauptversammlung

Wichtig: Geltungswirkung dieser Grundsätze ergibt sich einzig und allein
aus dem Gesetzestext selbst, die Aufführung im DCGK dient lediglich der
Wiedergabe

Ziel: Transparente und nachvollziehbare Darstellung des deutschen CG
Systems

\hypertarget{empfehlungen-soll}{%
\subsection{Empfehlungen (=SOLL)}\label{empfehlungen-soll}}

Wichtigster Bestandteil des DCGK, die knapp 100 Empfehlungen sind durch
das Wort „\textbf{Soll''} gekennzeichnet

Es besteht prinzipiell keine Befolgungspflicht für Unternehmen, falls
eine Empfehlung nicht umgesetzt wird, muss diese jedoch in der
Entsprechenserklärung benannt und der Grund der Nichteinhaltung
erläutert werden

„\textbf{Comply or Explain}'' Prinzip

Keine Befolgungspflicht aber DCGK Empfehlungen gelten als
\textbf{anerkannte nationale und internationale Standards guter und
verantwortungsvoller Unternehmensführung}

** Keine Rechtsnorm und kein Handelsbrauch nach HGB**

Die Nichteinhaltung könnte als Verstoß gegen die allgemeine
Sorgfaltspflicht von Vorstand und/oder Aufsichtsrat ausgelegt werden
(insbesondere bei nicht Einhaltung einer Vielzahl von Empfehlungen)

Kein automatischer Verstoß und Prüfung muss auf Einzelfallbasis erfolgen

\textbf{Gesetzliche Pflicht für Unternehmen}: In der
Entsprechenserklärung muss jährlich erklärt werden, ob den Empfehlungen
entsprochen wurde und falls Abweichungen bestehen warum diese nicht
befolgt wurden

\emph{§161 AktG}

\emph{1)Vorstand und Aufsichtsrat der börsennotierten Gesellschaft
erklären jährlich, dass den \[\...\] Empfehlungen der
„Regierungskommission DCGK'' entsprochen wurde und wird oder welche
Empfehlungen nicht angewendet wurden oder werden und warum nicht.
\[\...\] }

\emph{(2) Die Erklärung ist auf der Internetseite der Gesellschaft
dauerhaft öffentlich zugänglich zu machen.}

Beispiele für Empfehlungen (Auszug)

\begin{itemize}
\tightlist
\item
  Unabhängigkeit des AR (Indikatoren, Empfehlungen zur Anzahl von
  unabhängigen Mitgliedern)
\item
  Vorstandsvergütung (Festlegung von Ziel- und Maximalvergütungen,
  Anteil der langfristigen variablen Vergütung)
\item
  Einrichtung von Compliance Management System
\item
  Erarbeitung von Kompetenzprofilen für AR Mitglieder
\end{itemize}

\hypertarget{anregungen-sollte}{%
\subsection{Anregungen (=SOLLTE)}\label{anregungen-sollte}}

Anregungen werden im DCGK durch das Wort „\textbf{Sollte''}
gekennzeichnet (weniger als 10 Anregungen aktuell enthalten)

Keine gesetzliche Pflicht bei Nichtbeachtung von Anregungen darüber in
der Entsprechenserklärung zu berichten und ein Abweichen zu begründen
(ebenfalls keine gesetzliche Pflicht zur Befolgung)

Die Nichtbeachtung von Anregungen stellt keine Pflichtverletzung der
leitenden Unternehmensorgane dar

Anwendung bei Nichtbeachtung

Es besteht keine gesetzliche Pflicht die Nichtbeachtung einer Anregung
zu benennen und zu begründen, die Nichtbeachtung der Anregung selbst
stellt ebenfalls keine Pflichtverletzung durch die Unternehmensorgane
dar.

Beispiele für Anregungen

\begin{itemize}
\tightlist
\item
  Anregungen zur Fixvergütung von AR Mitgliedern
\item
  Kommunikation des AR mit Investoren
\item
  Dauer der Hauptversammlung
\end{itemize}

Formulierung: „Der Aufsichtsratsvorsitzende sollte in angemessenem
Rahmen bereit sein, mit Investoren über aufsichtsratsspezifische Themen
Gespräche zu führen.''

\hypertarget{entwicklung-von-dcgk-vorschriften-reglungssetzung}{%
\section{Entwicklung von DCGK Vorschriften --
Reglungssetzung}\label{entwicklung-von-dcgk-vorschriften-reglungssetzung}}

\hypertarget{regierungskommission-des-dcgk}{%
\subsection{Regierungskommission des
DCGK}\label{regierungskommission-des-dcgk}}

Der Vorsitzende wird durch das Bundesministerium für Justiz und
Verbraucherschutz ernannt (erstmalig 2001), die Auswahl und Abberufung
von neuen Kommissionsmitgliedern wird in der Kommission im Plenum
diskutiert und anschließend vom Vorsitzenden an das BMJV weitergeleitet

Kritik am Auswahlverfahren:

\begin{itemize}
\tightlist
\item
  Die Auswahlkriterien für Mitglieder/ Vorsitzenden sind nicht
  transparent (lediglich der berufliche Hintergrund ist vorgegeben,
  siehe unten)
\item
  Die Auswahl wird nicht öffentlich begründet
\item
  Keine öffentliche Ausschreibung
\end{itemize}

Problematisch da DCGK Empfehlungen rechtlich relevant sind

Tätigkeitsdauer: 4 Jahre nach Berufung mit zweimaliger Möglichkeit zur
Wiederbestellung (maximal 12 Jahre möglich)

Art der Tätigkeit: Persönliches Ehrenamt \textbf{ohne}
Aufwandsentschädigung oder Honorar

Finanzierung der Kommission: Kosten für Reisen, Organisation, Verwaltung
etc. wird durch das Deutsche Aktieninstitut e.V. getragen

Die Kommission (zwischen 10 und 15 Mitglieder) setzt sich aus
\textbf{Vertretern} der folgenden Bereiche zusammen:

\begin{itemize}
\tightlist
\item
  Vorständen und Aufsichtsräten kapitalmarktorientierter Unternehmen
  (Großteil)
\item
  Institutionellen/ private Investoren
\item
  Rechts- und Wirtschaftswissenschaftler
\item
  Gewerkschaften
\item
  Wirtschaftsprüfer
\end{itemize}

Aufgabe der Regierungskommission: Deutscher CG Kodex soll mit dem Ziel
der Selbstregulierung ständig aktualisiert und weiterentwickelt werden

Änderungen und Überprüfung auf jährlicher Basis

Ergänzungen / Änderungen werden in einem öffentlichen
Konsultationsverfahren erstellt, die Veröffentlichung geschieht durch
das BMJV über den Bundesanzeiger (BMJV kann Veröffentlichung des DCGK
nur als Ganzes ablehnen, einzelne Ergänzungen/ Aktualisierungen können
nicht abgelehnt oder modifiziert werden)

Kritik: Konsultationsverfahren ist ebenfalls intransparent, schriftliche
Stellungsnahmen können von Stakeholdern eingereicht werden, jedoch ist
nicht klar ob und nach welchen Maßstäben diese berücksichtigt werden

\hypertarget{corporate-governance-berichterstattung}{%
\section{Corporate Governance
Berichterstattung}\label{corporate-governance-berichterstattung}}

\hypertarget{entsprechenserkluxe4rung}{%
\subsection{Entsprechenserklärung}\label{entsprechenserkluxe4rung}}

Pflicht zur Abgabe einer Entsprechenserklärung gilt für den Vorstand und
den Aufsichtsrat jeder börsennotierten Gesellschaft in Deutschland\\
geregelt in §161 AktG

\hypertarget{erkluxe4rung-zum-corporate-governance-kodex}{%
\section{§ 161~Erklärung zum Corporate Governance
Kodex}\label{erkluxe4rung-zum-corporate-governance-kodex}}

(1\textbf{) Vorstand und Aufsichtsrat} der börsennotierten Gesellschaft
\textbf{erklären jährlich}, dass den vom Bundesministerium der Justiz
und für Verbraucherschutz im amtlichen Teil des Bundesanzeigers bekannt
gemachten \textbf{Empfehlungen der „Regierungskommission Deutscher
Corporate Governance Kodex'' entsprochen wurde und wird oder welche
Empfehlungen nicht angewendet wurden oder werden und warum nicht}.
Gleiches gilt für Vorstand und Aufsichtsrat einer Gesellschaft, die
ausschließlich andere Wertpapiere als Aktien zum Handel an einem
organisierten Markt im Sinn des § 2 Absatz 11 des
Wertpapierhandelsgesetzes ausgegeben hat und deren ausgegebene Aktien
auf eigene Veranlassung über ein multilaterales Handelssystem im Sinn
des § 2 Absatz 8 Satz 1 Nummer 8 des Wertpapierhandelsgesetzes gehandelt
werden.

\(2\) Die \textbf{Erklärung} ist auf der \textbf{Internetseite} der
Gesellschaft dauerhaft \textbf{öffentlich zugänglich} zu machen.

Bestandteile:

\begin{itemize}
\tightlist
\item
  DCGK Empfehlungen wurde und wird entsprochen (in Vergangenheit und
  Zukunft)
\item
  Welchen Empfehlungen wurde nicht entsprochen?
\item
  Warum wurde diesen Empfehlungen nicht entsprochen?
\end{itemize}

Die Entsprechenserklärung richtet sich primär an \textbf{Anleger} und
andere \textbf{Stakeholder des Unternehmens}

** Informationsfunktion des Entsprechenserklärung **(Vergangenheit und
Zukünft)

Die Erklärungspflicht soll disziplinierend als Anreiz dazu dienen, dass
Unternehmen den Empfehlungen des DCGK entsprechen. Abweichungen sollen
dadurch grundsätzlich vermieden werden und nur selten in gut
begründbaren Fällen in Betracht gezogen werden

\textbf{Anreizfunktion} \textbf{für Befolgung} durch Angst vor
Sanktionierungen durch Öffentlichkeitsdruck

\textbf{Wann muss Entsprechenserklärung abgegeben werden}?

Bei \textbf{\emph{Börsennotierung}}! (Gesellschaft ist an einem
\textbf{organisierten Markt notiert}, \textbf{unabhängig ob im In- oder
Ausland)}

Im Freiverkehr (=nicht organisierter Markt) gehandelte Unternehmen
müssen nur dann eine Entsprechenserklärung abgeben, wenn es andere
Wertpapiere an einem organisierten Markt ausgegeben hat (z.B. Anleihen)

Betrifft nur Gesellschaften, die nach deutschem Recht gegründet sind
(AG, SE, KGaA), nach ausländischem Recht gegründete Gesellschaften sind
nicht verpflichtet (auch wenn sie an einer deutschen Börse notiert sind)

\textbf{Welches Organ muss die Entsprechenserklärung abgeben? }

Explizite Pflicht von** Vorstand und Aufsichtsrat!**

Die zwei Organe tragen die Pflicht nicht gemeinsam in der Rolle des
gesetzlichen Vertreters, wie z.B. die Pflicht zur Aufstellung des
Jahresabschluss nach HGB

Durch die unmittelbare Ansprache soll die \textbf{Eigenverantwortung}
\textbf{beider Organe} für die Befolgung der DCGK Empfehlungen
hervorgehoben werden

Eine abweichende Entsprechenserklärung beider Organe ist möglich, kommt
in der Praxis aber sehr selten vor

\textbf{Wie wird mit organspezifischen DCGK Vorschriften verfahren?}

Jedes Organ muss sich mit nicht organspezifischen Empfehlungen und den
eigenen Zuständigkeitsbereichen beschäftigen (Vergangenheitserklärung
und Absichtserklärungen)

Empfehlungen für das jeweils andere Organ müssen zumindest rückwirkend
bewertet werden

Verifizierung von Vergangenheitserklärung des anderen Organs, nicht aber
der Absichtserklärungen

Beispiel: Vorstand muss sich mit allen vorstandspezifischen Empfehlungen
(Vergangenheit und Zukunft), allgemeinen Empfehlungen sowie der
Befolgung von Aufsichtsrat spezifischen Reglungen in den vergangenen
Perioden beschäftigen

\hypertarget{erkluxe4rungsinhalt-der-entsprechenserkluxe4rung}{%
\subsection{Erklärungsinhalt der
Entsprechenserklärung}\label{erkluxe4rungsinhalt-der-entsprechenserkluxe4rung}}

Zu erklären ist nach §161 AktG:

„\ldots, dass den vom Bundesministerium der Justiz und für
Verbraucherschutz im amtlichen Teil des Bundesanzeigers bekannt
gemachten Empfehlungen der „Regierungskommission Deutscher Corporate
Governance Kodex'' entsprochen wurde und wird oder welche Empfehlungen
nicht angewendet wurden oder werden und warum nicht.''

\begin{itemize}
\item
  \begin{itemize}
  \tightlist
  \item
    Entsprechenserklärung beinhaltet \textbf{Rechenschafts-,
    Versprechens- und Erläuterungspflichten}
  \end{itemize}
\end{itemize}

\textbf{Rechenschaftspflicht}: Rechenschaft wird über die Befolgung der
DCGK Empfehlungen in den letzten 12 Monaten seit der vorherigen
Erklärung abgegeben (=\textbf{Vergangenheitserklärung})

\textbf{Versprechenspflicht}: Die Absicht darüber DCGK Empfehlungen in
den folgenden 12 Monaten zu befolgen wird abgegeben
(=\textbf{Absichtserklärung})

\textbf{Erläuterungspflicht}: Die Erläuterungspflicht besteht
ausschließlich dann, wenn von einzelnen DCGK Empfehlungen abgewichen
wird. Die entsprechende Empfehlung soll dabei unter Nennung der
jeweiligen Textziffer genannt werden und zusätzlich eine Erklärung
darüber abgegeben werden, welche wesentlichen Gründe dazu geführt haben,
dass die jeweilige Empfehlung nicht beachtet wurde (\textbf{knappe Form,
konkrete Informationen, wesentliche Abwägungen})

\begin{itemize}
\tightlist
\item
  Erklärungspflicht gilt nur für DCGK Empfehlungen (=soll), nicht für
  Anregungen (=sollte)!*
\end{itemize}

\hypertarget{offenlegungspflichten}{%
\subsection{Offenlegungspflichten}\label{offenlegungspflichten}}

\begin{itemize}
\item
  Entsprechenserklärung \textbf{muss jährlich abgegeben werden} (nicht
  streng tagesgenau, eine Überschreitung um wenige Tage ist
  unbedenklich)
\item
  Entsprechenserklärung muss auf der \textbf{Internetseite der
  Gesellschaft dauerhaft öffentlich zugänglich} sein (erlischt formal
  nach 12 Monaten bei Abgabe einer neuen Entsprechenserklärung, nicht
  aktuelle Fassungen sollen jedoch nach DCGK Empfehlungen weitere 5
  Jahre öffentlich zur Verfügung gestellt werden
\item
  Entsprechenserklärung ist außerdem \textbf{Bestandteil der Erklärung
  zur Unternehmensführung im publikationspflichtigen Lagebericht} (nach
  §289 HGB)
\item
  Entsprechenserklärung muss gemeinsam mit dem Jahresabschluss dem
  Bundesanzeiger zur Veröffentlichung bereitgestellt werden
\item
  Im Anhang des Jahresabschlusses muss ein Vermerk erfolgen, dass die
  Entsprechenserklärung online zur Verfügung gestellt wurde (inklusive
  Adresse)
\end{itemize}

\hypertarget{aktualisierungspflichten}{%
\subsection{Aktualisierungspflichten}\label{aktualisierungspflichten}}

Aktualisierungspflichten unterscheiden sich \textbf{nach Kodex
Änderungen} (DCGK ändert Empfehlungen) und \textbf{Absichtsänderungen}
(Unternehmen ändert seine Absichten zur Befolgung der DCGK Empfehlungen)

\textbf{Absichtserklärungen}

Die Absichtserklärung beschreibt die Absicht zur zukünftigen Befolgung
von DCGK Empfehlungen

Eine \textbf{Absichtserklärung führt nicht zur Selbstbindung} an die
erklärten Absichten, Unternehmen kann auch unterjährig von einer
erklärten Absicht abweichen!

Falls ein Unternehmen unterjährig seine Absichten zur Befolgung von DCGK
Empfehlungen ändert, ergibt sich für das Unternehmen die
\textbf{Pflicht} ihre Entsprechenserklärung zeitnah zu aktualisieren

Aktualisierungspflicht bei unterjähriger Änderung von Absichten soll die
\textbf{Informationsfunktion der Entsprechenserklärung sicherstellen}

\textbf{Ergänzung der Entsprechenserklärung, keine neue
Entsprechenserklärung!}

** **Absicht und Empfehlung muss angegeben und begründet werden wie in
der regulären Erklärung, die Termine zur Abgabe der nächsten
Entsprechenserklärung bleiben von Ergänzungen unberührt

\textbf{Aktualisierungspflichten bei Kodex Änderungen?}

Eine unterjährige Änderung der DCGK Empfehlungen führt \textbf{nicht} zu
einer Aktualisierungspflicht der Unternehmen

Entsprechenserklärung verweist statisch auf die DCGK Empfehlungen, die
zum Verfassungszeitpunkt gültig waren, eine unterjährige Kodexänderung
beeinflusst somit nicht die Gültigkeit der Entsprechenserklärung (diese
verweist auf eine vorherige Version des DCGK)

Bei der Verfassung der nächsten Entsprechenserklärung wird anschließend
auf die aktualisierte DCGK Fassung eingegangen

\hypertarget{rechtsfolgen-von-fehlerhaften-entsprechenserkluxe4rungen}{%
\subsection{Rechtsfolgen von fehlerhaften
Entsprechenserklärungen}\label{rechtsfolgen-von-fehlerhaften-entsprechenserkluxe4rungen}}

Abgabe einer fehlerhaften Entsprechenserklärung stellt einen Verstoß
gegen §161 AktG dar

** Pflichtverletzung von Vorstand und Aufsichtsrat**

Es entsteht i.d.R. kein kausaler Schaden durch eine fehlerhafte
Entsprechenserklärung, das Entstehen von \textbf{Haftungsansprüchen im
Innenverhältnis ist daher sehr unwahrscheinlich}

Jedoch ist die Entsprechenserklärung eine \textbf{wichtige
Informationsgrundlage} (=Informationsfunktion der Entsprechenserklärung)
für \textbf{Abstimmungen auf der Hauptversammlung}, falls diese
\textbf{wesentliche} \textbf{Fehler} enthält basieren
Abstimmungsergebnisse somit auf der Grundlage von falschen Informationen

Die Anfechtbarkeit von Hauptversammlung-Beschlüssen kann durch
fehlerhafte Entsprechenserklärungen entstehen, Voraussetzung dafür ist
aber, dass die fehlerhaften Informationen wesentliche und relevante
Informationen für Abstimmungen betreffen

Beispiel: Entsprechenserklärung enthält Fehler über die Einhaltung von
DCGK Empfehlungen zur Wahl des Aufsichtsrats. Falsche Informationen
könnten hier wesentlich die Stimmabgabe beeinflussen und ein Beschluss
könnte angefochten werden

\hypertarget{erkluxe4rung-zur-unternehmensfuxfchrung}{%
\section{Erklärung zur
Unternehmensführung}\label{erkluxe4rung-zur-unternehmensfuxfchrung}}

Durch \textbf{HGB nach §289f }geregelte \textbf{Berichterstattung }die
im \textbf{Lagebericht} (Einzel- oder Konzernabschluss) enthalten sein
muss (alternativ auch online möglich mit entsprechendem Vermerk im
Lagebericht)

Erklärung ist für alle \textbf{börsennotierten Gesellschaften} (AG, SE
und KGaA) vorgeschrieben, \textbf{die Aktien oder andere Wertpapiere
(z.B. Anleihen) an einem organisierten Markt ausgegeben haben} (gleicher
Kreis der verpflichteten Gesellschaften wie auch bei Abgabe der
Entsprechenserklärung)

** Unterschiede zur Entsprechenserklärung**

\begin{itemize}
\tightlist
\item
  \textbf{Gesetzlicher Vertreter} der Gesellschaft ist zur Abgabe
  verpflichtet
\end{itemize}

\textbf{Vorstand} muss als gesetzlicher Vertreter die Erklärung zur
Unternehmensführung abgeben

\begin{itemize}
\tightlist
\item
  \textbf{Tatsachen zur Unternehmensführung (=Ist Zustand bei
  Erklärung)} müssen dargestellt werden, es werden \textbf{keine
  Absichten für zukünftiges Verhalten} aufgeführt
\end{itemize}

\textbf{Keine Aktualisierungspflicht} bei unterjähriger Veränderung des
IST Zustands

\hypertarget{inhalte-der-erkluxe4rung-zur-unternehmensfuxfchrung}{%
\subsection{Inhalte der Erklärung zur
Unternehmensführung}\label{inhalte-der-erkluxe4rung-zur-unternehmensfuxfchrung}}

\begin{enumerate}
\def\labelenumi{\arabic{enumi}.}
\item
  \textbf{Entsprechenserklärung (zur DCGK)}
\item
  \textbf{Angaben zu Unternehmensführungspraktiken}
\item
  Ausführungen zur Arbeitsweise und Vorstand und AR (insbesondere der
  Geschäftsordnung); Zusammensetzung und Arbeitsweise von Ausschüssen
  von Vorstand und AR (Verweis auf deren Internetseiten soll erfolgen)
\item
  Erfüllung von Soll Zielgrößen für Frauenanteil (AR, Vorstand und erste
  zwei Führungsebenen)
\item
  Falls Pflichtquoten für den Geschlechteranteil im AR vorgeschrieben
  sind, muss über deren Erfüllung ebenfalls berichtet werden
\item
  Große AGs müssen zusätzlich über ihr Diversitätskonzept für die
  Zusammensetzung von Vorstand und AR beschreiben
\end{enumerate}

Besondere Bedeutung haben die \textbf{Angaben zur
Unternehmensführungspraktiken}, hier sind Verweise auf die
\textbf{Unternehmenssatzung, interne Kodizes und Compliance Richtlinien
}anzuführen insofern solche im Unternehmen vorhanden sind (Beschränkung
auf die reine Nennung von relevanten Dokumenten und Verweis auf etwaige
Veröffentlichungen und Links)

Beispielhafte Vorgaben eines Diversity-Konzepts für Vorstand und AR
(Punkt 6 der Erklärung zur Unternehmensführung bei großen
Aktiengesellschaften)

\begin{itemize}
\tightlist
\item
  Berufshintergründe (welche Fachkenntnisse und Erfahrungen)
\item
  Geschlechteranteil (Soll Anteil für Frauen)
\item
  Alter (Generationenmix, Höchstalter)
\item
  Internationalität
\item
  Unabhängigkeit von AR Mitgliedern
\end{itemize}

\hypertarget{weitere-berichterstattungspflichten-mit-dcgk-bezug}{%
\section{Weitere Berichterstattungspflichten (mit DCGK
Bezug)}\label{weitere-berichterstattungspflichten-mit-dcgk-bezug}}

\hypertarget{bericht-des-aufsichtsrat-nach-171-ii-aktg}{%
\subsection{Bericht des Aufsichtsrat nach §171 II
AktG}\label{bericht-des-aufsichtsrat-nach-171-ii-aktg}}

Der Bericht des Aufsichtsrats hat eine \textbf{Rechenschafts- und
Informationsfunktion gegenüber der Hauptversammlung }(HV entscheidet
auch auf dessen Basis über Wiederbestellung und Entlastung des AR).

Bestandteil des Geschäftsberichts und Vorstellung auf Hauptversammlung

\textbf{(Pflicht-)Inhalte nach AktG}

\begin{itemize}
\tightlist
\item
  AR berichtet über das Ergebnis der \textbf{Prüfung von Rechnungslegung
  und Geschäftsführung}
\item
  \textbf{Arbeitsweise des AR} wird beschrieben
\item
  Bei Börsennotierten Gesellschaften müssen bestehende Ausschüsse und
  deren Mitgliederanzahlen beschrieben werden
\end{itemize}

\textbf{Ergänzende Bestandteile nach DCGK (=Empfehlungen)}

\begin{itemize}
\tightlist
\item
  An welchen Sitzungen des Aufsichtsrats und seiner Ausschüsse haben die
  jeweiligen AR Mitglieder teilgenommen?
\item
  Sind Interessenskonflikte aufgetreten, wenn ja wie wurde mit diesen
  umgegangen?
\end{itemize}

Es treten \textbf{Überschneidungen} zwischen \textbf{Erklärung zur
Unternehmensführung} sowie des Berichts des AR auf (insb.
\textbf{Ausführungen über die Arbeitsweise des AR})

Nicht vermeidbar, da Berichterstattungspflicht unterschiedliche Organe
betrifft

** Erklärung zur Unternehmensführung: Vorstand**

** Bericht des AR: AR**

\hypertarget{rechnungslegungsvorschriften-mit-dcgk-bezug}{%
\subsection{Rechnungslegungsvorschriften (mit DCGK
Bezug)}\label{rechnungslegungsvorschriften-mit-dcgk-bezug}}

Weitere Berichterstattungspflichten finden sich in den folgenden
Publikationen:

\begin{itemize}
\tightlist
\item
  \textbf{Anhang nach HGB} (Name und Vergütung der Organmitglieder;
  Honorar des Abschlussprüfers)
\item
  \textbf{Lagebericht nach HGB} (interne Kontrollsysteme,
  Risikomanagement, Vergütung des Vorstands,
  nichtfinanzielle-Leistungsindikatoren)
\item
  \textbf{Nicht-finanzielle Erklärung nach HGB} (Umwelt-, AN- und
  Sozialbelange, Achtung der Menschenrechte, Korruptionsbekämpfung)
\item
  \textbf{Pflichtangaben nach IFRS }(Vergütung, Risikomanagement,
  Beziehung zu Parteien)
\end{itemize}

\hypertarget{anforderungen-an-anreize-des-vorstands}{%
\section{Anforderungen an Anreize des
Vorstands}\label{anforderungen-an-anreize-des-vorstands}}

Generalnorm: \textbf{Der Vorstand hat unter eigener Verantwortung die
Gesellschaft zu leiten} (§76 AktG)

\begin{itemize}
\tightlist
\item
  Der Vorstand besitzt \textbf{weitreichende Handlungsfreiheit} bei der
  Leitung des Unternehmens
\item
  Der Vorstand \textbf{unterliegt nicht den Weisungen von Dritten} (auch
  AR kann dem Vorstand beispielsweise keine verbindlichen
  Handlungsanweisungen erteilen)
\item
  Der Vorstand leitet das Unternehmen nach eigener Verantwortung, die
  \textbf{Leitungsaufgabe} darf dementsprechend nicht an andere Organe
  delegiert werden \textbf{Vorstand trägt die Letztverantwortung}
\end{itemize}

\textbf{Leitungsaufgabe}

Keine detaillierte Beschreibung im Gesetzt, die Leitungsaufgabe kann
aber anhand der betriebswirtschaftlich anerkannten Grundsätze
ordnungsgemäßer Unternehmensführung konkretisiert werden

\begin{itemize}
\tightlist
\item
  Verantwortung über strategische und operative Unternehmensplanung
\item
  Verantwortung über die Organisation von Arbeitsabläufen und die
  Vorgabe von Richtlinien der Personalpolitik
\item
  Kontrolle und Umsetzung der Unternehmensplanung und
  Unternehmensorganisation
\end{itemize}

\textbf{Aufbau des Vorstands}

Mitglieder: Grundsätzlich setzt sich der Vorstand aus mehreren
Mitgliedern zusammen, ein Alleinvorstand ist möglich, aber kommt selten
vor

Häufig wird ein Vorstandsvorsitzender oder Vorstandssprecher ernannt

Der Vorstand ist grundsätzlich gemeinschaftlich mit der Geschäftsführung
beauftragt, Entscheidungen sind daher i.d.R. einstimmig zu fällen
(=Kollegialprinzip) (nach §77 AktG)

Die Satzung oder Geschäftsordnung darf abweichende Reglungen zur
Arbeitsweise des Vorstands enthalten! Die Befugnis zum Erlass einer
Geschäftsordnung liegt vorrangig beim Aufsichtsrat, falls dieser von
dieser Befugnis keinen Gebrauch macht, darf der Vorstand sich mit
einstimmigem Beschluss selbst eine Geschäftsordnung erteilen

\textbf{Mögliche Abweichungen von der gemeinschaftlichen, einstimmigen
Geschäftsführung nach AktG}

\begin{itemize}
\tightlist
\item
  Mehrheitsprinzip anstelle des Einstimmigkeitsprinzips
\item
  Ressortzuständigkeiten der Vorstandsmitglieder anstelle der
  gemeinschaftlichen Führung in allen Bereichen
\end{itemize}

Nicht zulässig sind in jedem Fall Reglungen, bei denen ein
Vorstandsmitglied gegen den Willen der Mehrheit handeln kann/darf!

\textbf{Ressortzuständigkeiten? (=Bereichsvorstand)}

Innerhalb seiner Zuständigkeit wird ein Vorstandsmitglied zur
eigenständigen Leitungsentscheidungen befugt. Das Vorstandsmitglied hat
jedoch Pflicht andere Vorstandsmitglieder über seine Entscheidungen/
Handlungen zu informieren

Andere Vorstandsmitglieder haben die Pflicht die Handlungen des
ressortzuständigen Vorstandsmitglieds zu überwachen

\textbf{Fallanwendung?}

\begin{itemize}
\tightlist
\item
  Prüfe ob Entscheidung grundsätzlich Teil der Leitungsaufgabe des
  Vorstands ist (z.B. Strategieentwicklung), falls ja trägt der Vorstand
  die Eigenverantwortung und darf die Planung nicht an Dritte delegieren
\item
  Prüfe ob Ressortzuständigkeiten und Abweichungen vom
  Einstimmigkeitsprinzip festgelegt sind, falls nicht muss Vorstand
  einstimmig über Entscheidungen abstimmen
\item
  Falls Ressortzuständigkeiten verteilt sind, kann eigenverantwortlich
  entschieden werden solange der restliche Vorstand informiert wird und
  das Handeln nicht gegen die Mehrheit des Vorstands verstößt
\end{itemize}

\hypertarget{grenzen-des-eigenverantwortlichen-handelns-des-vorstands}{%
\subsection{Grenzen des eigenverantwortlichen Handelns des
Vorstands}\label{grenzen-des-eigenverantwortlichen-handelns-des-vorstands}}

\textbf{Zustimmungsvorbehalte des Aufsichtsrats}

Nach §111 IV AktG müssen für den AR Zustimmungsvorbehalte festgelegt
werden die das eigenverantwortliche Handeln des Vorstands bei bestimmten
Entscheidungen begrenzen

Keine detaillierten Vorgaben aus AktG, der AR muss selbst einen Katalog
an zustimmungsbedürftigen Vorstandsentscheidungen erstellen (unter
Berücksichtigung von unternehmensspezifischen Besonderheiten)

Katalog der zustimmungsbedürftigen Entscheidungen muss in die
Geschäftsordnung oder Satzung des Unternehmens aufgenommen werden

\textbf{Verpflichtender Bestandteil} der zustimmungsbedürftigen
Entscheidungen: \textbf{Geschäfte von grundlegender Bedeutung, }dazu
zählen alle** **wesentlichen Investitions,- Finanz- und
Strategieentscheidungen

\textbf{Optionale Bestandteile}: Außergewöhnliche oder bedeutende
Geschäfte

\textbf{Nicht Bestandteil} des Katalogs dürfen Zustimmungsvorbehalte für
reine Geschäftsführungsaufgaben sein, darunter fallen u.a. operative
Entscheidungen

Beachte: Der AR kann anhand von Zustimmungsvorbehalten Entscheidungen
des Vorstands ablehnen, er hat aber nicht die Möglichkeit dem Vorstand
bestimmte Handlungen vorzuschreiben

Keine Handlungspflicht für Vorstand durch Verweigerung von AR

Der Vorstand kann alternativen Vorschlag entwickeln, den gleichen
Vorschlag erneut zur Abstimmung bringen und/oder bei anhaltender
Verweigerung durch den AR die Hauptversammlung zur Abstimmung bringen

\textbf{Beispiel für Zustimmungsvorbehalte des AR}

Der Vorstand bedarf der Zustimmung des AR bei Vornahme der folgenden
Geschäfte:

\begin{itemize}
\tightlist
\item
  Erwerb von Unternehmen deren Erwerbspreis 10 Mio. € überschreitet
\item
  Aufnahme neuer wesentlicher Geschäftsfelder
\item
  Aufnahme von Fremdkapital i.H.v. mehr als 10 Mio. €
\end{itemize}

\textbf{Fallanwendung?}

\begin{itemize}
\tightlist
\item
  Prüfe grundlegend ob Entscheidung unter die Leitungsaufgabe des
  Vorstands fällt, ob abweichende Reglungen über die Arbeit des
  Vorstands (Mehrheitsprinzip, Ressortzuständigkeiten) bestehen
\item
  Prüfe ob die Entscheidungen Geschäfte von grundlegender Bedeutung sind
  (wesentliche Investitions,- Finanz- oder Strategieentscheidungen),
  falls ja sind diese in jedem Fall zustimmungsbedürftig
\item
  Prüfe ob die unternehmensspezifischen Zustimmungsvorbehalte die
  anstehenden Entscheidungen des Vorstands betreffen, falls ja wird das
  eigenverantwortliche Handeln des Vorstands durch die Zustimmung des AR
  begrenzt
\end{itemize}

\textbf{Grenzen des eigenverantwortlichen Handelns des Vorstands -
Hauptversammlung}

Die HV ermöglicht es den Aktionären ihre Stimmrechte in Bezug auf
grundlegende Entscheidungen auszuüben

Die HV entscheidet über die folgenden Abstimmungspunkte

\begin{itemize}
\tightlist
\item
  Bestellung von AR Mitgliedern und Abschlussprüfer
\item
  Abbestellung von AR- und Vorstandsmitgliedern
\item
  Verwendung des Bilanzgewinns
\item
  Maßnahmen zur Kapitalbeschaffung
\end{itemize}

Entscheidungen der HV sind vom Vorstand \textbf{zwingend} umzusetzen

Die HV hat grundsätzlich keine Befugnisse (direkt) über Fragen der
Geschäftsführung (=Leitungsaufgabe) abzustimmen

Die Hauptversammlung kann jedoch indirekt, besonders in
Finanzierungsfragen, Einfluss auf Entscheidungen zur Geschäftsführung
nehmen (Verwendung des Jahresüberschuss, Dividendenzahlungen)

\textbf{Grenzen eigenverantwortlichen Handelns des Vorstands -- Satzung}

Die Satzung einer Aktiengesellschaft legt grundsätzliche
Unternehmensstrukturen fest

Mindestinhalte:

\begin{itemize}
\tightlist
\item
  Name und Sitz der Gesellschaft
\item
  Gegenstand des Unternehmens
\item
  Höhe des Grundkapitals
\end{itemize}

Weitere Inhalte können in die Satzung aufgenommen werden solange diese
im Einklang mit den Reglungen des AktG stehen

Der Vorstand muss die Unternehmenssatzung beim Wahrnehmen seiner
Leitungsaufgabe respektieren, besonders der \textbf{Gegenstand des
Unternehmens} kann dabei die Expansion in neue wesentliche
Geschäftsfelder, die vom Gegenstand des Unternehmens abweichen,
beschränken

Eine Satzungsänderung setzt zwingend einen \textbf{Beschluss der
Hauptversammlung} voraus

Um die Einschränkungen des eigenverantwortlichen Vorstandshandeln
möglichst wenig zu beschränken, wird der Unternehmensgegenstand daher
häufig sehr allgemein und weit formuliert

\textbf{Grenzen eigenverantwortlichen Handelns des Vorstands -- Sonstige
gesetzliche Vorschriften}

Der Vorstand hat darauf zu achten, dass seine Handlungen im Einklang mit
geltendem Recht stehen, seine Entscheidungsfreiheit wird durch die
folgenden Gesetzte zusätzlich eingeschränkt:

\begin{itemize}
\tightlist
\item
  Loyalitäts- und Treuepflichten gegenüber dem Unternehmen
  (Wettbewerbsverbot und Verschwiegenheitspflicht nach AktG)
\item
  Konkrete Vorstandspflichten (Führen von Büchern, Einrichtung eines
  Risikofrüherkennungssystems, Verhalten im Fall Überschuldung und
  Insolvenz nach AktG)
\item
  Pflichten als gesetzlicher Vertreter der Gesellschaft (Offenlegung von
  Jahresabschluss und Lagebericht nach HGB, branchenspezifische
  Reglungen, etc.)
\end{itemize}

\textbf{Fallanwendung?}

\begin{itemize}
\tightlist
\item
  Grundlegende Feststellung, dass Vorstand auf Rechtskonformität seiner
  Entscheidung zu achten hat
\item
  Prüfung auf Verstöße gegen geltendes Recht
\end{itemize}

\hypertarget{business-judgment-rule}{%
\section{Business Judgment Rule}\label{business-judgment-rule}}

Die eigenverantwortliche Unternehmensleitung erfordert vom Vorstand das
regelmäßige Treffen von weitreichenden Entscheidungen, ohne eine
perfekte Informationsgrundlage zu besitzen

Auch bei sorgfältiger Analyse und Abwägung durch den Vorstand können für
die Gesellschaft nachteilige Entscheidungen getroffen werden

Problem: Wann liegt eine Pflichtverletzung des Vorstands vor?

\begin{itemize}
\tightlist
\item
  Zu strenge Regulierungen würden durch Vorstand antizipiert werden und
  zu übertriebener Vorsicht zum Nachteil der Gesellschaft führen (z.B.
  Angst vor Haftungsrisiken führt dazu, dass notwendige
  Expansionsmaßnahmen nicht durchgeführt werden)
\item
  Aus juristischer Sicht ist es schwer zu beurteilen, ob Vorstand bei
  Entscheidungsfindung alle relevanten Informationen und Szenarien
  \textbf{sorgfältig und unvoreingenommen} überprüft hat
\end{itemize}

Lösung durch einen ex-ante Vertragsmaßstab bei dessen Beachtung
Haftungsrisiken der Unternehmensleitung unabhängig vom Ergebnis
ausgeschlossen werden

\textbf{Safe Harbor}

\textbf{Business Judgment Rule}: Spezifikation des Gesetzgebers unter
welchen Voraussetzungen eine Pflichtverletzung durch den Vorstand (und
somit Haftungsansprüche) nicht vorliegt

Wortlaut: „\textbf{Eine Pflichtverletzung liegt nicht vor}, wenn das
Vorstandsmitglied bei einer \textbf{unternehmerischen Entscheidung}
vernünftigerweise annehmen durfte, auf der Grundlage angemessener
Information** zum Wohle der Gesellschaft zu handeln**'' (§ 93 I 2 AktG)

\textbf{Voraussetzungen zur Anwendung des Business Judgment Rule}

\begin{longtable}[]{@{}ll@{}}
\toprule
\endhead
Voraussetzung/ Anforderung & Konkretisierung \\
Es handelt sich um \textbf{eine unternehmerische Entscheidung} des
Vorstands & \\
Handeln zum Wohle der Gesellschaft & ** \textbf{Schwierigkeit der
Gewichtung, möglicher Ansatz ist das Handeln im Aktionärsinteresse unter
Wahrung von sozialen Normen und rechtlichen Vorgaben (=}enlightened
shareholder value**) \\
Entscheidung auf angemessener Informationsgrundlage & Allerdings darf
Vorstand den Zeit- und Kostenaufwand des Aufwands „angemessen''
berücksichtigen \\
\bottomrule
\end{longtable}

Die Nichterfüllung der Business Judgment Rules führt dabei nicht
zwingend zu einer Pflichtverletzung des Vorstands, die Prüfung der
allgemeinen Sorgfaltspflicht des Vorstands nach AktG muss im Anschluss
ebenfalls erfolgen („wie hätte sich ein ordentlicher/ objektiver
Unternehmensleiter in dieser Situation entschieden?{}``)

\textbf{Fallanwendung}?

\begin{itemize}
\tightlist
\item
  Prüfe ob geplante Entscheidung das Risiko des Scheiterns in sich
  birgt, falls ja könnten sich daraus Haftungsrisiken für den Vorstand
  ergeben die einen Anreiz zur Übervorsicht mit sich bringen. Ein Ex-
  ante erfolgsversprechendes Projekt könnte aus diesem Grund nicht
  durchgeführt werden
\item
  Prüfe ob die Voraussetzungen der Business Judgment Rule erfüllt sind
\item
  Liegt eine unternehmerische Entscheidung vor? Fällt die Entscheidung
  unter die Leitungsaufgabe des Vorstands/ der eigenverantwortlichen
  Unternehmensleitung?
\item
  Erfolgt die Entscheidung im Rahmen des Unternehmensinteresses
  (Shareholder/ Stakeholder Interessen, keine Hinweise auf Handeln im
  Sonder- oder Eigeninteresse, Wahrung des Unternehmensfortbestands und
  der Rentabilität)
\item
  Erfolgt das Handeln des Vorstands auf einer angemessenen
  Informationsgrundlage? Gibt es eine ausreichende Dokumentation der
  Informationsbeschaffung und Analyse?
\end{itemize}

\hypertarget{anreize-der-unternehmensleitung}{%
\section{Anreize der
Unternehmensleitung}\label{anreize-der-unternehmensleitung}}

\hypertarget{haftungsrisiken}{%
\subsection{Haftungsrisiken}\label{haftungsrisiken}}

Fahrlässige oder schuldhafte Pflichtverletzungen des Vorstands können zu
Haftungsrisiken für entstandene Schäden führen

Die Beachtung der Business Judgment Rule sowie des Sorgfaltsmaßstab
schützt den Vorstand vor Haftungsrisiken und wirkt so als Anreiz auf die
Befolgung dieser Verhaltensmaßstäbe

Grundsätzlich sind jedoch sowohl BJR als auch die Sorgfaltspflicht
allgemein gehaltene Grundsätze und liefern keine spezifischen
Handlungsanforderungen

\hypertarget{reputationsrisiken}{%
\subsection{Reputationsrisiken}\label{reputationsrisiken}}

Die Reputation beschreibt das Prestige, das einer Person oder
Organisation aufgrund des vergangenen Verhaltens zugeschrieben wird

Erwartungen über künftiges Verhalten auf Grundlage des bisherigen
Verhaltens

Einwandfreies Verhalten in der Vergangenheit verbessert somit die
Reputation, Fehlverhalten in der Vergangenheit verschlechtert die
Reputation

Im Rahmen der Prinzipal Agenten Theorie spielt die Reputation eine
wichtige Rolle bei der Auswahl von Agenten (Reputation als Signal kann
dem Prinzipal die wahren Eigenschaften des Agenten offenbaren und somit
bestehende Informationsasymmetrien abbauen sowie adverse Selektion
verhindern)

Reputationsverlust führt dazu, dass Agenten einerseits bei zukünftigen
Entscheidungen von Prinzipalen mit geringerer Wahrscheinlichkeit
beauftragt werden und außerdem kann es dazu führen, dass die aktuell
gehaltene Position des Agenten verloren wird (betrifft alle ausgeführten
Ämter, die eine hohe Reputation verlangen wie z.B. die Aufgabe der
Unternehmensleitung und Sitz im Aufsichtsrat einer weiteren
Gesellschaft)

\textbf{Vorteile der Reputation als Anreizmechanismus }

Unmittelbare und umfassende Wirkungsweise führt zu hoher Anreizwirkung
(jegliches Fehlverhalten kann Reputation sofort negativ beeinflussen,)

\textbf{Nachteil der Reputation als Anreizmechanismus}

Reputationsverlust kann bereits bei mutmaßlichem Fehlverhalten
unverhältnismäßig stark ausfallen, schwer steuerbarer Mechanismus (vgl.
Shitstorm)

Ex-Ante (=aus vorheriger Sicht) Definition von Fehlverhalten oft unklar,
ex-post (=im Nachhinein) Beurteilung von tatsächlichem Fehlverhalten oft
unsorgfältig/ unkontrolliert

Praxisbeispiel: Schlechte Performance eines Unternehmens die auf externe
Faktoren zurückzuführen ist, kann schnell dazu führen, dass
Vorstandsmitglieder ihr Amt verlieren und ohne vorliegendes
Fehlverhalten einen Reputationsschaden erleiden

** Fallanwendung?**

\begin{itemize}
\tightlist
\item
  Was plant der Vorstand/Unternehmensleitung und könnte der
  Umsetzungserfolg dieser Maßnahme die Reputation des Vorstands
  beeinflussen?
\item
  Falls ja, erfolgreiche Umsetzung würde Reputation der beteiligten
  Agenten erhöhen (in allen Ämtern). Ein Scheitern der Maßnahme könnte
  andererseits zu einem Reputationsverlust führen, die Agenten könnten
  dadurch alle ihre ausgeübten Ämter verlieren
\item
  Reputation hat eine starke Anreizwirkung auf Agenten falls diese in
  reputationsabhängigen Ämtern aktiv sind. Eine Einschätzung der
  Auswirkungen einer Entscheidung auf die Reputation ist dabei jedoch ex
  ante oft nur schwer möglich, der Reputationsverlust kann Agenten auch
  bei vermeintlich guten Entscheidungen im Nachhinein unkontrolliert
  treffen (z.B. in Form von übermäßigen Shitstorms etc.)
\end{itemize}

\hypertarget{verguxfctung-des-vorstands}{%
\subsection{Vergütung des Vorstands}\label{verguxfctung-des-vorstands}}

Die Erbringung eines optimalen Arbeitseinsatzes durch den Agenten aus
Sicht des Prinzipals kann auch durch den Einsatz von
Vergütungsmechanismen erreicht werden

\textbf{Ziel}: Interessenskonflikte reduzieren durch Vergütungssystem
des Agenten (=Vorstand)

\textbf{Grundproblem}: Arbeitseinsatz des Agenten ist i.d.R. nicht
perfekt beobachtbar aufgrund von Informationsasymmetrien

Für den erfolgreichen Abbauen von Interessenskonflikten und eine
effiziente Anreizwirkung auf die Vorstandsmitglieder ist die Wahl der
zur Vergütung genutzten Bemessungsgrundlage von hoher Bedeutung

\textbf{Anforderungen an Bemessungsgrundlage}

\begin{itemize}
\tightlist
\item
  \textbf{Anreizkompatibilität}: Nutzenfunktion des Prinzipals und
  Nutzenfunktion des Agenten verlaufen bei Entlohnung auf deren Basis
  gleich (höhere Entlohnung des Vorstands nur wenn auch höherer ROI von
  Shareholdern, z.B. bei aktienkursbasierter Bemessungsgrundlage)
\item
  \textbf{Beeinflussbarkeit}: Der Agent kann die Bemessungsgrundlage
  durch die tatsächliche Erbringung von guter Arbeitsleistung
  beeinflussen (z.B. durch Berücksichtigung einer Benchmark bei der
  Vergütung auf Basis des Aktienkurses, Bemessungsgrundlage ist nicht
  von gesamtwirtschaftlichen Konjunkturtrends abhängig)
\item
  \textbf{Manipulationsfreiheit}: Der Agent kann die Bemessungsgrundlage
  nicht durch verdeckte Aktionen verändern, um dadurch seine Vergütung
  zu erhöhen (keine signifikante Beeinflussung durch
  Abschreibungspolitik etc. möglich)
\end{itemize}

\textbf{Fallanwendung: Vergleich von Bemessungsgrundlagen}

\begin{longtable}[]{@{}
  >{\raggedright\arraybackslash}p{(\columnwidth - 4\tabcolsep) * \real{0.32}}
  >{\raggedright\arraybackslash}p{(\columnwidth - 4\tabcolsep) * \real{0.32}}
  >{\raggedright\arraybackslash}p{(\columnwidth - 4\tabcolsep) * \real{0.32}}@{}}
\toprule
\endhead
Bemessungsgrundlage & Gesperrte Aktien\footnote{Wahl des Aufsichtsrats
  bei AG und KGaA auf Hauptversammlung, bei GmbH auf der
  Gesellschaftsversammlung} und Aktienoptionen & Bilanzieller
Jahresüberschuss \\
Anreizkompatibilität & Ist gegeben, sowohl Shareholder als auch Vorstand
profitiert von einer Erhöhung des Aktienkurses (sowohl kurz- als auch
langfristig) & Prinzipiell auch gegeben, da eine hoher Jahresgewinn sich
auch positiv auf den Unternehmenswert auswirken dürfte. Jedoch können
Investitionen den Jahresüberschuss kurzfristig senken, auch wenn diese
langfristig positiv auf den Unternehmenswert wirken können (Anreiz auf
kurzfristige Maximierung des Gewinns stärker als auf langfristige?) \\
Beeinflussbarkeit & Prinzipiell gegeben, jedoch wird der Aktienkurs auch
durch viele externe Faktoren beeinfluss auf deren Entwicklung der
Vorstand keinen direkten Einfluss nehmen kann (Konjunktur, Zinspolitik,
etc.) & Beeinflussbarkeit ist gegeben (stärker als bei Aktienkurs),
Vorstand hat direkten Einfluss auf Aufwendungen und Erträge des
Unternehmens durch seine strategischen und operativen Entscheidungen \\
M anipulationsfreiheit & Vorstand kann versuchen den Aktienkurs durch
das Verbreiten von positiven Meldungen zu beeinflussen (Erfolg dieser
Manipulation hängt jedoch von seiner Glaubwürdigkeit ab und würde bei
Missbrauch Schaden nehmen)

Begrenzte Möglichkeiten zur Manipulation & M anipulationsfreiheit ist
grundsätzlich durch die Bilan zierungsvorschriften und Prüfungspflichten
gesichert.

Der Vorstand kann trotzdem Ermessensspielräume und Wahlrechte ausnutzen,
um den Jahresüberschuss zu beeinflussen (Abschreibungen, Bilanzpolitik
etc.) \\
\bottomrule
\end{longtable}

\hypertarget{anreiz-auf-die-unternehmensleitung-wettbewerb}{%
\section{Anreiz auf die Unternehmensleitung --
Wettbewerb}\label{anreiz-auf-die-unternehmensleitung-wettbewerb}}

\hypertarget{insolvenzrisiko}{%
\subsection{Insolvenzrisiko}\label{insolvenzrisiko}}

\textbf{Anreiz des Vorstands}: Insolvenz des Unternehmens soll vermieden
werden da diese seine \textbf{Anstellung} und seine
\textbf{Vergütungsansprüche} gefährdet. Zusätzlich könnten sich
\textbf{Haftungsansprüche} gegen den Vorstand ergeben und auch die
Gefahr eines \textbf{Reputationsschaden} ist hoch

Ein \textbf{hoher Wettbewerbsdruck} auf den Absatzmärkten des
Unternehmens führt zu einem erhöhten Insolvenzrisikos

Im Vergleich zu (quasi-) Monopolisten, die ihre Preise bei drohender
Insolvenz erhöhen können, müssen Unternehmen die im (vollständigen)
Wettbewerb stehen die vom Markt vorgegeben Preise akzeptieren

Wettbewerbsdruck steigert das Insolvenzrisikos durch die stärkere
Sanktionierung von Fehlverhalten der Unternehmensleitung (z.B. quiet
Life, empire building etc.)

\hypertarget{benchmarking}{%
\subsection{Benchmarking}\label{benchmarking}}

Durch einen starken Wettbewerb zwischen Unternehmen einer Branche
verbessert sich auch die Möglichkeit der Erfolgsmessung der
Unternehmensleistung durch den Einsatz von relativen
Bemessungsgrundlagen

Aggregation der Ergebnisse von Unternehmen einer Branche ermöglicht die
präzisere Beurteilung des Arbeitseinsatzes der Unternehmensleitung eines
Unternehmens dieser Branche

Je mehr Mitbewerber in der Branche aktiv sind, desto aussagekräftiger
wird das Ergebnis. Anforderung an Manipulationsfreiheit einer
Bemessungsgrundlage wird durch Benchmarking besser erfüllt als beim
Einsatz von absoluter BMG

Benchmarking erhöht damit die Anreizwirkung auf den Vorstand sich
anzustrengen (seine tatsächliche Leistung wird für die Prinzipale besser
beobachtbar durch den Abbau von Informationsasymmetrien)

\textbf{Formal:}

Y = x + e1 + e2

mit y=Projektergebnis; x=Arbeitseinsatz Vorstand; e1=branchenspezifische
Einflüsse; e2=sonstige Einflüsse

y -- e1 = x + e2

y -- e1 kann als relatives Erfolgsmaß verwendet werden, e1 kann anhand
der Profitabilität der Mitbewerber ermittelt werden

\textbf{Fallanwendung}?

\begin{itemize}
\tightlist
\item
  Prüfe ob das Unternehmen in einer wettbewerbsintensiven Branche
  befindet? Falls ja besteht bei Misserfolg der geplanten Maßnahmen ein
  erhöhtes Insolvenzrisiko des Unternehmens im Vergleich zu Unternehmen
  die geringerem Wettbewerbsdruck unterliegen
\item
  Beschreibe Folgen von Insolvenz auf Vorstandleitung (Verlust des Amts,
  Verlust von Vergütungansprüchen, Gefahr von Haftungsrisiken und
  sonstigen rechtlichen Ansprüchen gegen Vorstand, Reputationsschaden)
\item
  Bei starken Wettbewerbsdruck können Mitwettbewerber des Unternehmens
  identifiziert werden, um deren Performance zum Zwecke des
  Benchmarkings zu verwenden. Die Perfomance des Unternehmens lässt sich
  durch die Berücksichtigung der Performance der Gesamtbranche besser
  einschätzen (Arbeitseinsatz des Vorstands wird besser messbar für
  Prinzipale). Die Verwendung einer relativen Bemessungsgrundlage unter
  Berücksichtigung der Gesamtbranchenentwicklung dient dem Abbau von
  Informationsasymmetrien, die Leistung des Unternehmensvorstands wird
  besser beurteilbar wodurch gleichzeitig eine verstärkte Anreizwirkung
  auf den Vorstand entsteht
\end{itemize}

\hypertarget{uxfcberwachung-durch-den-kapitalmarkt}{%
\section{Überwachung durch den
Kapitalmarkt}\label{uxfcberwachung-durch-den-kapitalmarkt}}

Auch die Überwachung von externen Akteuren auf dem Kapitalmarkt kann
eine Anreizwirkung auf den Vorstand von börsennotierten Unternehmen
ausüben

Informationsgrundlage der externen Akteure am Kapitalmarkt: Entwicklung
des Aktienkurses des entsprechenden Unternehmens (Überwachung sowohl ex
post als auch es ante möglich)

Ex Ante: Geplante Expansionsvorhaben des Vorstands lassen sich von
externen Akteuren ex ante anhand der kurzfristigen Kursentwicklung des
expandierenden Unternehmens überwachen

Ex Post: Die langfristige Kursentwicklung lässt sich ex post als
Bewertungsmaßstab für den Erfolg der Expansionspolitik des Unternehmens
verwenden

Anreizwirkung für den Vorstand ergibt sich aus dem Risiko der Ablösung
bei negativer Kursentwicklung sowie einer Einbuße an
Vergütungsansprüchen (falls diese Kursabhängig ist)

\hypertarget{unternehmensuxfcbernahmen-als-cg-mechanismus}{%
\subsection{Unternehmensübernahmen als CG
Mechanismus}\label{unternehmensuxfcbernahmen-als-cg-mechanismus}}

Die Qualität der Vorstandsarbeit hat einen unmittelbaren Einfluss auf
den Aktienkurs eines Unternehmens (schlechte Vorstandsarbeit verringert
den Unternehmenswert)

Externe Investoren haben den Anreiz niedrig bewertete Unternehmen zu
übernehmen, falls die niedrige Bewertung des Unternehmens durch
schlechte Vorstandsarbeit zu Stande gekommen ist

Externer Investor kann profitables Geschäft machen durch Übernahme des
entsprechenden Unternehmens und anschließende Neubesetzung des Vorstands

Vorstand hat Anreiz gute Arbeit zu leisten, da er bei niedriger
Unternehmensbewertung auf Grundlage seiner Arbeitsleistung das Risiko
einer externen Übernahme erhöht und somit sein Amt gefährdet

\textbf{Die Gefahr besteht primär, wenn die interne Kontrolle des
Unternehmens durch den Aufsichtsrat bereits versagt hat }(dieser sollte
eigentlich bereits vorher aktiv werden)

\textbf{Fallanwendung?}

Sachverhalt: AG kündigt Expansionsvorhaben an und der Aktienkurs fällt
noch am selben Tag um 10\%

\begin{itemize}
\tightlist
\item
  Kapitalmarkt bewertet Expansionsvorhaben negativ, falls
  Vorstandsvergütung aktienkursabhängig ist ergibt sich bereits hier ein
  Anreiz für den Vorstand die Expansionsvorhaben zu überdenken
\item
  Falls weiterer Festhalt an Expansionsvorhaben muss die Entwicklung des
  Aktienkurses in der Folgezeit betrachtet werden. Falls sich der Kurs
  nicht wieder erholt, entstehen Anreize für externe Investoren zur
  Übernahme der AG (lukrative Übernahme falls Investor durch Austausch
  des Vorstands die Expansionsvorhaben der AG verhindern kann)
\item
  Die Gefahr der Übernahme durch externe Investoren entfaltet eine
  weitere Anreizwirkung auf den Vorstand der AG seine Expansionsvorhaben
  zu überdenken
\end{itemize}

\hypertarget{vorstandsverguxfctung}{%
\section{Vorstandsvergütung}\label{vorstandsverguxfctung}}

\hypertarget{verguxfctungsstruktur-verguxfctungsbestandteile}{%
\subsection{Vergütungsstruktur/
Vergütungsbestandteile}\label{verguxfctungsstruktur-verguxfctungsbestandteile}}

Die Vergütung des Vorstands lässt sich primär in die zwei folgenden
Gruppen einordnen:

\begin{longtable}[]{@{}ll@{}}
\toprule
\endhead
Fixe Bestandteile & Variable Bestandteile \\
Fest vereinbarte Vergütungen, die nicht vom Eintritt bestimmter
Bedingungen abhängig sind & Alle Vergütungen, die nicht fest vereinbart
sind und deren Zahlung vom Eintritt bestimmter Bedingungen abhängig
sind \\
& gesperrte Aktien oder gesperrte Aktienoptionen \\
\bottomrule
\end{longtable}

Kennzahl: Anteil der variablen Vergütung an der Gesamtvergütung

\begin{itemize}
\tightlist
\item
  Europäischer Durchschnitt 50/50
\item
  USA Durchschnitt 70/30 (variabel/fix)
\end{itemize}

\textbf{Struktur von Bonuszahlungen}

\begin{itemize}
\tightlist
\item
  Bemessungsgrundlage sind i.d.R. Gewinngrößen
\item
  Häufig werden Unter- und Obergrenzen (caps \& floors) festgelegt
\item
  Höhe der Bonuszahlung hängt vom Grad der Zielerreichung der BMG ab
  (oft linearer Zusammenhang)
\item
  Caps \& Floors, falls eingesetzt, beschränken die Anreizwirkung auf
  Vorstand
\end{itemize}

\textbf{Aktienbasierte Vergütungen}

\begin{itemize}
\tightlist
\item
  Sowohl feste als auch variable Vergütung kann in Form von Aktien oder
  Aktienoptionen erfolgen
\item
  Bei Einsatz von Sperrfristen für die Veräußerung entsteht eine
  wünschenswerte Anreizwirkung auf Vorstandsmitglieder. Der Wert ihrer
  Vergütung hängt von der zukünftigen Wertentwicklung des Unternehmens
  ab
\item
  Vorstände werden sowohl an positiven als auch an negativen
  Kursentwicklungen beteiligt
\end{itemize}

\hypertarget{gestaltung-der-verguxfctungsstruktur-mit-ausrichtung-auf-eine-nachhaltige-unternehmensfuxfchrung}{%
\subsection{Gestaltung der Vergütungsstruktur mit Ausrichtung auf eine
nachhaltige
Unternehmensführung}\label{gestaltung-der-verguxfctungsstruktur-mit-ausrichtung-auf-eine-nachhaltige-unternehmensfuxfchrung}}

Börsennotierte Unternehmen sind nach AktG §87 dazu angehalten ihre
Vergütungsstruktur auf eine \textbf{nachhaltige Unternehmensführung}
auszurichten

\begin{itemize}
\item
  Empfehlung zur Verwendung von mehrjährigen Bemessungsgrundlagen
\item
  Der Aufsichtsrat soll eine Begrenzungsmöglichkeit nach oben (=cap) für
  außerordentlich hohe Entwicklungen der BMG festlegen

  \begin{itemize}
  \tightlist
  \item
    Problematisch bei aktienbasierten Vergütungen!
  \item
    DCGK sieht Begrenzung nur auf die Gewährung vor, AktG jedoch auch
    auf die Auszahlung!
  \end{itemize}
\end{itemize}

\textbf{Mehrjährige Bemessungsgrundlagen} (Beispiele)

\begin{itemize}
\tightlist
\item
  Durchschnittlicher Gewinn der letzten 3 Perioden
  (=vergangenheitsbezogen)
\item
  Durchschnittlicher Gewinn der aktuellen sowie der zwei folgenden
  Perioden (=zukunftsorientierte Durchschnittsbildung mit nachgelagerter
  Auszahlung)
\item
  Aktienbasierte Vergütung auf Basis der zukünftigen Kursentwicklung (ab
  einer Sperrfrist von mindestens 2-3 Jahren ist die Bedingung einer
  zukunftsorientierten mehrjährigen BMG erfüllt
\end{itemize}

Mindestens 2-3 Perioden für Klassifizierung als langfristige BMG
notwendig!

Langfristige, variable und aktienbasierte Entlohnung nach DCGK erfüllt
die Kriterien der mehrjährigen BMG nach AktG (Sperrfrist ab 4 Jahre)

\hypertarget{verguxfctungssystem-nach-dcgk}{%
\subsection{Vergütungssystem nach
DCGK}\label{verguxfctungssystem-nach-dcgk}}

Soll Vorschrift nach DCGK (G1 \& G6)

\begin{enumerate}
\def\labelenumi{\arabic{enumi}.}
\item
  Festlegung von \textbf{angemessener Ziel- und maximaler
  Gesamtvergütung} für jedes einzelne Vorstandsmitglied
\item
  Festlegung von \textbf{Festvergütung} (=fixe Bestandteile),
  \textbf{kurzfristiger variabler Vergütung} sowie \textbf{langfristiger
  variabler Vergütung}
\item
  Der Anteil der langfristigen variablen Vergütung soll den der
  kurzfristigen variablen Vergütung übersteigen
\item
  Festlegung der Leistungskriterien die für die Gewährung von variablen
  Vergütungsbestandteilen ausschlaggebend sind (=Bemessungsgrundlagen).

  \begin{enumerate}
  \def\labelenumii{\alph{enumii}.}
  \tightlist
  \item
    Finanziellen und nichtfinanzielle Kriterien
  \item
    Operative und strategische Kriterien
  \end{enumerate}
\item
  Zusammenhang zwischen dem Grad der Zielerreichung der festgelegten
  Leistungskriterien und der Höhe der gewährten variablen Vergütung
\item
  Auszahlungszeitpunkt (unmittelbar/verzögert) und Art der Auszahlung
  (bar/Aktien)
\end{enumerate}

Aktienbasierte Entlohnung wird durch den DCGK ab einer Sperrfrist von 4
Jahren als langfristige variable Entlohnung charakterisiert

** Fallanwendung?**

\begin{itemize}
\tightlist
\item
  Kategorisierung aller Vergütungsbestandteile in fixe, kurzfristig
  variable und langfristige variable Vergütungsbestandteile
\item
  Prüfe ob Vorgaben aus AktG erfüllt sind (=Verwendung von mehrjährigen
  Bemessungsgrundlagen und caps zur Ausrichtung auf nachhaltige
  Unternehmensführung)
\item
  Prüfe ob DCGK Empfehlungen umgesetzt wurden (Ziel- und
  Gesamtvergütung, langfristige variable Vergütung höherer Anteil als
  kurzfristiger variabler Anteil, etc.)
\end{itemize}

\hypertarget{angemessenheit-der-vorstandsverguxfctung}{%
\subsection{Angemessenheit der
Vorstandsvergütung}\label{angemessenheit-der-vorstandsverguxfctung}}

Ausgangslage: Starker Anstieg der Vorstandsvergütung seit den 1990er
Jahren für CEOs von Top 50 US-Unternehmen

\textbf{Grundsätzlich}:

\begin{itemize}
\tightlist
\item
  Aufsichtsrat ist für die Festlegung der Höhe von Vorstandsvergütung
  zuständig
\item
  Vorstand hat jedoch Möglichkeit Einfluss auf AR zu nehmen (Ausnutzung
  der eigenen Macht/ Einfluss sowie von Informationsvorteilen)
\end{itemize}

\textbf{Begründung für die Höhe von Vorstandsgehältern?}

\begin{itemize}
\tightlist
\item
  Wertbeitrag eines CEOs ist abhängig von der Unternehmensgrößte
  (=relativer Wertbeitrag)
\item
  Top CEOs (=``Superstar CEOs'') leisten in der Regel einen höheren
  relativen Wertbeitrag für das Unternehmen
\item
  Top CEOs sind gleichzeitig selten und besitzen dadurch einen hohe
  Verhandlungsmacht
\end{itemize}

\textbf{Rechenbeispiel}

Top CEO leistet einen relativen Wertbeitrag der 1\% höher ist als der
von durchschnittlichen CEO Kandidaten

Bei einem ursprünglichen Unternehmenswert von 10 Mrd. US-\$ entspricht
dieser Mehrwert einer Wertsteigerung von 100 Mio. Euro für das
Unternehmen

Der starke Zusammenhang zwischen der Marktkapitalisierung und der Höhe
der Vorstandsvergütung deutet darauf hin, dass hohen Vergütungen für
Topmanager den Markpreisen ihrer Wertsteigerung entsprechen

\hypertarget{angemessene-vorstandsverguxfctungen-nach-aktg}{%
\subsection{Angemessene Vorstandsvergütungen nach
AktG}\label{angemessene-vorstandsverguxfctungen-nach-aktg}}

Das AktG schreibt vor, dass die gewährten Gesamtbezüge der einzelnen
Vorstandsmitglieder sich an seinen \textbf{Aufgaben} und
\textbf{Leistungen} sowie der (gesamtwirtschaftlichen\textbf{) Lage der
Gesellschaft} zu orientieren haben

\textbf{Aufgaben}

\begin{itemize}
\tightlist
\item
  Berücksichtigung von Verantwortlichkeiten der Vorstandsmitglieder
  (Höhere Vergütung bei besonders verantwortungsvollen
  Ressortzuständigkeiten, Vorstandsvorsitz, etc.)
\item
  Auch die Fähigkeiten und Erfahrungen, die mit der Aufgabenübertragung
  einhergehen sind bei der Vergütungsfestlegung für einzelne
  Vorstandsmitglieder zu berücksichtigen
\end{itemize}

\textbf{Leistungen}

\begin{itemize}
\tightlist
\item
  Leistungen der Vorstandsmitglieder werden primär durch die variablen
  Vergütungsbestandteile berücksichtigt
\item
  Arbeitseinsatz ist i.d.R. nicht direkt beobachtbar und aussagekräftig
  weshalb häufig Erfolgsgrößen (z.B. Jahresüberschuss, EBIT) als
  Bemessungsgrundlage berücksichtigt werden
\item
  Alternativ kann gute Leistung aber auch durch die Erhöhung der
  Fixvergütung in den folgenden Perioden berücksichtigt werden
\end{itemize}

\textbf{Lage der Gesellschaft}

\begin{itemize}
\tightlist
\item
  Größe der Gesellschaft (Vorstand einer großen Gesellschaft erhält
  höhere Vergütung als der einer kleineren Gesellschaft)
\item
  Branchenbesonderheiten
\item
  Ertragslage der Gesellschaft (eine schlechte Ertragslage kann jedoch
  unter Umständen ebenfalls eine Erhöhung der Vergütung erforderlich
  machen auf Grund des höheren Arbeitsaufwands und dem erhöhten
  Insolvenzrisikos, das der Vorstand trägt)
\end{itemize}

Eine angemessene Vorstandsvergütung nach AktG berücksichtigt die
Faktoren Aufgaben, Leistungen und Lage bei der Festlegung

\textbf{Fallanwendung}?

\begin{itemize}
\tightlist
\item
  Gesamtbezüge und Unterschiede zwischen den einzelnen
  Vorstandsmitgliedern auflisten
\item
  Angemessene Vergütung nach AktG beschreiben und die drei grundlegenden
  Einflussfaktoren beschreiben
\item
  Aufgaben: Ressortzuständigkeiten auf Anspruch, getragene Verantwortung
  sowie notwendige Fähigkeiten und Erfahrung überprüfen
\item
  Leistungen: Analyse der Vergütungsmechanismen von vergangenen und
  künftigen Leistungen der Vorstandsmitglieder (Bonus und
  Bemessungsgrundlage sowie aktienbasierte Vergütung)
\item
  Lage: Ertragslage des Unternehmens, Größe und eventuelle
  Branchenspezifika angemessen?
\end{itemize}

\hypertarget{pruxfcfung-der-angemessenheit-der-vorstandsverguxfctung-anhand-der-uxfcblichkeit}{%
\subsection{Prüfung der Angemessenheit der Vorstandsvergütung anhand der
Üblichkeit}\label{pruxfcfung-der-angemessenheit-der-vorstandsverguxfctung-anhand-der-uxfcblichkeit}}

Eine \textbf{angemessene} Vorstandsvergütung darf die \textbf{übliche}
Vergütung nicht ohne besonderen Grund überschreiten (AktG §87)

Die übliche Vergütung stellt eine \textbf{weiche Obergrenze} für die
Höhe der Gesamtbezüge eines Vorstandsmitglieds dar. Bei besonderen
Gründen kann die Abweichungen von der üblichen Vorstandsvergütung
deshalb trotzdem noch als angemessen gelten.

Besondere Gründe: Z.B. die intensive Konkurrenz am internationalen Markt
für Führungskräfte kann eine Erhöhung über landesübliche
Vergütungsansprüche erforderlich machen

\textbf{Zwei Möglichkeiten der Prüfung }

\textbf{Horizontaler Vergleich}: Vergleich der Bezüge von Vorständen mit
den Bezügen anderer Vorstände in vergleichbaren Unternehmen. Im
Idealfall stammen die Vergleichsvorstände aus der gleichen Branche und
weisen eine ähnliche Unternehmensgröße auf, die Vergleichsgruppe bildet
die \textbf{Peer Group}. Landesüblichkeiten sollen ebenfalls
berücksichtigt werden und Unternehmen sind nach DCGK dazu angehalten die
Zusammensetzung der Peer Group transparent zu machen

\begin{itemize}
\item
  \textbf{Schwierigkeiten bei Bestimmung der Peer Group}: Unternehmen
  sollen „passgenau'' ausgewählt werden, bei zu strenger Auswahl läuft
  man jedoch der Gefahr auf eine zu kleine Peer Group zu erstellen,
  deren Vergleich nicht aussagekräftig ist. Auch kann die Peer Group
  strategisch ausgewählt werden, um eine höhere Vergütung zu ermöglichen
\item
  \textbf{Bestimmung der konkreten Vergütung anhand des relevanten
  Werts}: Die Vergütung anhand des Mittelwerts kann zu einer Gefahr des
  Verlusts von überdurchschnittlich guten Managern führen, die Vergütung
  oberhalb des Mittelwerts kann schnell zu einer allgemeinen
  Aufwärtsspirale der Vorstandsvergütung der gesamten Branche führen
\end{itemize}

\textbf{Vertikaler Vergleich}: Unternehmensinterner Vergleich zwischen
Vorstandsvergütung und der Vergütung der übrigen Belegschaft. Die
Vorstandsvergütung wird mit den Vergütungen der oberen Führung sowie der
Gesamtbelegschaft ins Verhältnis gesetzt, um daraus eine Kennzahl zu
ermitteln. Die Entwicklung dieser Kennzahl soll nach DCGK über den
Zeitverlauf als Maßstab zur Beurteilung der Üblichkeit herangezogen
werden

\begin{itemize}
\item
  \textbf{Schwierigkeiten bei der Verwendung des vertikalen Vergleichs}:
  Nur begrenzte Anwendbarkeit in der Praxis da sehr unterschiedliche
  Verantwortungen und Qualifikationen für Vorstände und die restliche
  Belegschaft
\item
  Die \textbf{Festlegung einer üblichen Höhe} der ermittelten Kennzahl
  ist ebenfalls problematisch, da sehr große Unterschiede in
  Abhängigkeit von Geschäftsmodell und organisatorischem Aufbau des
  Unternehmens auftreten
\end{itemize}

\textbf{Fallanwendung?}

\begin{itemize}
\tightlist
\item
  Berechne falls notwendig die Höhe der Gesamtbezüge für
  Vorstandsmitglieder sowie den Mittelwert der Peer Group
\item
  Beschreibe, dass Üblichkeit als weiche Obergrenze zu Prüfung der
  Angemessenheit zu verwenden ist
\item
  Horizontaler Vergleich: Falls Vergütung über dem üblichen Mittelwert
  liegt, ist zu prüfen ob es besondere Gründe für die Überschreitung der
  Üblichkeit gibt (Zusammenstellung/ Vergleichbarkeit zwischen
  Unternehmen und Peer Group, Notwendigkeit der Zahlung von höheren
  Vergütungen durch internationaler Wettbewerb um Führungskräfte?)
\end{itemize}

\hypertarget{anreizstuxe4rke}{%
\section{Anreizstärke}\label{anreizstuxe4rke}}

Ausgangslage: Nach der Prinzipal-Agenten Theorie können Anreize aus der
Vergütung von Vorstandsmitgliedern deren Interessenskonflikte mit
Aktionären und dem Aufsichtsrat verringern

Arbeitseinsatz des Agenten ist als Bemessungsgrundlage nicht sinnvoll/
beobachtbar

Vorstand soll angereizt werden im Sinne der Prinzipale zu handeln

\textbf{Anreizkompatibilität}: Vorstand erhält höhere Vergütung immer
dann, wenn auch die Aktionäre einen höheren Nutzen erfahren

Anreizkompatibilität ist abhängig von der Wahl der Bemessungsgrundlage
für variable Vergütungsbestandteile, benötigt wird ein positiver
Zusammenhang mit der Nutzenfunktion der Aktionäre

\textbf{Formale Bedingungen an Bemessungsgrundlage} (x=Arbeitseinsatz;
y=Erfolgsmaß)

u(Prinzipal) = f(y) Anreizkompatibilität

y = f(x), z.B. y=x + e mit e=sonstige Faktoren Beeinflussbarkeit der
Erfolgsgröße durch Agenten

y =/ f (x, z) mit z=verdeckte Faktoren Manipulationsfreiheit

\textbf{Nutzen des Aktionärs}: Anlagenrendite = Kurssteigerung +
Dividendenzahlungen

Die Wahl des Aktienkurses als Bemessungsgrundlage der variablen
Vergütung des Vorstands stellt Anreizkompatibilität durch den positiven
Zusammenhang mit der Nutzenfunktion des Aktionärs sicher. Die Entlohnung
des Vorstands anhand von gesperrten Aktien und Aktienoptionen erfüllt
diese Bedingung ebenfalls

Gewinngrößen (z.B. EBIT oder Jahresüberschuss): Kein unmittelbarer
Zusammenhang mit Aktienkurs, Anreizkompatibilität daher fragwürdig

\hypertarget{kennzahlen-zur-ermittlung-der-anreizstuxe4rke}{%
\section{Kennzahlen zur Ermittlung der
Anreizstärke}\label{kennzahlen-zur-ermittlung-der-anreizstuxe4rke}}

\hypertarget{relative-anreizstuxe4rke-pay-performance-sensitivity}{%
\subsection{Relative Anreizstärke / Pay-Performance
Sensitivity}\label{relative-anreizstuxe4rke-pay-performance-sensitivity}}

Maß für die Übereinstimmung von Interessen zwischen Prinzipal und Agent,
wie stark ist der Agent an dem Anlageerfolg der Aktionäre beteiligt?

Interpretation: Wie stark partizipiert der Vorstand von einer Erhöhung
des Aktienkurses um 1€?

Bestandteile: Effektiver Anteilsbesitz des Vorstands (gesperrte Aktien
oder gesperrte Aktienoptionen aus Gegenwart und vergangenen Perioden
solange noch gesperrt), der sich aus der variablen Vergütung ergibt

Relative Anreizstärke = Anzahl Aktien des Vorstandsmitglieds /
Gesamtzahl ausstehender Aktien

Relatives Ergebnis in \%

\textbf{Rechenbeispiel}

Vergütungsbestandteile die Erfolgskennzahlen (z.B. Jahresüberschuss,
EBIT) zur Bemessung verwenden werden, falls vorhanden, nicht beachtet
für die Berechnung!

Wie viele Aktien und Aktienoptionen besitzen die Vorstandsmitglieder
unter Auflage einer Mindesthaltedauer?

\begin{itemize}
\tightlist
\item
  Jedes Vorstandsmitglied 40 000 Stück aus vergangenen Perioden und 40
  000 Stück aus der aktuellen Periode (jeweils mit laufender Sperrfrist)
\item
  Gesamtzahl ausstehender Aktien 200 000 000 des Unternehmens
\end{itemize}

Relative Anreizstärke = 80 000 / 200 000 000 = 0,0004 = 0,04\%

Interpretation: Ein Vorstandsmitglied ist anhand seiner aktienbasierten
Vergütung mit einem effektiven Anteil von 0,04\% am Unternehmenserfolg
beteiligt

\textbf{Alternative Formel bei gesperrten Aktienoptionen}

Relative Anreizstärke = (Anzahl Aktien Vorstand + Aktienoptionen *
Optionsdelta) / Gesamtzahl Aktien

\textbf{Beobachtete Werte}

\begin{itemize}
\tightlist
\item
  USA Durchschnitt: 0,5\% (Deutschland niedriger)
\item
  Starker Anstieg in den 1990er Jahren, Rückgang seit 2002
\end{itemize}

\hypertarget{absolute-anreizstuxe4rke-equity-at-stake}{%
\subsection{Absolute Anreizstärke / Equity at
Stake}\label{absolute-anreizstuxe4rke-equity-at-stake}}

Neben der relativen Beteiligung des Vorstands am Unternehmenserfolg
lässt sich auch die absolute Beteiligung berechnen

Absolute Anreizstärke: Wie ändert sich das Vermögen des Vorstands bei
einem Anstieg der Unternehmensaktie um 1\%?

\textbf{Berechnungsmöglichkeiten Equity at Stake}

\begin{enumerate}
\def\labelenumi{\arabic{enumi}.}
\tightlist
\item
  =Anzahl Aktien des Vorstands * Aktienkurs * 1\%
\item
  =Effektiver Anteilsbesitz * Anzahl Aktien gesamt * Aktienkurs * 1\%
\item
  =Effektiver Anteilsbesitz * Marktkapitalisierung * 1\%
\item
  =Vermögen des Vorstands / Kursänderung in \%
\end{enumerate}

\textbf{Beobachtete Werte}

\begin{itemize}
\tightlist
\item
  500.000\$ im Durchschnitt für S\&P 500 Unternehmen
\end{itemize}

Trotz niedriger relativer Anreizstärke von börsennotierten Unternehmen
(Tendenz eher sinkend) finden sich sehr hohe absolute Anreizstärken

\textbf{Rechenbeispiel }

Aktienbesitz Vorstandsmitglied: 80.000 gesperrte Aktien

Aktienkurs: 50 €

Gesamtzahl ausstehender Aktien: 200 000 000

Absolute Anreizstärke = 80.000 * 50€ *1\% =65.000€

Alternative Berechnung (basierend auf 0,065\% relativer Anreizstärke)

Absolute Anreizstärke = 0,065\% * 200 000 000 *50€ *1\% =65.000€

Interpretation: Ein Kursanstieg der Unternehmensaktie führt dazu, dass
sich das Vermögen eines jeden Vorstandsmitglieds um 65.000€ erhöht

\hypertarget{corporate-social-responsibility}{%
\section{Corporate Social
Responsibility}\label{corporate-social-responsibility}}

Definition: \textbf{Die Verantwortung von Unternehmen für die Auswirkung
ihrer Tätigkeiten auf die Gesellschaft}. Bestandteil der CSR sind
\textbf{sämtliche Aktivitäten}, die dazu dienen dieser Verantwortung
gerecht zu werden

CSR Aktivitäten lassen sich in die folgenden \textbf{zwei Kategorien}
einteilen:

\begin{enumerate}
\def\labelenumi{\arabic{enumi}.}
\tightlist
\item
  Beachtung von \textbf{rechtlichen Vorgaben }
\item
  Beachtung von \textbf{freiwilligen Standards} (z.B. UN Global Compact)
\end{enumerate}

Die \textbf{vier Dimensionen} der CSR umfassen:

\begin{itemize}
\tightlist
\item
  \textbf{Menschenrechte}: Achtung der internationalen Menschenrechte im
  allgemeinen und Sicherstellung, dass das Unternehmen keine Mitschuld
  an Menschenrechtsverletzungen trägt
\item
  \textbf{Arbeitnehmerbelange}: Das Recht auf Kollektivverhandlungen
  wird sichergestellt und das Unternehmen setzt sich gegen Zwangsarbeit,
  Kinderarbeit und Diskriminierung ein
\item
  \textbf{Umweltbelange}: Förderung des Umweltbewusstseins sowie
  Entwicklung und Verbreitung von umweltfreundlichen Technologien
\item
  \textbf{Korruptionsbekämpfung}: Unternehmen setzt sich gegen
  Korruption, Erpressung und Bestechung ein
\end{itemize}

\textbf{Verantwortlichkeit?}

Der Vorstand trägt als rechtlicher Vertreter des Unternehmens die
Verantwortung dafür sicherzustellen, dass rechtliche Vorgaben und
freiwillig eingegangenen Selbstverpflichtungen eingehalten werden

Die Gesamtheit der Grundsätze und Maßnahmen eines Unternehmens zur
Einhaltung bestimmter Regeln und damit zur Vermeidung von Regelverstößen
wird durch den DCGK als Compliance Management System bezeichnet
(rechtliche Vorgaben \& freiwillige Standards, zu denen sich das
Unternehmen verpflichtet hat)

Zur Sicherstellung der Erfüllung seiner Verantwortung richtet der
Vorstand i.d.R. ein \textbf{Compliance-Management-System} ein

\textbf{Bestandteile von Compliance-Management-Systemen}
(beispielsweise)

\begin{itemize}
\tightlist
\item
  Unternehmenskultur und insb. Integrität der Unternehmensleitung
\item
  Systematische Identifikation von relevanten rechtlichen Vorgaben (hohe
  Komplexität insb. bei international tätigen Konzernen)
\item
  Kommunikation von Vorschriften, Erstellung und Verbreitung von
  Richtlinien und Schulungen
\item
  Überwachung der Einhaltung und Regelbeachtung (z.B. durch interne
  Revisionen und Whistleblower-Hotlines)
\end{itemize}

Die Beachtung der CSR ist Bestandteil der Compliance Aufgabe

\textbf{Ausprägungen der CSR}

\begin{itemize}
\tightlist
\item
  Compliance: Beachtung von rechtlichen Anforderungen
\item
  Strategische CSR
\item
  Präferenzbasierte CSR
\end{itemize}

\hypertarget{strategische-csr-profitable-csr-mauxdfnahmen}{%
\subsection{Strategische CSR (profitable CSR
Maßnahmen)}\label{strategische-csr-profitable-csr-mauxdfnahmen}}

Strategische CSR Maßnahmen beschreiben Tätigkeiten des Unternehmens, die
positive Auswirkungen auf alle Stakeholder inklusive der Aktionäre des
Unternehmens haben

„Planet, People, Profit''

``Doing Well by Doing Good''

Strategische CSR Maßnahmen sollen zu einer „Win-Win'' Situation führen,
die Verantwortung des Unternehmens wird erfüllt und gleichzeitig
profitieren Shareholder und Stakeholder von diesen Maßnahmen (simultane
Zielerfüllung)

\textbf{Beispiele für strategische CSR}

\begin{itemize}
\tightlist
\item
  Unternehmen führt verbesserte Umweltschutzmaßnahmen ein durch deren
  Beachtung die Abfallmenge reduziert wird und weniger Entsorgungskosten
  anfallen
\item
  Verbesserung des Arbeitsschutzes führt zu sinkender Zahl von Fehltagen
  und somit zu einer verbesserten Produktionsleistung
\end{itemize}

\textbf{Weitere Möglichkeiten für Profitabilitätssteigerung durch
strategische CSR}

\begin{itemize}
\tightlist
\item
  Beachtung von Stakeholder Interessen führt zu Innovation und neuen
  Geschäftsfeldern
\item
  Verbesserung der Reputation bei Kunden und Geschäftspartnern führen zu
  Wettbewerbsvorteilen gegenüber der Konkurrenz (Voraussetzung dafür ist
  jedoch eine glaubhafte Kommunikation der CSR Maßnahmen an Stakeholder
  und ein geringerer Kostenaufwand der CSR Maßnahme im Vergleich zum den
  zusätzlichen Gewinnen)
\item
  Verbesserung der Arbeitnehmerbelange führt zu stärker
  Mitarbeiterzufriedenheit (Vorteile in der Personalbeschaffung,
  Arbeitseinsatz etc.)
\item
  Gute Reputation kann als Schutz gegen Öffentlichkeitsdruck dienen und
  das Unternehmen vor negativen Kampagnen beschützen
\end{itemize}

\textbf{Fallanwendung?}

\begin{itemize}
\tightlist
\item
  Erfüllt Unternehmen nur die rechtlichen Vorgaben handelt es sich um
  eine normale Compliance Maßnahme zur Sicherstellung der
  Regeleinhaltung
\item
  Unternehmen „übertrifft'' die rechtlichen Vorgaben (Stakeholder
  profitieren in einer der vier Dimensionen der CSR) CSR Maßnahme
\item
  Wenn der Nettogewinn aus der Umsetzung einer CSR Maßnahme für das
  Unternehmen und die Shareholder positiv ist, handelt es sich um eine
  strategische CSR Maßnahme
\end{itemize}

\hypertarget{pruxe4ferenzbasierte-csr-unprofitable-csr-mauxdfnahmen}{%
\subsection{Präferenzbasierte CSR (unprofitable CSR
Maßnahmen)}\label{pruxe4ferenzbasierte-csr-unprofitable-csr-mauxdfnahmen}}

Präferenzbasierte CSR Maßnahmen wirken sich positiv auf einer der vier
Dimensionen der CSR aus, führen aber im Gegensatz zur strategischen CSR
zu keiner Verbesserung oder zu einer Verringerung der Profitabilität

Beispiel: Innovationen im Produktionsverfahren wirkt sich positiv auf
die Umwelt aus (weniger Schadstoffe z.B.) aber führt zu keiner
signifikanten Kostenreduktion

Voraussetzung für die Durchführung von präferenzbasierten CSR Maßnahmen
ist die Unterstützung durch Shareholder. Bei fehlender Akzeptanz können
diese den Aufsichtsrat unter Druck setzten, um den Vorstand zu einer
Abkehr von der entsprechenden Maßnahme zu bewegen

\textbf{Unter welchen Umständen unterstützen Aktionäre präferenzbasierte
CSR Maßnahmen?}

\begin{itemize}
\tightlist
\item
  Aktionäre besitzen soziale Präferenzen (Geldanlage erfolgt nicht
  ausschließlich zur finanziellen Bereicherung, soziale Belange oder
  Nachhaltigkeit sind ebenfalls von Bedeutung)
\item
  Resultierender Nutzen (finanzieller und nicht-finanzieller Nutzen) der
  Aktionäre aus CSR Maßnahme übersteigt deren Kosten
\item
  Die Maßnahme ist kostengünstiger durch das Unternehmen umsetzbar als
  durch den Aktionär selbst (Aktionäre kennt seine eigenen sozialen
  Präferenzen und kann ansonsten durch Spenden etc. gezielter seine
  sozialen Ziele verfolgen während das Unternehmen wirtschaftliche Ziele
  verfolgt)
\end{itemize}

\textbf{Fallanwendung}?

\begin{itemize}
\tightlist
\item
  Handelt es sich um profitable oder unprofitable CSR Maßnahme?
\item
  Falls unprofitable CSR Maßnahme beschreibe auf welcher Dimension
  Vorteile für welche Stakeholder entstehen
\item
  CSR Maßnahme sollte aus Sicht der Aktionäre nur dann durchgeführt
  werden, wenn Aktionäre ebenfalls soziale Präferenzen bei ihrer
  Kapitalanlage verfolgen, der soziale Mehrwert die finanziellen
  Nachteile aufwiegt und die Maßnahme durch das Unternehmen außerdem
  effizienter und/oder kostengünstiger ausgeführt werden kann als durch
  die Aktionäre selbst
\end{itemize}

\hypertarget{herausforderungen-der-umsetzung-von-csr-mauxdfnahmen-aus-sicht-der-corporate-governance}{%
\section{Herausforderungen der Umsetzung von CSR Maßnahmen aus Sicht der
Corporate
Governance}\label{herausforderungen-der-umsetzung-von-csr-mauxdfnahmen-aus-sicht-der-corporate-governance}}

Ziel der Corporate Governance: Vorteilhafte CSR Maßnahmen sollen von der
Unternehmensleitung durchgeführt werden

Welche Anreize wirken dabei auf den Vorstand?

\begin{longtable}[]{@{}
  >{\raggedright\arraybackslash}p{(\columnwidth - 2\tabcolsep) * \real{0.49}}
  >{\raggedright\arraybackslash}p{(\columnwidth - 2\tabcolsep) * \real{0.49}}@{}}
\toprule
\endhead
CSR als Bestandteil der Compliance Aufgabe (Einhaltung von relevanten
rechtlichen Vorschriften und freiwilligen Standards) & Entscheidend für
die Anreizwirkung ist in allen Fällen die konsequente Überwachung und
Sanktionierung von Fehlverhalten \\
Strategische CSR (profitable CSR) & \\
Präferenzbasierte CSR (unprofitable CSR) & Es ergeben sich vielfältige
Herausforderungen bei der Förderung von unprofitablen CSR Maßnahmen
durch Corporate Governance Mechanismen \\
\bottomrule
\end{longtable}

\hypertarget{modell-des-entscheidungskalkuxfcls-pruxe4ferenzbasierter-csr}{%
\subsection{Modell des Entscheidungskalküls präferenzbasierter
CSR}\label{modell-des-entscheidungskalkuxfcls-pruxe4ferenzbasierter-csr}}

Entscheidung ob unprofitable CSR Maßnahme durchgeführt werden soll
basiert auf der Kosten-Nutzenabwägung der Aktionäre und basiert auf den
folgenden Komponenten:

\begin{itemize}
\tightlist
\item
  Finanzieller Nutzen der Maßnahme für Shareholder
\item
  Sozialer Nutzen der Maßnahme für Stakeholder (Auswirkungen auf die
  Allgemeinheit)
\end{itemize}

\textbf{Modell}: Entscheidung zwischen Technologie A (=sauber) und
Technologie B (=dreckig). Entscheidung für dreckige Technologie lässt
einen Schaden d entstehen, der zu Lasten der Allgemeinheit eintritt

Es gilt:

\begin{enumerate}
\def\labelenumi{\arabic{enumi}.}
\tightlist
\item
  Π(sauber) \textless{} Π(dreckig) dreckige Technologie ist profitabler
\item
  Π(dreckig) -- d \textless{} Π(sauber) Gesamtwohlfahrt der sauberen
  Technologie ist größer
\end{enumerate}

\begin{longtable}[]{@{}llll@{}}
\toprule
\endhead
Technologie & Profit & Schaden & Wohlfahrt \\
A (sauber) & Π(sauber) & 0 & Π(sauber) \\
B (dreckig) & Π (dreckig) & d & Π(dreckig) -- d \\
\bottomrule
\end{longtable}

\textbf{Szenario 1: Aktionäre besitzen keine sozialen Präferenzen }

Entscheidung basiert allein auf finanziellem Nutzen und Aktionäre wählen
den Einsatz der dreckigen Technologie da Π(dreckig)\textgreater{}
Π(sauber)

\textbf{Szenario 2: Aktionäre besitzen soziale Präferenzen}

Annahme: Aktionäre gewichten finanziellen und sozialen Nutzen

\begin{itemize}
\tightlist
\item
  Sozialer Nutzen gewichtet mit β (=Auswirkungen auf die Allgemeinheit/
  Wohlfahrt)
\item
  Finanzieller Nutzen gewichtet mit (1- β)
\end{itemize}

\textbf{Nutzenfunktion des Aktionärs}: U(x) = (1- β) *Profit + β *
Wohlfahrt

U(A) = (1- β) * Π(sauber) + β * Π(sauber) = Π(sauber)

U(B) = (1- β) * Π(dreckig) + β *\[ Π(dreckig) - d\] = Π(dreckig) -- β *
d

Entscheidung für saubere Technologie unter Erfüllung der folgenden
Bedingung:

Π(sauber) \textgreater= Π(dreckig) -- β * d

\textbf{Entscheidungsfaktoren:}

\begin{itemize}
\tightlist
\item
  Höhe des Profitverzichts
\item
  Höhen des Schadens d
\item
  Gewichtung sozialer Präferenzen durch Aktionäre
\end{itemize}

Präferenzbasierte CSR Maßnahme ist aus Aktionärssicht dann
durchzuführen, wenn der finanzielle Schaden der Maßnahme kleiner ist als
der präferenzgewichtete Schaden für die Allgemeinheit

\textbf{Rechenbeispiel} mit β = 0,5

\begin{longtable}[]{@{}llll@{}}
\toprule
\endhead
Technologie & Profit & Schaden & Wohlfahrt \\
A (sauber) & 100 & 0 & 100 \\
B (dreckig) & 120 & 30 & 90 \\
\bottomrule
\end{longtable}

\textbf{Aktionärsnutzen:}

U(A)= 0,5 * 100 + 0,5* 100 = 100

U(B)= 0,5 * 120 + 0,5 * 90 = 105

U(B) \textgreater{} U(A)

\textbf{Fallanwendung?}

\begin{itemize}
\tightlist
\item
  Berechne den Profit von CSR Maßnahme und der Alternative (negativer
  Profit möglich)
\item
  Berechne/ ermittle den Schaden oder Wohlfahrtsgewinn der Maßnahme
\item
  Antwort ob CSR Maßnahme durchgeführt werden sollte, hängt der der
  Gewichtung der Wohlfahrt aus Sicht der Aktionäre ab (Wert β), der Höhe
  des finanziellen Nachteils aus der CSR Maßnahme und der Höhe des
  Wohlfahrtschadens ab
\item
  Falls soziale Präferenzen der Aktionäre die Entscheidung nicht
  beeinflussen (β = 0), entscheiden sich die Aktionäre für die
  Alternative mit dem höheren Profit
\item
  Berechne kritischen Wert β ab dem die Aktionäre sich für die
  Durchführung der präferenzbasierten CSR Maßnahme entscheiden würden
\end{itemize}

\hypertarget{messbarkeit-pruxe4ferenzbasierter-csr}{%
\subsection{Messbarkeit präferenzbasierter
CSR}\label{messbarkeit-pruxe4ferenzbasierter-csr}}

Damit der Vorstand entscheiden kann ob eine unprofitable CSR Maßnahme
durchgeführt werden sollte, ist es notwendig die Parameter aus dem
Modell messbar zu machen

\begin{itemize}
\tightlist
\item
  Höhe des entstehenden finanziellen Nachteils: Π(sauber) - Π (dreckig)
\item
  Präferenzgewichtung der Shareholder: β
\item
  Höhe des negativen Gesamtwohlfahrtsschaden: d
\end{itemize}

Die \textbf{Höhe des entstandenen finanziellen Nachteils} ist zu
berechnen bzw. zu schätzen durch den Vergleich der beiden
Alternativszenarios auf den Unternehmensgewinn. Finanzberichtserstattung
als Informationsgrundlage

Der \textbf{Gesamtwohlfahrtsschaden} muss identifiziert und
quantifiziert werden. CSR Berichterstattung als Informationsgrundlage

\textbf{Soziale Präferenzen} der Aktionäre können durch Umfragen und
Abstimmungen der Hauptversammlung bestimmt werden

Die Schwierigkeit bei der Messung (Identifikation, Quantifizierung etc.)
der Parameter führen aus Sicht der Corporate Governance auch bei der
Umsetzung von präferenzbasierten CSR Maßnahmen zu Agency Problemen auf
Grundlage von Informationsasymmetrien und Interessenskonflikten

\begin{itemize}
\tightlist
\item
  \textbf{Geringe Anstrengungen} des Vorstands: Geringe Profitabilität
  des Unternehmens auf Grund von fehlendem Arbeitseinsatzes des
  Vorstands kann zur Vertuschung als präferenzbasieret CSR Maßnahme
  „getarnt'' werden
\item
  \textbf{Selbstbereicherung: }Vorstand führt CSR Maßnahmen aus
  Eigeninteresse durch und bewertet seine eigenen sozialen Präferenzen
  höher als die der Aktionäre (Vorstandsnutzen im Entscheidungskalkül
  anstelle des Aktionärsnutzen)
\item
  \textbf{Streben nach Prestige}: CSR Maßnahmen werden vom Vorstand aus
  Prestige Gründen durchgeführt und das Entscheidungskalkül der
  Aktionäre wird missachtet
\item
  E\textbf{ntrenchment}: CSR Maßnahme wird genutzt, um Verbündete zu
  gewinnen, die die Position des Vorstands stärken sollen (=Zementierung
  der eigenen Position)
\end{itemize}

\textbf{Fallanwendung?}

\begin{itemize}
\tightlist
\item
  Identifikation und Quantifizierung der relevanten Parameter für die
  Beurteilung ob CSR Maßnahme durchgeführt werden sollte
\item
  Sofern nicht alle drei Parameter bekannt sind, kann die
  Vorteilhaftigkeit der CSR Maßnahme nicht vollständig beurteilt werden
  (Einschätzung kann jedoch vorgenommen werden bei Begründung)
\item
  Hinweise auf die Gefahr von Agency Konflikten bei der Durchführung von
  CSR Maßnahmen ohne vollständige Information über Parameter (geringe
  Anstrengung, Selbstbereicherung, Streben nach Prestige und
  Entrenchment)
\end{itemize}

\hypertarget{soziale-pruxe4ferenzen-der-aktionuxe4re-aggregation-und-risiken}{%
\subsection{Soziale Präferenzen der Aktionäre -- Aggregation und
Risiken}\label{soziale-pruxe4ferenzen-der-aktionuxe4re-aggregation-und-risiken}}

Die individuell unterschiedlich ausgeprägten sozialen Präferenzen der
Shareholder müssen zur Beurteilung von präferenzbasierten CSR Maßnahmen
aggregiert werden. Eine Form der Aggregation erfolgt dabei durch die
Abstimmung über CSR Maßnahmen im Rahmen der Hauptversammlung

Beachte: Die Gewichtung der Aktionärsstimme wird dabei anhand des
Anteilsbesitzt der Aktionäre bestimmt

Das Ergebnis bildet die sozialen Präferenzen des Medianaktionärs ab

Ein zusätzlicher Vorteil der Miteinbeziehung der Aktionäre in die
Entscheidung über CSR Maßnahmen besteht darin, dass die Aktionäre durch
die Partizipation an der Entscheidung eine höhere Verantwortung für die
Auswirkungen auf die Umwelt fühlen könnten

Ein anderer Mechanismus der Beurteilung von CSR Maßnahmen besteht in der
\textbf{Selbstselektion} durch die Aktionäre. Unternehmen können nach
außen kommunizieren in welchem Umfang sie bereit sind CSR Maßnahmen
umzusetzen (z.B. Unternehmensleitbild, CSR Berichtserstattung oder der
Wahl der Unternehmensform), die Aktionäre können demnach nur in die
Unternehmen investieren, deren Unternehmenspolitik zu ihren sozialen
Präferenzen passt

\textbf{Risiken bei der Umsetzung von präferenzbasierten CSR Maßnahmen}

Gefahr des Amoral Drift: Unternehmen die präferenzbasierte CSR Maßnahmen
durchführen möchten, werden durch die Gefahr einer feindlichen Übernahme
letztendlich von der Durchführung abgehalten

Voraussetzung dafür sind heterogene soziale Präferenzen der Aktionäre

Ein Aktionär ohne soziale Präferenzen hat im Fall der Nichtdurchführung
der CSR Maßnahme einen Aktionärsnutzen in Höhe von Π(dreckig) der größer
ist als Π(sauber)

Falls der Aktionär ohne soziale Präferenzen über ausreichende
finanzielle Mittel verfügt, kann er einen Gewinn erzielen indem er ein
Übernahmeangebot i.H.v. von Preis P abgibt solange der preis P kleiner
als Π(dreckig) ist

Aktionäre mit sozialen Präferenzen könnten bereit sein dieses Angebot
anzunehmen, wenn der Preis p \textgreater{} Π(sauber) ist und der
zusätzliche finanzielle Gewinn den Wohlfahrtsverlust ausgleicht

Gegenmaßnahmen

\begin{itemize}
\tightlist
\item
  Schutzmechanismen vor feindlichen Übernahmen
\item
  Ankeraktionäre (z.B. Stiftungen) deren soziale Präferenzen starken
  Einfluss ausüben
\end{itemize}

\textbf{Fallanwendung?}

\begin{itemize}
\tightlist
\item
  Sind soziale Präferenzen der Aktionäre bekannt? Falls nicht besteht
  die Möglichkeit die Entscheidung über CSR Maßnahme auf
  Hauptversammlung zur Abstimmung zu bringen
\item
  Alternativ kann Unternehmen auch die CSR Politik klar nach außen
  kommunizieren, um eine Selbstselektion der Aktionäre anhand ihrer
  sozialen Präferenzen zu erreichen
\item
  Gefahr der feindlichen Übernahme: Aktienkurs kann durch Stoppen der
  CSR Maßnahme um Betrag x erhöht werden (Voraussetzung CSR Maßnahme
  negativ eingepreist durch Kapitalmarkt)
\item
  Aktionär ohne soziale Präferenzen kann ein Übernahmeangebot zum
  aktuellen Kurs mit einer zusätzlichen Prämie abgeben (Prämie
  \textless{} = x), durch das Stoppen der CSR Maßnahme steigt der
  Unternehmenswert und er macht einen Gewinn in Höhe von (x -- Prämie)
\end{itemize}

\textbf{Soziale Präferenzen der Aktionäre am Beispiel von „Socially
Responsible Investments'' (=SRI)}

Aktuell sind etwa 25\% der investierten Fondsvermögens in SRI Fonds
angelegt

ESG Kriterien dienen häufig als Einflussfaktor bei der Aktienauswahl für
SRI Fonds

\begin{itemize}
\tightlist
\item
  Unternehmen werden nach ESG Score ausgewählt
\item
  Blacklisting von Unternehmen bestimmter Branchen (Rüstung, Öl)
\item
  Whitlisting von Unternehmen mit guten ESG Scores
\end{itemize}

SRI Fonds treten als Aktionäre bei Hauptversammlungsabstimmung auf und
können dadurch die sozialen Präferenzen des Medianwählers signifikant
erhöhen

\begin{itemize}
\tightlist
\item
  β des Medianwählers steigt durch SRI Einfluss auf Hauptversammlungen
\item
  SRI Fonds kann Vorstand im Dialog zur Umsetzung von CSR Maßnahmen
  auffordern (falls Einfluss hoch genug ist)
\item
  Wahrscheinlichkeit für die Umsetzung von präferenzbasierten CSR
  Maßnahmen erhöht sich
\end{itemize}

Jedoch bei Aktienauswahl anhand von ESG Scores haben die SRI einen
geringen Einfluss auf Unternehmen mit niedrigen ESG Scores, da diese
untergewichtet oder aussortiert werden

\begin{itemize}
\tightlist
\item
  Einfluss stark bei „guten'' Unternehmen und sehr gering bei
  „schlechten''
\item
  Es entsteht ein Anreiz für Unternehmen nicht in eine Blacklist
  aufgenommen zu werden und keine unterdurchschnittlichen ESG Scores zu
  erzielen, da dadurch der Zufluss von signifikantem Anteil an
  Börsenkapital ausbleiben könnte
\end{itemize}

\textbf{Fallanwendung?}

\begin{itemize}
\tightlist
\item
  Werden Aktien des Unternehmens von SRI Fonds gehalten? Falls ja steigt
  die durchschnittliche soziale Präferenz des Medianwähler und eine
  Durchführung von präferenzbasierten CSR könnte begünstigt werden
\item
  SRI Fonds kann Unternehmen außerdem direkt kontaktieren, um CSR
  Maßnahmen zu fördern
\item
  Anreiz an Vorstand ESG Score zu behalten oder nicht in eine Blacklist
  aufgenommen zu werden, um Finanzierung durch SRI Fonds nicht zu
  gefährden
\end{itemize}

\textbf{Rechenbeispiel}

Aktionäre: A, B und C halten jeweils 33,33\% an Unternehmen

β(A)=0; β(B)=0,6; β(C)=0,7

Technologie A bringt Profit 200 und Schaden 0

Technologie B bringt Profit 250 und Schaden 100

Nutzen für jeden Aktionär berechnen (jeweils für beide Alternativen)

2 Stimmen für A (Aktionäre B und C) gegen 1 Stimme für B (Aktionär A)

\hypertarget{csr-berichterstattung}{%
\section{CSR Berichterstattung}\label{csr-berichterstattung}}

Die CSR Berichterstattung erfolgt in Form der \textbf{Nichtfinanziellen
Erklärung} sowie der Erklärung der \textbf{Nichtfinanziellen
Leistungsindikatoren}

\hypertarget{nichtfinanzielle-erkluxe4rung}{%
\subsection{Nichtfinanzielle
Erklärung}\label{nichtfinanzielle-erkluxe4rung}}

\textbf{Abgabepflicht}: Große kapitalmarktorientierte Unternehmen,
Kreditunternehmen und Versicherungsunternehmen mit mehr als 500
Arbeitnehmern

Die Abgabe der Nichtfinanziellen Erklärung kann wahlweise als
\textbf{Bestandteil des Lageberichts} oder einzeln als \textbf{separater
Bericht} veröffentlicht werden

\textbf{Prüfungspflicht}

\begin{itemize}
\tightlist
\item
  Keine Prüfungspflicht im Rahmen der Abschlussprüfung
\item
  Prüfungspflichtig durch den Aufsichtsrat (dieser kann optional einen
  Wirtschaftsprüfer beauftragen)
\end{itemize}

\textbf{Inhalt: }Neben der Beschreibung des Geschäftsmodells muss das
Unternehmen auf mindestens 5 nichtfinanzielle Aspekte eingehen

\begin{enumerate}
\def\labelenumi{\arabic{enumi}.}
\tightlist
\item
  Umweltbelange (z.B. Emissionen, sparsamer Umgang mit Ressourcen,
  Anteil erneuerbarer Energien)
\item
  Arbeitnehmerbelange (z.B. Arbeitsbedingungen, Gleichstellung)
\item
  Sozialbelange (z.B. Dialog mit Kommunen und lokalen Gruppierungen)
\item
  Achtung der Menschenrechte (z.B. Überprüfung von Arbeitsbedingungen in
  Lieferantenkette)
\item
  Bekämpfung von Korruption und Bestechung (z.B. Kontrollmechanismen und
  Maßnahmen)
\end{enumerate}

Zu allen 5 Aspekten hat das Unternehmen zu erklären:

\begin{itemize}
\tightlist
\item
  welches \textbf{Konzept} verfolgt wurde und welche \textbf{Ergebnisse}
  (=CSR Aktivität) damit erzielt wurden,
\item
  \textbf{Risiken} mit potential schwerwiegenden Folgen, die \textbf{aus
  Geschäftstätigkeit und den Geschäftsbeziehungen} (insb. Lieferkette)
  entstehen könnten,
\item
  Darstellung bedeutsamer \textbf{Leistungsindikatoren} und
  \textbf{Zusammenhänge zum Jahresabschluss}
\end{itemize}

Die Inhalte der Nichtfinanziellen Erklärung sind rechtlich nur vage
präzisiert, es existieren jedoch unterschiedliche nationale und
internationale Rahmenwerke die Unternehmen als Leitfaden dienen können
(z.B. GRI -- Standards für Nachhaltigkeitsberichtserstattung)

Der Berichtsumfang ergibt sich primär an dem Grundsatz der
\textbf{Wesentlichkeit} und der \textbf{Relevanz für Stakeholder}

\hypertarget{nichtfinanzielle-leistungsindikatoren}{%
\subsection{Nichtfinanzielle
Leistungsindikatoren}\label{nichtfinanzielle-leistungsindikatoren}}

\textbf{Abgabepflicht}: Verpflichtender Bestandteil des Lageberichts
\textbf{für große Kapitalgesellschaften}

\textbf{Prüfungspflicht}

\begin{itemize}
\tightlist
\item
  Durch Abschlussprüfer verpflichtend im Rahmen der Lageberichtsprüfung
\item
  Ebenfalls vom Aufsichtsrat verpflichtend zu prüfen
\end{itemize}

\textbf{Fallanwendung?}

\begin{itemize}
\tightlist
\item
  Prüfung ob Unternehmen \textbf{mehr als 500 Arbeitnehmer} hat,
  \textbf{börsennotiert} ist und/oder ein \textbf{Versicherungs- oder
  Kreditinstitut} ist. Falls ja besteht die Pflicht zur Abgabe einer
  \textbf{nichtfinanziellen Erklärung} (Bestandteil des Lageberichts
  oder separat)
\item
  Falls das Unternehmen eine \textbf{große Kapitalgesellschaft} ist,
  muss im Rahmen des Lageberichts ebenfalls über \textbf{bedeutsame
  nichtfinanzielle Leistungsindikatoren} berichtet werden
\item
  In der nichtfinanziellen Erklärung muss das Unternehmen mindestens die
  5 gesetzlich vorgeschriebenen Aspekte behandeln (Umwelt, AN, soziales,
  Menschenrechte, Bestechung und Korruption). Jeder Aspekt muss im
  Hinblick auf \textbf{Konzepte und Ergebnisse}, \textbf{Risiken die aus
  Geschäftstätigkeit und Unternehmenskontakten, }sowie der einer
  Darstellung von** Leistungsindikatoren und den Verbindungen zum
  Jahresabschluss **erörtert werden
\item
  In Abhängigkeit von der Unternehmenstätigkeit müssen wesentliche
  Aspekte und ihre Auswirkungen auf Stakeholder eventuell besonders
  intensiv beleuchtet werden (Automobilhersteller hat z.B. auf die
  Umweltfreundlichkeit des Produkts einzugehen und muss Risiken der
  Lieferkette erörtern)
\item
  Zur Konkretisierung der vagen gesetzlichen Vorgaben kann das
  Unternehmen auf nationale oder internationale Richtlinien zur
  Nachhaltigkeitsberichtserstattung zurückgreifen
\end{itemize}

\hypertarget{kosten-und-nutzen-einer-freiwilligen-csr-berichtserstattung}{%
\section{Kosten und Nutzen einer freiwilligen CSR
Berichtserstattung}\label{kosten-und-nutzen-einer-freiwilligen-csr-berichtserstattung}}

CSR Berichterstattung ist nur für einen kleinen Teil der deutschen
Unternehmen im Rahmen der Nichtfinanziellen Erklärung verpflichtend
(\textgreater500 AN + Börsennotierung und/oder Versicherungs- oder
Kreditinstitut)

Sollten nicht verpflichtete Unternehmen einen freiwilligen CSR Report
erstellen?

Lohnt es sich für verpflichtete Unternehmen freiwillig umfangreichere
CSR Reporte zu veröffentlichen?

\textbf{Kosten der CSR Berichtserstattung}

\begin{itemize}
\tightlist
\item
  Umfangreicher Aufwand durch Informationssammlung, Auswertung und
  Zusammenführung in verschiedensten Bereichen
\item
  Kosten der Prüfung falls Glaubwürdigkeit durch freiwillige Prüfung
  erhöht werden soll
\item
  Indirekte Kosten durch das Preisgeben von Betriebsgeheimnissen und
  sensiblen Unternehmensdaten (Ausnutzung durch Konkurrenz, Kunden oder
  Lieferanten möglich)
\end{itemize}

\textbf{Nutzen der freiwilligen CSR Berichtserstattung}

\textbf{Compliance Nutzen}

\begin{itemize}
\tightlist
\item
  Compliance Aufgabe im Rahmen der CSR wird durch ausführliches
  Auseinandersetzten mit potenziellen Risiken wahrscheinlich besser
  erfüllt Risikomanagement
\item
  Vertrauen bei Stakeholdern und Shareholdern (und wohlmöglich des
  Gesetzgebers) erhöht sich durch besseres Risikomanagement und
  Transparenz
\end{itemize}

\textbf{Strategische CSR}

\begin{itemize}
\tightlist
\item
  Strategische CSR Maßnahmen können gezielt durchgeführt werden, um die
  Reputation des Unternehmens bei \textbf{Stakeholdern} zu verbessern
  (als Ergänzung zu Werbekampagnen z.B.)
\item
  \textbf{Shareholder} erhalten umfangreiche und glaubwürdige
  Informationen über die Umsetzung und den Erfolg von strategischen CSR
  Maßnahmen
\end{itemize}

\textbf{Präferenzbasierte CSR}

\begin{itemize}
\tightlist
\item
  Präferenzbasierte CSR ist ohne glaubwürdige und vertrauenswürdige
  Informationen über Nutzen, Wohlfahrtsschaden und Kosten kaum
  durchführbar. Ohne umfangreiche Berichtserstattung drohen Agency
  Konflikte
\item
  Shareholder mit sozialen Präferenzen können zusätzlich angesprochen
  werden, um präferenzbasierte CSR Maßnahmen zu unterstützen
\end{itemize}

\textbf{Für welche Unternehmen überwiegt der Nutzen einer freiwilligen
bzw. einer umfangreicheren CSR Berichtserstattung die Kosten?}

\begin{enumerate}
\def\labelenumi{\arabic{enumi}.}
\item
  \textbf{Unternehmensgröße: }Die absoluten Kosten der
  Berichtserstellung fallen für große Unternehmen relativ weniger ins
  Gewicht als für kleinere Unternehmen. Große Unternehmen profitieren
  von Skaleneffekten und sind eher bereit die zusätzlichen Kosten zu
  tragen
\item
  \textbf{Anteilseignerstruktur:} Je höher der Anteil an Shareholdern
  mit hohen sozialen Präferenzen (z.B. hoher Anteil in SRI Fonds oder
  sozialen Stiftungen) desto stärker profitiert Unternehmen
  voraussichtlich von (umfangreicherer) Berichtserstattung
\item
  \textbf{Geschäftsmodell:}

  \begin{enumerate}
  \def\labelenumii{\alph{enumii}.}
  \tightlist
  \item
    Strategische CSR: Hohe Bedeutung für Unternehmen mit engen
    Kundebeziehungen und großem Einfluss des Images (z.B.
    Konsumgüterbranche)
  \item
    CSR als Compliance-Aufgabe: Unternehmen aus Branchen mit negativen
    Auswirkungen auf die Allgemeinheit und schlechtem Image (z.B.
    Energiebranche, Rüstung) können signalisieren, dass relevante
    Risiken und ihre Auswirkungen minimiert und adressiert wurden
  \end{enumerate}
\item
  Unternehmen mit hoher/vorbildlicher Aktivität in der CSR Umsetzung
  profitiert stärker als Unternehmen mit geringeren Aktivitäten
\end{enumerate}

\textbf{Fallanwendung?}

\begin{itemize}
\tightlist
\item
  Prüfe ob Unternehmen einen erhöhten Nutzen durch die freiwillige
  und/oder umfangreichere Berichterstattung über CSR Maßnahmen hat
\item
  Kann Unternehmen strategische CSR Maßnahmen durch die
  Berichterstattung nutzen, um seine Reputation zu verbessern?
\item
  Ist Unternehmen in einer Branche tätig, die allgemein ein schlechtes
  Image durch ihre Auswirkungen auf die Allgemeinheit hat?
\item
  Kann Unternehmen die Compliance Aufgabe durch ein verbesserten
  Risikomanagement besser umsetzen?
\item
  Besteht ein hoher Anteil von Shareholdern mit ausgeprägten sozialen
  Präferenzen?
\item
  Kann das Unternehmen aufgrund seine Größe Skaleneffekte ausnutzen?
\end{itemize}

\hypertarget{effekte-einer-verpflichtenden-csr-berichtserstattung}{%
\section{Effekte einer verpflichtenden CSR
Berichtserstattung}\label{effekte-einer-verpflichtenden-csr-berichtserstattung}}

Aktuell betrifft die Pflicht Unternehmen mit mehr als 500 AN und einer
Börsennotierung/Kredit- oder Versicherungsunternehmen

Sollte die willkürliche Begrenzung erweitert werden?

\begin{longtable}[]{@{}
  >{\raggedright\arraybackslash}p{(\columnwidth - 2\tabcolsep) * \real{0.49}}
  >{\raggedright\arraybackslash}p{(\columnwidth - 2\tabcolsep) * \real{0.49}}@{}}
\toprule
\endhead
Vorteile einer erweiterten CSR Berichtserstattung & Nachteile einer
erweiterten CSR Berichtserstattung \\
& Keine vollständige Harmonisierung \\
\bottomrule
\end{longtable}

\textbf{Fallanwendung?}

\begin{itemize}
\tightlist
\item
  Prüfe ob Unternehmen gesetzlich verpflichtet ist CSR Reporting zu
  betreiben?
\item
  Falls nicht muss geprüft werden, ob Anreize zum freiwilligem CSR
  Reporting bestehen
\item
  Prüfe ob Unternehmen ein „Negativbeispiel'' für unzureichende CSR
  Aktivitäten ist, falls ja würde eine erweiterte Verpflichtung Anreize
  dazu setzten die CSR Anstrengungen zu erhöhen
\end{itemize}

\hypertarget{uxfcberwachungsfunktion-des-aufsichtsrats}{%
\section{Überwachungsfunktion des
Aufsichtsrats}\label{uxfcberwachungsfunktion-des-aufsichtsrats}}

AR wird von den Aktionären gewählt (=Agent der Aktionäre)

AR hat im Interesse der Aktionäre zu handeln Versprechen der
Aufgabenerfüllung gegenüber den Shareholdern

\textbf{Der AR hat die Geschäftsführung zu überwachen (§111 I AktG)}

Genauer: Überwachung der \textbf{strategischen und operativen
Unternehmensführung} durch den \textbf{Vorstand} auf
\textbf{Rechtmäßigkeit, Ordnungsmäßigkeit und Wirtschaftlichkeit}

\textbf{3 Dimensionen der Überwachungsfunktion des AR}

\textbf{Rechtmäßigkeit}: Beachtet der Vorstand gesetzliche Vorschriften?

\begin{itemize}
\tightlist
\item
  Betrifft unmittelbare Gesetze (z.B. AktG), Satzung der
  Geschäftsordnung, aber auch Gesetze aus den Bereichen Arbeitsrecht,
  Steuerrecht, Umweltschutzrecht etc.
\end{itemize}

\textbf{Ordnungsmäßigkeit}: Werden die Arbeitsabläufe im Unternehmen
durch den Vorstand ordnungsmäßig organisiert und überwacht?

\begin{itemize}
\tightlist
\item
  Unternehmensplanung unter Festlegung und Berücksichtigung von
  Unternehmenszielen sowie Implementation von Kontrollmechanismen
  (Compliance System, RMS, RFS, interne Revisionen)
\end{itemize}

\textbf{Wirtschaftlichkeit: }Ist die Liquidität des Unternehmens
sichergestellt? Werden Ressourcen effizient eingesetzt, um einen
nachhaltigen und langfristigen Ertrag zu erwirtschaften?

Die \textbf{Reichweite} der Überwachung des Vorstands durch den AR
ergibt sich primär aus den Berichterstattungspflichten des Vorstands
gegenüber dem AR (§90 I AktG). Der Vorstand berichtet dem AR zwingend
über:

\begin{itemize}
\tightlist
\item
  Unternehmenslage: Geschäfte von wesentlicher Bedeutung und allgemeine
  wirtschaftliche Lage, Umsatz und Rentabilität, Geschäftspolitik,
  Planabweichungen
\item
  Unternehmensrisiken: Risikolage, RM und Compliance
\item
  Unternehmensplanung; Finanz- Investitions-, und Personalplanung,
  Geschäftspolitik
\end{itemize}

AR hat zusätzlich die Pflicht zur Gewährleistung seiner
Überwachungsfunktion die Berichterstattung des Vorstands spezifischer zu
definieren und zusätzliche Informationen einzuholen

\textbf{Aufgaben des AR}

\begin{itemize}
\tightlist
\item
  Ex Ante Überwachung: Überwachung von zukünftigen Aktivitäten des
  Vorstands
\item
  Ex Post Überwachung: Beurteilung von abgeschlossenen Aktivitäten des
  Vorstands
\end{itemize}

Beispiele der Ex Ante Überwachung (zukunftsgerichtete Überwachung mit
gewisser Beratungsfunktion)

\begin{itemize}
\tightlist
\item
  Festlegung der Vorstandsvergütung durch den AR bevor dieser
  Vertragsbestandteil wird
\item
  Bestellung neuer Vorstandsmitglieder
\item
  Zustimmungsvorbehalte für Geschäfte grundlegender und bedeutender Art
\end{itemize}

Beispiele der Ex Post Überwachung

\begin{itemize}
\tightlist
\item
  Prüfung des Jahresabschlusses
\item
  Abberufung von Vorstandsmitgliedern
\end{itemize}

\textbf{Fallanwendung?}

\begin{itemize}
\tightlist
\item
  Prüfe ob Ereignis unter die Berichterstattungspflichten des Vorstands
  gegenüber des AR fällt (Unternehmenslage, Unternehmensrisiken und
  Unternehmensplanung)
\item
  Falls ja hat AR die Wirtschaftlichkeit, Rechtmäßigkeit und
  Ordnungsmäßigkeit zu prüfen. Bei Bedarf muss der AR zusätzliche
  Informationen vom Vorstand einfordern
\item
  (Handelt es sich um Ex post oder ex ante Überwachung?)
\end{itemize}

\textbf{Beratungsfunktion des AR?}

AktG: Überwachung der Geschäftsführung durch AR

DCGK: Der AR \textbf{überwacht und berät} den Vorstand bei der Leitung
des Unternehmens (6 DCGK)

AR soll den Vorstand neben der Kontrolle auch im Sinne einer
begleitenden Kontrolle beraten

\begin{itemize}
\tightlist
\item
  Ex Ante Überwachung -- zukunftsgerichtete Überwachung der
  Unternehmensleitung hat Beratungscharakter
\item
  Zustimmungsvorbehalte des AR haben ebenfalls eine Beratungsfunktion
  gegenüber dem Vorstand
\end{itemize}

Der Vorstand kann AR bereits vor dem Treffen einer wesentlichen
Entscheidung konsultieren, um Vorbehalte frühzeitig zu identifizieren.
\textbf{Der Umfang der Beratung liegt jedoch im Ermessensspielraum des
AR}

Der AR kann je nach Grad der Beratung und Einbindung sowohl als
„Mitspieler'' des Vorstands als auch als „Schiedsrichter'' agieren

\textbf{Mitspieler -- Hoher Grad der (zukunftsgerichteten) Beratung des
Vorstands durch AR }

\begin{itemize}
\tightlist
\item
  Ex Ante: Hoher Grad der Einbindung/ Beratung in die Geschäftstätigkeit
  (z.B. in strategische und operative Angelegenheiten) Umfangreicher
  Einfluss des AR \textbf{vor} Vorstandshandlungen
\item
  Ex Post: Wenn AR ex ante bereits stark involviert ist, ist das
  Ergebnis der Vorstandsleistung bereits stark durch AR beeinflusst
  (keine klare Trennung von Vorstands- und AR Leistung mehr möglich).
  Der AR wird hier wenig intervenieren da er sich quasi selbst überprüft
\end{itemize}

Unabhängigkeit des AR sinkt!

Die Rolle des stark eingebundenen AR ist aus der Sicht der Prinzipal
Agenten Theorie sehr kritisch zu sehen! Ex Post Überwachungsfunktion des
AR wird durch Interessenskonflikte behindert

\textbf{Schiedsrichter }

\begin{itemize}
\tightlist
\item
  Ex Ante: Geringe Einmischung/ Beratung in die Geschäftspolitik und
  wenig Interventionen
\item
  Ex Post: Ergebnis basiert allein auf Vorstandsleistung und der AR ist
  unabhängig davon. Umfangreiche Interventionen und Prüfungen sind
  deutlich wahrscheinlicher als beim Mitspieler
\end{itemize}

Unabhängigkeit des AR ist höher als beim „Mitspieler''

** Ein hoher Grad der Beratung und Einbeziehung des AR durch den
Vorstand bevor eine Entscheidung getroffen wird, wirkt sich negativ auf
die Unabhängigkeit des ARs ex post der Entscheidung aus! **

\hypertarget{struktur-des-aufsichtsrats}{%
\section{Struktur des Aufsichtsrats}\label{struktur-des-aufsichtsrats}}

Vorgaben zum Aufbau des AR nach AktG

Größe: Minimum 3 bis maximum 21 Mitglieder des AR sind zulässig

AR Mitglieder werden als Vertreter der Aktionäre von den Aktionären
\textbf{über die Hauptversammlung gewählt. }Der AR schlägt neue
Mitgliedskandidaten auf der HV vor\textbf{. }Nach \textbf{DCGK }wird
eine\textbf{ Einzelwahl der Mitglieder empfohlen}

Nach Wahl durch die HV darf ein AR Mitglied \textbf{eine maximale
Amtsperiode von 5 Jahren antreten}, die \textbf{Wiederbestellung ist
grundsätzlich möglic}h nach erneuter Wahl durch die HV

Die vorzeitige Abberufung eines AR Mitglieds kann nur mit einer 75\%
Mehrheit der Hauptversammlung beschlossen werden

Nach der Bestellung der AR Mitglieder durch die HV wählt der AR unter
sich selbst den Aufsichtsratsvorsitzenden und seinen Stellvertreter

\textbf{Aufgabe des AR Vorsitzenden}

\begin{itemize}
\tightlist
\item
  Vorbereitung und Leitung der Sitzungen des AR
\item
  DCGK Empfehlung: Austausch mit Vorstand und Investoren
\end{itemize}

\textbf{Drittbeteiligungsgesetz}

Gültig für Unternehmen mit 500+ Arbeitnehmern (inklusive beherrschter AN
und exklusive leitender Angestellter) unter der Voraussetzung, dass das
Mitbestimmungsgesetz nicht angewandt wird

Bedeutung: AR muss zu 1/3 mit Arbeitnehmervertretern besetzt werden
(Anzahl der AR Mitglieder muss durch 3 teilbar sein in diesem Fall). Das
Geschlechterverhältnis der AN-Vertreter im AR soll dem des
Gesamtunternehmens entsprechen

\textbf{Mitbestimmungsgesetz}

Gültig für Unternehmen mit 2000+ Arbeitnehmern (hier inklusive leitender
Angestellter und beherrschter AN)

Bedeutung

\begin{itemize}
\tightlist
\item
  AR muss sich zur Hälfte aus AN-Vertretern und zur anderen Hälfte aus
  Anteilseignern zusammensetzen (sog. Parität)
\item
  Aktionärsvertreter dürfen jedoch den Aufsichtsratsvorsitzenden
  bestimmen, der bei stimmgleichen Abstimmungen eine doppelte Stimme
  besitzt (=Aktionärsvertreter haben höheren Einfluss)
\item
  Falls das Unternehmen börsennotiert ist, sind die gesetzlichen
  Geschlechterquoten bei der Besetzung des AR einzuhalten (mindestens
  30\% Frauen und 30\% Männer)
\end{itemize}

Unternehmen die dem MitbestG unterliegen haben genaue Vorgaben über die
Anzahl der AR Mitglieder:

\begin{itemize}
\tightlist
\item
  2.000-10.000 AN 12 AR Mitglieder
\item
  10.001-20.000 AN 16 AR Mitglieder
\item
  20.001 + AN 20 AR Mitglieder
\end{itemize}

\textbf{Fallanwendung}?

\begin{itemize}
\tightlist
\item
  Prüfe ob Unternehmen mehr als 2.000 AN hat? Mitbestimmungsgesetz!
\item
  Prüfe ob Unternehmen mehr als 500 AN hat und nicht unter das MitbestG
  fällt? Drittbeteiligungsgesetz!
\item
  Prüfe für Unternehmen, die dem MitbestG unterliegen die Anzahl der AN
  und der resultierenden Anzahl an AR Mitgliedern (50\% AN-Vertreter und
  50\% Aktionärsvertreter, Aktionärsvertreter wählen den
  AR-Vorsitzenden)
\item
  Falls das Unternehmen nach MitbestG börsennotiert ist, muss zusätzlich
  die gesetzliche Geschlechterquote bei der Besetzung des AR eingehalten
  werden (min 30\% pro Geschlecht)
\end{itemize}

\textbf{Effekt der Aufsichtsratsgröße}

Tendenz von Unternehmen die AR zu verkleinern, große AR haben zwei
primäre Nachteile

\begin{enumerate}
\def\labelenumi{\arabic{enumi}.}
\tightlist
\item
  Koordinationsprobleme: Kommunikation und Zielabstimmung werden mit
  zunehmender Zahl an Mitgliedern überproportional erschwert
\item
  Free Rider Problem: Die Kosten der Anstrengung (z.B. einer notwendigen
  Kritik) trägt ein einzelnes AR Mitglied eigenständig, die Vorteile
  dieser Anstrengung trägt jedoch der ganze AR bzw. das Unternehmen.
  Einzelnes AR Mitglied hat also Anreiz nicht tätig zu werden da der
  Nutzen größer ist, wenn er auf die Aktion eines anderen Mitglieds
  wartet
\end{enumerate}

\textbf{Lösungen zur Problematik großer Aufsichtsräte?}

Koordinations- und Free Rider Probleme können durch die Zuweisung
bestimmter Aufgabengebiete an kleinere Ausschüsse reduziert werden

DCGK Empfehlung: Bildung von Ausschüssen in Abhängigkeit der Größe des
AR und den spezifischen Gegebenheiten des Unternehmens

Gesetzliche Vorgaben des AktG zu Ausschüssen

\begin{itemize}
\tightlist
\item
  Einrichtung prinzipiell möglich zur Vorbereitung und Überwachung unter
  der Auflage das regelmäßig an den restlichen AR berichtet wird und die
  letztendliche Entscheidung in Form der Beschlussfassung bei wichtigen
  Aufgaben ebenfalls durch den gesamten AR erfolgt
\end{itemize}

Der gesamte AR trägt Letztverantwortung trotz Ausschussbildung

\textbf{Ausnahme}: P\textbf{rüfungsausschuss des AR} für die Überwachung
von IKS, Rechnungslegung ist durch \textbf{AktG vorgeschrieben} (DCGK
Empfehlung: ebenfalls Überwachung der Compliance Aufgabe)

DCGK Empfehlung: Neben Prüfungsausschuss soll ebenfalls ein
Normierungsausschuss gebildet werden. Der Normierungsausschuss ist für
die Vorbereitung der Wahl und des Vorschlagens neuer Mitglieder für den
AR verantwortlich

\textbf{Weitere mögliche AR Ausschüsse}

Personalausschuss zur \textbf{Vorbereitung} der Auswahl des Vorstands,
Vergütungsausschuss zur \textbf{Vorbereitung} der Festlegung der
Vergütungsansprüche, Strategieausschüsse, Präsidialausschüsse
(Vorbereitung von Sitzungen), Vermittlungsausschuss etc.

\textbf{Fallanwendung}?

\begin{itemize}
\tightlist
\item
  Prüfe ob AR groß ist (ca. ab 16 Mitgliedern) und ob den Problematiken
  großer AR durch das Bilden von Ausschüssen entgegengewirkt werden kann
\item
  Empfehlenswert nach DCGK ist zumindest die Bildung eines Normierung-
  und Prüfungsausschusses
\item
  Je nach Unternehmenslage könnten weitere Ausschüsse temporär oder
  dauerhaft eingerichtet werden (z.B. temporären Personalausschuss bei
  einer anstehenden Neubesetzung des Vorstands)
\end{itemize}

\hypertarget{mitbestimmung-anteilseigner-vertreter-im-ar}{%
\section{Mitbestimmung -- Anteilseigner-Vertreter im
AR}\label{mitbestimmung-anteilseigner-vertreter-im-ar}}

Prinzipiell sollten alle Stakeholder (+ Shareholder) die durch das
Vorstandshandeln negativ betroffen sein könnten die Möglichkeit besitzen
einen gewissen Einfluss auf die Geschäftsführung zu nehmen

Bedeutende Stakeholder: Vorstandshandlungen die Sie betreffen (=negative
Externalitäten)

\begin{itemize}
\tightlist
\item
  Arbeitnehmer: Arbeitsplatzabbau
\item
  Gesellschaft: Umweltschädigungen
\end{itemize}

Frage: Sollte diese Einflussnahme über Eingriffsrechte in Form einer
Beteiligung am AR erfolgen oder ist ein gesetzlicher Schutz bzw.
vertragliche Reglungen vorzuziehen?

Die Absicherung der Ansprüche aller Stakeholder ist nur schwer möglich
durch eine AR Beteiligung, die Gewährung gesetzlicher und vertraglicher
Ansprüche ist außerdem i.d.R. effizienter

\textbf{Shareholder}: Anspruch auf Residualgewinn (=Unternehmensumsatz-
sämtliche Aufwendungen)

Shareholder müssen als Träger des unternehmerischen Risikos auf eine
vollständige Absicherung verzichten

Shareholder werden nachrangig bedient und besitzen im Gegensatz zu
anderen Stakeholdern (z.B. AN, Lieferanten) geringere Möglichkeiten sich
abzusichern

Shareholder-Vertreter im AR könnte durch die schlechteren
Absicherungsmechanismen gegenüber sonstigen Stakeholdern gerechtfertigt
werden

Eine Gewährleistung der Profitabilität des Unternehmens durch die
Kontrolle des Vorstands mithilfe von Shareholder -Vertretern im AR
erfüllt ebenfalls einen positiven Nebeneffekt für sonstige Stakeholder
des Unternehmens, die Fähigkeit der Erfüllung von anderen gesetzlichen
und vertraglichen Verpflichtungen wird ebenfalls abgesichert

Vorteile einer reinen Besetzung des AR mit Shareholder-Vertretern?

\begin{itemize}
\tightlist
\item
  Weitgehendend homogene Interessen (Profitabilität)
\item
  Geringere Gefahr von Koordinationsproblemen
\end{itemize}

\textbf{Fallanwendung}?

\begin{itemize}
\tightlist
\item
  Ermittle alle Stakeholder Gruppen, die durch Vorstandshandlungen
  negativ beeinflusst werden könnten (Gesellschaft bei Umweltschaden, AN
  bei Stellenabbau, Aktionäre bei Gewinneinbußen)
\item
  Ermittle rechtliche und vertragliche Absicherungen der
  Stakeholdergruppen (exklusive Shareholder): Arbeitsschutzgesetzte für
  AN; Umweltschutzgesetze für Gesellschaft
\item
  Shareholder können sich gegen das Risiko einer Minderung ihres Returns
  nicht vertraglich absichern, da sie das unternehmerische Risiko
  tragen. Eine Absicherung und Interessensvertretung der Aktionäre durch
  Vertreter im AR kann daher eine wichtige Funktion in der
  Sicherstellung der Profitabilität des Unternehmens darstellen
  (ebenfalls wichtig für sonstige Stakeholder)
\end{itemize}

\hypertarget{mitbestimmung--arbeitnehmervertreter-im-ar}{%
\section{Mitbestimmung -Arbeitnehmervertreter im
AR}\label{mitbestimmung--arbeitnehmervertreter-im-ar}}

Prinzipiell sind die Arbeitnehmer bereits durch vertragliche und
gesetzliche Reglungen gegenüber negativen Konsequenzen des
Vorstandshandelns geschützt (im Gegensatz zu Shareholdern)

Reicht der gesetzliche und vertragliche Schutz jedoch wirklich aus, um
die AN ausreichend zu schützen?

Hauptproblem: \textbf{Hold Up Probleme bei der Tätigung von
firmenspezifischen Investitionen} durch Arbeitnehmer

Hold Up: Eine Partei im Vertragsverhältnis leistet eine irreversible
Vorleistung, die von der anderen Vertragspartei opportunistisch
ausgenutzt wird

Beispiel: AN bildet sich in firmenspezifischem Bereich freiwillig vor
und erhofft daraus eine Beförderung/Gehaltserhöhung durch den
Arbeitgeber

\begin{itemize}
\tightlist
\item
  Die firmenspezifische Investition ist weder vertraglich noch
  gesetzlich abgesichert
\item
  Firmenspezifische Leistung ist nicht liquidierbar und kann nicht an
  anderen Arbeitgeber vermarktet/transferiert werden
\end{itemize}

Arbeitgeber kann die nachteilige Lage des Arbeitnehmers ausnutzen, er
belohnt die zusätzliche Investition nicht und profitiert trotzdem
opportunistisch von ihr

Arbeitnehmer antizipiert dieses Verhalten des Arbeitgebers und vermeidet
die firmenspezifische Investition ohne Absicherungsmechanismus

Lösungsmöglichkeit des Hold Up Problems: Arbeitnehmer erhalten
Eingriffsrechte im AR, um opportunistisches Verhalten der
Geschäftsführung zu vermeiden

\textbf{Weitere Vorteile der AN-Vertretung im AR}

\begin{itemize}
\tightlist
\item
  Hohes firmenspezifisches Wissen und Kenntnisse der internen Abläufe
  ermöglichen effektivere \textbf{Überwachung} (von Schwachpunkten) und
  hilfreiche \textbf{Beratung} des Vorstands
\item
  \textbf{Kontakt} zwischen AN und AG wird gefördert, AN werden als
  wichtiger Produktionsfaktor stärker an das Unternehmen gebunden
\end{itemize}

Eine \textbf{Vermeidung der Mitbestimmung} durch Unternehmen wird häufig
beobachtet, Möglichkeiten dazu sind das „\textbf{klein bleiben''}
(\textless2.000AN), die \textbf{Teilung des Unternehmens}, der
\textbf{Wechsel von AG zu SE} vor Überschreitung der Schwelle oder das
einfache (unrechtmäßige) \textbf{Ignorieren der Mitbestimmung}

\textbf{Fallanwendung? }

\begin{itemize}
\tightlist
\item
  Prüfe ob Mitbestimmung
\item
  Prüfe ob Unternehmensbranche hohe firmenspezifische Investitionen
  verlangt? Falls ja auf Hold Up Probleme hinweisen
\item
  Beschreibe Mitbestimmung als Lösungsmöglichkeit des Hold Up Problems
  und erwähne die weiteren Vorteile einer AN Vertretung im AR (Kontakt,
  Überwachung und Beratung)
\end{itemize}

\hypertarget{voraussetzungen-fuxfcr-ar-mitglieder}{%
\section{Voraussetzungen für AR
Mitglieder}\label{voraussetzungen-fuxfcr-ar-mitglieder}}

\hypertarget{kompetenzprofil}{%
\subsection{Kompetenzprofil}\label{kompetenzprofil}}

\textbf{Einzelnes AR Mitglied:}

\begin{itemize}
\item
  Natürliche Person mit unbeschränkter Geschäftsfähigkeit
\item
  Voraussetzung zur persönlichen und eigenverantwortlichen Ausübung der
  AR Tätigkeit: Grundlegende Kenntnisse der Rechte und Pflichten aller
  Unternehmensorgane und grundlegende Kenntnisse der
  Betriebswirtschaftslehre (jedes Mitglied)
\item
  Spezifische Kenntnisse leiten sich aus dem konkreten Aufgabenbereich
  des einzelnen AR Mitglieds ab und müssen nicht zwingend bei jedem AR
  Mitglied vorhanden sein

  \begin{itemize}
  \tightlist
  \item
    Kenntnisse des Personalwesens für Auswahl des Vorstands
  \item
    Kenntnisse des Bilanzrechts für Rechnungslegungskontrolle
  \item
    Controlling Kenntnisse zur Beurteilung der Wirtschaftlichkeit etc.
  \end{itemize}
\end{itemize}

\textbf{Fallanwendung}?

\begin{itemize}
\tightlist
\item
  Stelle fest, dass alle AR Mitglieder natürliche Personen sind mit
  uneingeschränkter Geschäftsfähigkeit
\item
  Prüfe ob alle Mitglieder grundlegende Kenntnisse der Rechte und
  Pflichten der Unternehmensorgane sowie grundlegende Kenntnisse der BWL
  zur persönlichen und eigenverantwortlichen Ausübung der AR Tätigkeit
  besitzen
\end{itemize}

\textbf{Spezifische Anforderung für AR Mitglieder des
Prüfungsausschusses gemäß AktG §100 V}

\begin{itemize}
\tightlist
\item
  Mindestens 1 Mitglieder verfügt über Sachverstand auf dem Gebiet der
  Rechnungslegung oder Abschlussprüfung
\item
  Prüfungsausschuss-Mitglieder müssen insgesamt mit dem Sektor vertraut
  sein, in dem das Unternehmen tätig ist (z.B. Erfahrung in
  Automobilbranche für Automobilhersteller)
\end{itemize}

Zur Sicherstellung eines AR, der insgesamt über alle erforderlichen
Spezialkompetenzen verfügt, sollen gemäß \textbf{DCGK Empfehlungen
Kompetenzprofile} für jedes einzelne Mitglied erarbeitet werden

Ziele bei Gesamtzusammensetzung des AR \textbf{gemäß DCGK}:

\begin{itemize}
\tightlist
\item
  Angemessene Gesamtanzahl an AR Mitgliedern
\item
  Alters- und Zugehörigkeitsgrenzen
\item
  Sicherstellung von Diversität (Geschlecht, Alter, Beruflicher
  Hintergrund, National/Internationale Zusammenstellung)
\end{itemize}

Determinanten der \textbf{Kompetenzprofile} einzelner Mitglieder des AR
\textbf{gemäß DCGK}:

\begin{itemize}
\tightlist
\item
  Kenntnisse
\item
  Fähigkeiten
\item
  Erfahrungen
\end{itemize}

\textbf{Grundidee der AR Kompetenzprofile?}

\begin{itemize}
\tightlist
\item
  Unternehmen soll grundsätzlich überlegen, welche Kenntnisse,
  Fähigkeiten und Erfahrungen der AR in seiner Gesamtzusammensetzung
  aufweisen sollte
\item
  Abgleich der aktuellen Profile der AR Mitgliedern mit Soll
  Kompetenzprofilen
\item
  Bei Abweichung soll im Zuge der nächsten AR Wahl die lückenhaften
  Kompetenzprofile ergänzt werden, um so langfristig die gewünschte
  Gesamtzusammenstellung des AR zu erreichen
\item
  Nicht jedes AR Mitglied muss alle Kompetenzen aufweisen, es geht immer
  um die Gesamtzusammensetzung des AR
\end{itemize}

\hypertarget{todo}{%
\subsection{TODO}\label{todo}}

Mit Folie 222 im neuen Skript abgleichen und ggf. aktualisieren

\hypertarget{unabhuxe4ngigkeit-des-ar}{%
\subsection{Unabhängigkeit des AR}\label{unabhuxe4ngigkeit-des-ar}}

Die ordnungsgemäße Ausübung seiner Überwachungsfunktion (ex post und ex
ante inklusive Beratungsfunktion) erfordert vom AR neben der
\textbf{Kompetenz} ebenfalls die \textbf{Unabhängigkeit}

\textbf{Kompetenz: }Ermöglicht die Analyse des Vorstandshandel auf
Rechtmäßigkeit, Ordnungsmäßigkeit und Wirtschaftlichkeit

\textbf{Unabhängigkeit}: Angemessene Reaktion auf festgestellte Probleme
im Interesse der Gesamtgesellschaft

\textbf{Rechtliche Vorgaben zur Unabhängigkeit gemäß AktG}

Unabhängigkeit wird insbesondere anhand der institutionelle Trennung
definiert:

\begin{enumerate}
\def\labelenumi{\arabic{enumi}.}
\item
  Vorstandsamt und Aufsichtsratsamt innerhalb der gleichen Gesellschaft
  ist nicht möglich

  \begin{enumerate}
  \def\labelenumii{\alph{enumii}.}
  \tightlist
  \item
    Gilt für Konzerne und Tochtergesellschaften (abhängige
    Gesellschaften) ebenfalls
  \end{enumerate}
\item
  Verbot von Überkreuzprüfungen (A ist Vorstand in Unternehmen X und AR
  in Unternehmen Y, B ist AR in Unternehmen X und Vorstand in
  Unternehmen Y)
\end{enumerate}

Trotzdem besteht die Gefahr von Unabhängigkeiten durch
Interessenkonflikte (z.B., wenn AR Mitglied in enger Beziehung zu
Vorstand oder bestimmten Aktionärsgruppen steht)

\textbf{Empfehlungen nach DCGK}

\begin{itemize}
\tightlist
\item
  Prinzipiell muss nicht jedes einzelne AR Mitglied unabhängig sein
\item
  \textbf{Mehr als die Hälfte der Anteilseignervertreter sind unabhängig
  (von Gesellschaft und Vorstand)}
\item
  \textbf{Vorsitzender des AR, des Prüfungsausschuss und des
  Vergütungsausschuss sind unabhängig}
\end{itemize}

Unabhängigkeit gemäß DCGK: AR Mitglied steht in \textbf{keiner
persönlichen oder geschäftlichen Beziehung zu der Gesellschaft und dem
Vorstand} durch die ein \textbf{wesentlicher} und \textbf{nicht nur
vorübergehender Interessenskonflikt} begründet wird

Prüfung der Unabhängigkeit erfolgt anhand von Indikatoren, die für jedes
AR Mitglied zu überprüfen sind

\textbf{DCGK 4 Indikatoren der Unabhängigkeit}

\begin{enumerate}
\def\labelenumi{\arabic{enumi}.}
\item
  Persönliche Beziehung: AR Mitglied ist naher Familienangehöriger eines
  Vorstandsmitglieds
\item
  Tätigkeit im AR seit über 12 Jahren
\item
  AR Mitglied ist direkt oder als Gesellschafter an einem anderen
  Unternehmen beteiligt, dass wesentliche Geschäftsbeziehungen zu dem
  Unternehmen unterhält (aktuell oder bis zum Jahr der Ernennung als AR
  unterhalten hat)
\item
  AR Mitglied war innerhalb der zwei Jahre vor seiner Ernennung selbst
  Vorstandsmitglied der betroffenen Gesellschaft
\end{enumerate}

\begin{verbatim}
a.  Nur in Ausnahmefällen ist nach **AktG** ein Wechsel von Vorstand
    zu AR innerhalb von 2 Jahren möglich (Wahlvorschlag durch
    Aktionäre mit mindestens 25% Stimmanteilen als Ausnahme)  
    2-jährige Cooling Off Periode im Normalfall
b.  Falls Ausnahmefall in jedem Fall als abhängig einzustufen,
    maximal 2 frühere Vorstandsmitglieder sind im AR nach DCGK
    einzusetzen
\end{verbatim}

Erfüllung eines Indikators deutet auf fehlende Unabhängigkeit hin, führt
aber nicht zwingend dazu. Falls Mitglied trotz Erfüllung als unabhängig
aufgeführt werden soll, muss dies Erläutert und begründet werden

\textbf{Ergänzende Empfehlungen nach DCGK}

\begin{itemize}
\tightlist
\item
  Falls das Unternehmen einen kontrollierenden Aktionär{[}\^{}3{]} hat,
  soll mindestens ein AR Mitglied unabhängig von diesem sein (falls 6+
  Mitglieder im AR dann sollen mindestens 2 AR Mitglieder unabhängig vom
  kontrollierenden Aktionär sein)
\item
  Offenlegung von Anzahl und Namen aller als unabhängig eingestufter AR
  Mitglieder offenlegen
\end{itemize}

\textbf{Verhältnis von abhängigen und unabhängigen AR Mitgliedern?}

\begin{itemize}
\tightlist
\item
  Unternehmen soll eigenständig eine angemessene Anzahl festlegen
\item
  Mindestens die Hälfte der Anteilseignervertreter sollen unabhängig
  sein falls es \textbf{keinen kontrollierenden Aktionär} gibt
\item
  Weniger als die Hälfte des AR unabhängig ist möglich für den Fall,
  dass es einen kontrollierenden Aktionär gibt
\end{itemize}

\textbf{Fallanwendung}?

\begin{itemize}
\item
  Prüfe ob die gesetzliche Unabhängigkeitsanforderungen erfüllt sind

  \begin{itemize}
  \tightlist
  \item
    AR Mitglieder sind nicht gleichzeitig Vorstandsmitglieder (gilt auch
    für abhängige Gesellschaften)
  \item
    Früher Vorstandsmitglieder erfüllen die 2-jährige Cool -Off
    Wartefrist (Ausnahme bei 25\% Stimmen der Hauptversammlung)
  \item
    Keine Überkreuzbeziehungen in dem AR Mitglied und Vorstandsmitglied
    in zwei Gesellschaften über Kreuz sich gegenseitig überwachen
  \end{itemize}
\item
  Einstufung der Unabhängigkeit der Anteilsvertreter im AR --
  Unabhängigkeit im Sinne von keinen persönlichen oder geschäftlichen
  Beziehungen zu Vorstand, Gesellschaft oder Mehrheitsaktionären
\item
  Falls keine Mehrheitsaktionäre vorhanden sind, sollte mindestens die
  Hälfte der Anteilseigner-Vertreter unabhängig sein (falls
  Mehrheitsaktionär vorhanden ist, auch geringerer Anteil möglich)
\item
  Prüfe die Unabhängigkeit der einzelnen AR Mitglieder anhand der 4
  Indikatoren der Unabhängigkeit gemäß DCGK:
\end{itemize}

\begin{verbatim}
-   Persönliche Beziehungen in Form von Familienzugehörigkeit
    zwischen AR Mitglied und Vorstandsmitgliedern
-   Tätigkeit im AR bereits seit mehr als 12 Jahren
-   AR Mitglied ist direkt oder als Gesellschafter an anderem
    Unternehmen beteiligt, dass wesentliche Geschäftsbeziehungen zu
    seinem Unternehmen unterhält
-   AR war innerhalb der letzten 2 Jahre nicht selbst
    Vorstandsmitglied
\end{verbatim}

\textbf{Trade Off -- Wechsel von Vorstand in AR}

\begin{itemize}
\tightlist
\item
  Grundsätzlich erst nach 2-jähriger Wartezeit möglich (Ausnahme
  Vorschlag mit 25\% Aktionärsstimmrechten)
\item
  Wechsel innerhalb der 2 Jahre ist nach DCGK als Unabhängigkeit
  einzustufen
\item
  Contra Off: Ehemaliges Vorstandsmitglied überwacht im AR teilweise
  sein eigenes Vorstandshandeln in der Vergangenheit und kann daher
  nicht als unabhängig betrachtet werden
\item
  Pro: Ehemaliges Vorstandsmitglied bringt sehr viele firmenspezifische
  Kenntnisse in den AR mit ein.
\item
  Trade Off zwischen Unabhängigkeit und hoher spezifischer Kompetenz
\end{itemize}

\textbf{Interessenskonflikte}

Neben den Beziehungen zwischen AR und Vorstand bzw. kontrollierenden
Aktionären können AR Mitglieder auch mit anderen Stakeholdergruppen in
Beziehungen stehen

\begin{itemize}
\tightlist
\item
  Kunden
\item
  Lieferanten
\item
  Kreditgeber
\item
  Sonstige Stakeholder
\end{itemize}

Gefahr: AR Mitglieder handeln im Sonderinteresse ihrer Beziehungen
anstelle des allgemeinen Unternehmensinteresse

Besonders da die Mitgliedschaft im AR ein Nebenamt darstellt, treten
häufig Interessenkonflikte zwischen dem Hauptamt und der AR-Tätigkeit
auf

DCGK Empfehlungen bei Auftraten von Interessenskonflikten

\begin{itemize}
\item
  Offenlegung gegenüber dem AR durch betroffenes Mitglied
\item
  Bericht und Umgang gegenüber der HV schildern
\item
  Bei Entscheidungen, die durch den Interessenskonflikt beeinflusst
  werden könnten, soll sich das betroffene AR Mitglied enthalten
\item
  Grundsätzlich kein sofortiger Grund zur Amtsniederlegung, jedoch
  Empfehlung zur Amtsniederlegung bei:

  \begin{itemize}
  \tightlist
  \item
    Wesentlichen oder permanenten Interessenskonflikten
  \item
    Aufgabe der Beratung der Unternehmensleitung bei wesentlichen
    Wettbewerbsentscheidungen
  \end{itemize}
\end{itemize}

\hypertarget{zeiteinsatz-der-ar-mitglieder}{%
\section{Zeiteinsatz der AR
Mitglieder}\label{zeiteinsatz-der-ar-mitglieder}}

Grundsätzlich ist AR Mitgliedschaft eine Nebentätigkeit

Minimum des Zeitaufwands in börsennotierten Gesellschaften: 2 Sitzungen
pro Halbjahr (inklusive 2 Tage für Vor- und Nachbearbeitung)

Durchschnitt für DAX/MDAX Unternehmen liegt bei 3 Sitzungen pro Halbjahr

Zeitaufwand geschätzt zwischen 12-18 Tage

Zusätzlicher Aufwand für Ausschusssitzungen oder Aufsichtsratsvorsitz,
zusätzliche Kommunikation mit Vorstand oder anderen Stakeholdern

Gesetzlicher Schutz vor Overboarding: Damit die Sorgfaltspflicht bei der
Ausübung des AR Amt nicht verletzt wird gilt eine Beschränkung auf
\textbf{die maximale Ausübung von 10 Aufsichtsratsmitgliedschaften} in
verpflichtenden AG oder GmbH Ars gemäß AktG (bis zu 5 AR Ämter innerhalb
eines Konzerns werden jedoch als Einzelamt gewertet und der
Aufsichtsratsvorsitz wird doppelt gewertet)

Ziel der Reglung: Ausreichende Arbeits- und Zeiteinsatz der AR
Mitglieder sicherstellen, jedoch viele Schlupflöcher (mehrere Ämter in
Konzern, ausländische Gesellschaften oft nicht definiert, etc.)

Sicherstellung durch AktG ist fragwürdig

Ergänzende Empfehlungen des DCGK: Sicherstellung ausreichendes
Zeiteinsatzes

\begin{itemize}
\tightlist
\item
  Vorstandsmitglieder sollen sich auf zwei externe AR Mitgliedschaften
  beschränken
\item
  Vorstandsmitglieder sollen nicht den AR Vorsitz in anderer
  börsennotierter Gesellschaft übernehmen
\item
  Maximal 5 AR Ämter (inklusive vergleichbarer Posten im Ausland)
  zeitgleich für sämtliche AR Mitglieder die kein Vorstandsamt bekleiden
\item
  Transparenz durch Offenlegung wie oft jedes AR Mitglied an Sitzungen
  und Ausschüssen teilgenommen hat
\end{itemize}

\hypertarget{verguxfctung-des-ar}{%
\subsection{Vergütung des AR}\label{verguxfctung-des-ar}}

Vergütung für AR Mitglieder wird i.d.R. in der Unternehmenssatzung
festgelegt oder durch die HV beschlossen

\textbf{Reglung des AktG}: Angemessene Vergütung von AR Amt in
Anbetracht von

\begin{enumerate}
\def\labelenumi{\arabic{enumi}.}
\tightlist
\item
  Aufgaben des AR Mitglieds
\item
  Lage der Gesellschaft (insbesondere Größe z.B.
  SDAX\textless MDAX\textless DAX für Median AR-Vergütung)
\end{enumerate}

Ergänzende DCGK Empfehlungen

\begin{itemize}
\tightlist
\item
  Berücksichtigung von Aufgaben, Vorsitz und Mitgliedschaft in
  Ausschüssen
\item
  Empfehlung von fixer Vergütung, falls variable Bestandteile sollen
  diese auf nachhaltige Unternehmensentwicklung beschränkt sein
\end{itemize}

\hypertarget{funktion-der-hauptversammlung-kurzform}{%
\section{Funktion der Hauptversammlung
(Kurzform)}\label{funktion-der-hauptversammlung-kurzform}}

\begin{enumerate}
\def\labelenumi{\arabic{enumi}.}
\tightlist
\item
  Die HV ermöglicht es den Aktionären die \textbf{koordinierte Ausübung
  ihrer Stimmrechte} \textbf{gewichtet nach dem Anteil ihres
  Aktienbesitzes} auszuüben. \emph{Direkt} stimmen die Aktionäre über
  die \textbf{Bestellung der AR Mitglieder und des Abschlussprüfers} ab
  (indirekt kann die HV dadurch auch Einfluss auf die
  \textbf{Unternehmensleitung} ausüben indem die \textbf{Überwachung}
  \textbf{dieser \emph{delegiert}} wird). Einen direkten Einfluss auf
  unternehmerische Entscheidungen kann die HV nur im Rahmen der
  \textbf{Gewinnverwendung (Höhe der Dividendenzahlung}) sowie bei
  \textbf{wesentlichen Strukturentscheidungen} nehmen. Weiterhin können
  Aktionäre im Rahmen der Billigung von Vergütungsmechanismen und der
  Entlastung des Vorstands ihre \emph{Meinung} zum Ausdruck bringen
\end{enumerate}

\textbf{Daten zur HV}: Jährlicher Turnus, Leitung durch AR Vorsitzenden,
DCGK Empfehlungen: 4-6 Stunden maximale Dauer

\textbf{Direkte Einflussnahme der HV}: Verwendung des Bilanzgewinns,
grundlegende Strukturentscheidungen (Satzungsänderungen,
Kapitalbeschaffung und Herabsetzung, Auflösung der Gesellschaft)

\textbf{Delegation der Einflussnahme durch HV: }Bestellung von
Aufsichtsrat, Abschlussprüfer und falls notwendig auch Sonderprüfern

\textbf{Billigung (=Meinungsäußerungen der HV): }Entlastung von Vorstand
und AR, Votum über Vergütungsystem

\textbf{Stimmrechte}: 1 Stimme pro Aktie, grundsätzlich einfache
Mehrheit jedoch qualifizierte Mehrheit (ab 75\%) für Satzungsänderungen,
Vorstand ist verpflichtet zur Umsetzung, Wahloptionen je nach Thema
durch AR oder Vorstand

\begin{enumerate}
\def\labelenumi{\arabic{enumi}.}
\tightlist
\item
  Die Hauptversammlung beinhaltet eine \textbf{Diskussion der
  Tagesordnungspunkte}, Aktionäre besitzen dabei ein \textbf{Rede- und
  Auskunftsrecht. Vorstand hat Fragen zu beantworten} unter der
  Bedingung, dass diese mit dem entsprechenden
  \textbf{Tagesordnungspunkt zusammenhängen} und dass das
  \textbf{Unternehmen keinen Schaden durch die Beantwortung erleidet}.
  Aktionäre können außerdem \textbf{Gegenanträge zur Tagesordnung}
  stellen und bei einer ausreichenden \textbf{Mehrheit die Tagesordnung
  um weitere Punkte erweitern}. Auch eine \textbf{außerordentliche
  Hauptversammlung} kann bei Mehrheiten einberufen werden
\end{enumerate}

Rede- und Auskunftsrecht gemäß AktG: Rederecht der Aktionäre darf
zeitlich angemessen beschränkt werden, Auskunftsrecht gliedert sich
folgendermaßen:

\begin{itemize}
\tightlist
\item
  Verantwortlichkeit: Vorstand beantwortet primär Fragen, andere
  Parteien können ergänzen unter Einstimmung des Vorstands
\item
  Reichweite: Frage bezieht sich auf Tagesordnungspunkt, weiter Rahmen
  der Fragen möglich jedoch auch Begrenzungen durch Vorstand möglich
\item
  Verweigerungsrecht des Vorstands: Bei Schaden für die Gesellschaft
\end{itemize}

Initiativrecht der Aktionäre: Gegenanträge der Aktionäre, Ergänzung der
Tagesordnung durch Aktionärsgruppe mit 5\% Grundkapital oder 500 000
Euro Aktionärskapital (z.B. Beauftragung von Sonderprüfern), Einberufung
einer außerordentliche HV von Aktionärsgruppe mit 5\% Grundkapital

\begin{enumerate}
\def\labelenumi{\arabic{enumi}.}
\tightlist
\item
  Für Kleinaktionäre ist die Ausübung des Stimmrechts i.d.R. irrational,
  da der Aufwand, den für sie zu erwartetem Nutzen, übersteigt. Häufig
  wird eine Art von „Durchwinken'' der Tagesordnungspunkte der bei
  Abstimmungen der HV beobachtet. Negative Meinungsäußerungen und
  geringe Zustimmungsquoten der Aktionäre können jedoch großen Einfluss
  auf die Unternehmensleitung haben und werden von diesen regelmäßig
  sehr ernst genommen
\end{enumerate}

Rationale Apathie: Geringer Anreiz zum Engagement durch Kleinaktionäre.
Geringer Nutzen des einzelnen Kleinaktionärs steht hohen Kosten für
Beschaffung und Auswertung von relevanten Informationen gegenüber

Zustimmungsraten: Erreichen von einfachen Mehrheiten gelingt oft
(durchwinken) und unmittelbarer Einfluss auf Vorstand ist durch
Aktionäre hier gering. Jedoch erzeugen niedrige Zustimmungsraten auch
bei durchgewunkenen Entscheidungen einen hohen Druck auf den Vorstand
(niedrige Zustimmungsraten führen z.B. oft zeitnah zur Ablösung des
Vorstands oder einer Verringerung der Vergütung des Vorstands)

Aktionäre übe rn geringen unmittelbaren Einfluss aus aber durchaus hohen
mittelbaren, indirekten Einfluss

\begin{verbatim}
Gesperrte Aktien dürfen über eine bestimmte Frist vom Besitzer
nicht veräußert werden, der Wert der Vergütung hängt demnach von der
aktuellen als auch von der zukünftigen Entwicklung der
Unternehmenskapitalisierung ab

Gesperrte Aktien dürfen über eine bestimmte Frist vom Besitzer
nicht veräußert werden, der Wert der Vergütung hängt demnach von der
aktuellen als auch von der zukünftigen Entwicklung der
Unternehmenskapitalisierung ab
\end{verbatim}

Kontrollierende Aktionäre sind zumindest Mehrheitsaktionäre

\hypertarget{governance-in-public-private-partnerships-ans-ngos}{%
\subsection{Governance in Public-Private Partnerships ans
NGOs}\label{governance-in-public-private-partnerships-ans-ngos}}

Main Topic: What are the challenges and differences of (Corperate)
Governance in public sector organizations and NGOs?

\hypertarget{governance-in-the-public-sector-p1}{%
\paragraph{Governance in the public sector
(p1)}\label{governance-in-the-public-sector-p1}}

What is the public sector?

\begin{itemize}
\tightlist
\item
  Governmental controlled structures,
\item
  public utilities (e.g.~public transportation services)
\item
  tax funded
\end{itemize}

What about public money?

\begin{itemize}
\tightlist
\item
  mainly collected from taxes
\item
  little competitve pressure
\item
  large scale
\end{itemize}

Why the need for CG in public sector? \textgreater{} ensure, that
organizations fulfill their job and purpose to the public that largely
funds them

Important elements in public sector governance include:

\begin{enumerate}
\def\labelenumi{\arabic{enumi}.}
\tightlist
\item
  internal structures (boards, sub-committees, ..)
\item
  external accountability (financial budgets, reports, ..)
\item
  code of conduct for members
\item
  open and transparent to the public
\end{enumerate}

Differences to private sector Governance?

\begin{itemize}
\tightlist
\item
  Private:

  \begin{enumerate}
  \def\labelenumi{\arabic{enumi}.}
  \tightlist
  \item
    Stakeholders as the main focus group for accountability
  \item
    ``classic'' agency problem
  \end{enumerate}
\item
  public sector:

  \begin{enumerate}
  \def\labelenumi{\arabic{enumi}.}
  \tightlist
  \item
    Agency problem with different properties, principals are very divers
    group made of members of society
  \item
    Who are the main people that public sector organizations should be
    accountable to?
  \item
    Constanly changing environment (stupid slide)
  \end{enumerate}
\end{itemize}

\hypertarget{npm-reforms-of-1980s}{%
\subparagraph{NPM Reforms of 1980s}\label{npm-reforms-of-1980s}}

\begin{itemize}
\tightlist
\item
  New Public Management (=NPM) is a managemnt philosophy that is getting
  more adapted by publi organizations since the 1980.
\item
  Aim: Enhance Efficiency, Productivity and Quality of public
  organizations
\item
  Focus: Market orientations and reforms to mirror the management and
  organization style of private sector companies
\end{itemize}

\hypertarget{governance-in-ngos}{%
\paragraph{Governance in NGOs}\label{governance-in-ngos}}

What are NGOs?

\begin{itemize}
\tightlist
\item
  non-profit organizations: surplus doesnt get paid to owners
\item
  varying sizes and power
\item
  seperate entities that are not part of the governmental structure
\item
  formal organizazions like companies, public sector entities etc
\end{itemize}

Governance of NGOS?

Similiar questions as in public sector organizations, 1. Who are the
shareholders? 2. how to measure accountability?

Stake/Shareholders are very divers as :well:

\begin{itemize}
\tightlist
\item
  Donors/ Funders
\item
  Staff and volunteers
\item
  regulators
\item
  governments
\item
  people that benefit from ngos direct actions
\item
  \ldots{}
\end{itemize}

Measuring Success for NGOs ?

\begin{itemize}
\tightlist
\item
  Difficult since the metric is outside of the organization (much easier
  for private companies)
\item
  Hard to collect and interpret data for accountability in many cases
\item
  Different metrics are used but flawd

  \begin{itemize}
  \tightlist
  \item
    cost per total expenditure: how many percent of a donation are spend
    on aim, costs, \ldots{}
  \item
    SROI - Social Return on Investment: Estimates socials value
    generated dollar equivalent
  \end{itemize}
\end{itemize}

\end{document}
