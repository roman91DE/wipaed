% Options for packages loaded elsewhere
\PassOptionsToPackage{unicode}{hyperref}
\PassOptionsToPackage{hyphens}{url}
%
\documentclass[
  10pt,
  ignorenonframetext,
]{beamer}
\usepackage{pgfpages}
\setbeamertemplate{caption}[numbered]
\setbeamertemplate{caption label separator}{: }
\setbeamercolor{caption name}{fg=normal text.fg}
\beamertemplatenavigationsymbolsempty
% Prevent slide breaks in the middle of a paragraph
\widowpenalties 1 10000
\raggedbottom
\setbeamertemplate{part page}{
  \centering
  \begin{beamercolorbox}[sep=16pt,center]{part title}
    \usebeamerfont{part title}\insertpart\par
  \end{beamercolorbox}
}
\setbeamertemplate{section page}{
  \centering
  \begin{beamercolorbox}[sep=12pt,center]{part title}
    \usebeamerfont{section title}\insertsection\par
  \end{beamercolorbox}
}
\setbeamertemplate{subsection page}{
  \centering
  \begin{beamercolorbox}[sep=8pt,center]{part title}
    \usebeamerfont{subsection title}\insertsubsection\par
  \end{beamercolorbox}
}
\AtBeginPart{
  \frame{\partpage}
}
\AtBeginSection{
  \ifbibliography
  \else
    \frame{\sectionpage}
  \fi
}
\AtBeginSubsection{
  \frame{\subsectionpage}
}
\usepackage{amsmath,amssymb}
\usepackage{lmodern}
\usepackage{iftex}
\ifPDFTeX
  \usepackage[T1]{fontenc}
  \usepackage[utf8]{inputenc}
  \usepackage{textcomp} % provide euro and other symbols
\else % if luatex or xetex
  \usepackage{unicode-math}
  \defaultfontfeatures{Scale=MatchLowercase}
  \defaultfontfeatures[\rmfamily]{Ligatures=TeX,Scale=1}
\fi
\usetheme[]{Szeged}
\usecolortheme{beaver}
% Use upquote if available, for straight quotes in verbatim environments
\IfFileExists{upquote.sty}{\usepackage{upquote}}{}
\IfFileExists{microtype.sty}{% use microtype if available
  \usepackage[]{microtype}
  \UseMicrotypeSet[protrusion]{basicmath} % disable protrusion for tt fonts
}{}
\makeatletter
\@ifundefined{KOMAClassName}{% if non-KOMA class
  \IfFileExists{parskip.sty}{%
    \usepackage{parskip}
  }{% else
    \setlength{\parindent}{0pt}
    \setlength{\parskip}{6pt plus 2pt minus 1pt}}
}{% if KOMA class
  \KOMAoptions{parskip=half}}
\makeatother
\usepackage{xcolor}
\IfFileExists{xurl.sty}{\usepackage{xurl}}{} % add URL line breaks if available
\IfFileExists{bookmark.sty}{\usepackage{bookmark}}{\usepackage{hyperref}}
\hypersetup{
  pdftitle={Betriebspraktikum - M.Sc. Wirtschaftspädagogik},
  pdfauthor={Roman Hoehn},
  hidelinks,
  pdfcreator={LaTeX via pandoc}}
\urlstyle{same} % disable monospaced font for URLs
\newif\ifbibliography
\usepackage{graphicx}
\makeatletter
\def\maxwidth{\ifdim\Gin@nat@width>\linewidth\linewidth\else\Gin@nat@width\fi}
\def\maxheight{\ifdim\Gin@nat@height>\textheight\textheight\else\Gin@nat@height\fi}
\makeatother
% Scale images if necessary, so that they will not overflow the page
% margins by default, and it is still possible to overwrite the defaults
% using explicit options in \includegraphics[width, height, ...]{}
\setkeys{Gin}{width=\maxwidth,height=\maxheight,keepaspectratio}
% Set default figure placement to htbp
\makeatletter
\def\fps@figure{htbp}
\makeatother
\setlength{\emergencystretch}{3em} % prevent overfull lines
\providecommand{\tightlist}{%
  \setlength{\itemsep}{0pt}\setlength{\parskip}{0pt}}
\setcounter{secnumdepth}{-\maxdimen} % remove section numbering
\ifLuaTeX
  \usepackage{selnolig}  % disable illegal ligatures
\fi

\title{Betriebspraktikum - M.Sc. Wirtschaftspädagogik}
\subtitle{Kölner Wirtschaftsfachschule - Wifa Gruppe - GmbH}
\author{Roman Hoehn}
\date{1/6/2022}

\begin{document}
\frame{\titlepage}

\begin{frame}[allowframebreaks]
  \tableofcontents[hideallsubsections]
\end{frame}
\hypertarget{erfahrungsbereich-unternehmen-und-beruf}{%
\section{Erfahrungsbereich Unternehmen und
Beruf}\label{erfahrungsbereich-unternehmen-und-beruf}}

\begin{frame}{Erfahrungsbereich Unternehmen und Beruf}
\begin{figure}
\centering
\includegraphics{pics/wifa_logo.png}
\caption{Kölner Wirtschaftsfachschule - Logo}
\end{figure}
\end{frame}

\begin{frame}{Unternehmensprofil}
\protect\hypertarget{unternehmensprofil}{}
\begin{itemize}
\tightlist
\item
  Privater Anbieter von Aus- und Weiterbildungsdienstleistungen
\item
  Deutschlandweite 16 Schulungszentren mit ca. 120 festen Mitarbeitern
\item
  Offizieler Bildungspartner von u.a.:

  \begin{enumerate}
  \tightlist
  \item
    Microsoft Corporation
  \item
    DATEV Software \& Consulting
  \item
    FSGU Akademie
  \end{enumerate}
\end{itemize}
\end{frame}

\begin{frame}{Geschäftsbereiche}
\protect\hypertarget{geschuxe4ftsbereiche}{}
\begin{enumerate}
\tightlist
\item
  AVGS - Staatlich finanzierte Arbeitsvermittlung und/oder
  Arbeitserhaltungsmaßnahmen
\item
  Berufliche Rehabilitation, insbesondere als Träger für die deutsche
  Rentenversicherung
\item
  Berufsbegleitende Weiterbildungen im Rahmen der
  Personalentwicklung(``near the job'')
\end{enumerate}
\end{frame}

\begin{frame}{Aus-und Weiterbildungsportfolio}
\protect\hypertarget{aus-und-weiterbildungsportfolio}{}
\begin{itemize}
\tightlist
\item
  Umschulungen in insgesamt 8 statlich anerkannten Ausbildungsberufen
\item
  Berufliche Teilqualifikationen (MS Office und Datev)
\item
  Kooperationspartner für IHK Zertifizierungsprüfungen
\end{itemize}
\end{frame}

\begin{frame}{Aufbauorgaisation und Unternehmensstruktur}
\protect\hypertarget{aufbauorgaisation-und-unternehmensstruktur}{}
\begin{itemize}
\tightlist
\item
  Vor 2020: Dezentralisiertes Struktur, weitgehenst autarke Verwaltung
  der Standorte unter Dachverwaltung
\item
  Seit 2020: Strategische Neuorientierung aufgrund der neuen
  Rahmenbedingungen der Sars-Cov 2 Epidemie
\end{itemize}
\end{frame}

\begin{frame}{Betriebliche Handlungsalternativen}
\protect\hypertarget{betriebliche-handlungsalternativen}{}
\begin{itemize}
\tightlist
\item
  Privatwirtschaftlich, d.h. primär profitorientierte Ausrichtung des
  Unternehmens
\item
  Zielkonflikt zu pädagogischen/sozialen Verantwortung?
\item
  Primäre (innerbetriebliche) Erfolgskennzahl: Arbeitsvermittlungsquote!

  \begin{itemize}
  \tightlist
  \item
    Anreiz zur Vermittlung in Niedriglohnsektor und
    Zeitarbeitsverhältnisse
  \item
    Keine nachhaltig/langfristige Ausrichtung
  \item
    Gesamtökonomischer Effekt fragwürdig
  \end{itemize}
\end{itemize}
\end{frame}

\begin{frame}{Betriebliche Anforderungen}
\protect\hypertarget{betriebliche-anforderungen}{}
\begin{itemize}
\tightlist
\item
  Fachliche Expertise
\item
  Didaktische/Pädagogische Fähigkeiten
\item
  Zeitmanagement
\end{itemize}
\end{frame}

\hypertarget{betriebliche-handlungsalternativen-1}{%
\section{Betriebliche
Handlungsalternativen}\label{betriebliche-handlungsalternativen-1}}

\begin{frame}{Ausgangssituation}
\protect\hypertarget{ausgangssituation}{}
\begin{itemize}
\tightlist
\item
  Betrieb im Wandel: Erarbeitung eines digitalen Schulungskonzepts
\item
  \textbf{Vorher}: Standorte arbeiten stark autonom:

  \begin{enumerate}
  \tightlist
  \item
    Erarbeitung eigener Unterrichts/Unterweisungskonzepte
  \item
    Wenig Kontakt zu anderen Geschäftsstellen
  \item
    Fokus auf individuelle Betreuung
  \end{enumerate}
\item
  \textbf{Zielvorgabe}: Standortübergreifende Zusammenarbeit bei der
  Erstellung eines digitalen Unterrichtsangebots
\end{itemize}
\end{frame}

\begin{frame}{Schwierigkeiten}
\protect\hypertarget{schwierigkeiten}{}
\begin{itemize}
\tightlist
\item
  Auftrag der Geschäftsleitung zur Umstellung ohne genaue Ziel- und
  Rahmenvorgaben
\item
  Mangelhafte Kommunikation- und Vernetzung zwischen Geschäftsstellen
\item
  Keine Erfahrung in der digitalen Unterrichtsgestaltung
\item
  Moral Hazard: ``Die anderen machen das schon\ldots{}''
\end{itemize}
\end{frame}

\begin{frame}{Lösungsvorschlage}
\protect\hypertarget{luxf6sungsvorschlage}{}
Welche Maßnahmen könnte die Geschäftsleitung ergreifen um das Problem

\begin{itemize}
\tightlist
\item
  in der \textbf{akuten} Lage,
\item
  \textbf{präventiv} für zukünftige Projekte
\end{itemize}

zu lösen?
\end{frame}

\begin{frame}{Betriebliche Lösungsvorschläge}
\protect\hypertarget{betriebliche-luxf6sungsvorschluxe4ge}{}
\begin{itemize}
\tightlist
\item
  Einführung eines wöchentlichen Meetings aller Beteiligten
\item
  Führung einer Skillmatrix: Erfassung von individuellen Kernkompetenzen
\item
  Ernennung von Ansprechpartnern und Verantwortlichen
\end{itemize}
\end{frame}

\hypertarget{unterweisungseinheit}{%
\section{Unterweisungseinheit}\label{unterweisungseinheit}}

\begin{frame}{Inhalt}
\protect\hypertarget{inhalt}{}
\begin{itemize}
\tightlist
\item
  Zielgruppe: Kaufmännische Umschüler
\item
  Thema: Verweis Funktionen in Microsoft Excel
\item
  Rahmenbedingungen: Online über Microsoft Teams
\end{itemize}
\end{frame}

\begin{frame}{Vorgehensweise}
\protect\hypertarget{vorgehensweise}{}
\begin{itemize}
\tightlist
\item
  Erstellung eines Zeitplans
\item
  Vorbereitung von interaktiven Übungsaufgaben
\item
  Einbezug von Visualisierungstools
\end{itemize}
\end{frame}

\begin{frame}{Schwierigkeiten}
\protect\hypertarget{schwierigkeiten-1}{}
\begin{itemize}
\tightlist
\item
  Fehlendes Feedback bei Online Unterweisungen
\item
  Aktivierung der Teilnehmer zur Teilnahme?

  \begin{enumerate}
  \tightlist
  \item
    Gruppenarbeitsphasen einsetzten?
  \item
    Vermehrter Einsatz von Online Tools?
  \item
    Ausweichen auf Video Formate?
  \end{enumerate}
\end{itemize}
\end{frame}

\begin{frame}{Ihre Meinung}
\protect\hypertarget{ihre-meinung}{}
Nach mehreren Semestern Erfahrungen mit Online Seminaren/Vorlesungen:

\begin{itemize}
\tightlist
\item
  welche Methoden halten Sie für sinnvoll?
\item
  Welche Methoden gefallen Ihnen nicht?
\item
  was macht eine gute Online Schulung aus?
\end{itemize}
\end{frame}

\end{document}
